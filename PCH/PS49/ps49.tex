\documentclass{article}

\usepackage[letterpaper,portrait,top=0.4in, left=0.6in, right=0.6in, bottom=1in]{geometry}

\usepackage{amsmath, amsfonts, amsthm, amssymb}
\usepackage{graphicx, float}
\usepackage{mathtools}
\usepackage{titlesec}
\usepackage{interval}
\usepackage{hyperref}
\usepackage{titling}
\usepackage{vwcol}
\usepackage{setspace}
\usepackage{empheq}
\usepackage{cancel}
\usepackage{esdiff}
\usepackage{multicol}
\usepackage{mdframed}
\usepackage{esdiff}
\usepackage{multicol}
\usepackage{tikz}
\usepackage{varwidth}

\intervalconfig {
	soft open fences
}

\newcommand{\alignedintertext}[1]{%
	\noalign{%
		\vskip\belowdisplayshortskip
		\vtop{\hsize=\linewidth#1\par
		\expandafter}%
		\expandafter\prevdepth\the\prevdepth
	}%
}

% \allowdisplaybreaks

%opening
\title{Problem Set \#49}
\author{Jayden Li}
\date{February 7, 2024}

\allowdisplaybreaks

\begin{document}
\setlength{\abovedisplayskip}{0pt}
\fontsize{12pt}{12pt}\selectfont
\maketitle

\section*{Problem 2}
The $x$-coordinate of every point on the parametric equation must satisfy $x\in\interval{1}{4}$ and the $y$-coordinate must satisfy $y\in\interval{2}{3}$. The points on the equation must be within this closed region.

\section*{Problem 3}
\begin{itemize}
\item[(a)]
	III. The range of $x$ is $\interval{1}{2}$ and III is the only graph where every coordinate has $x$-coordinate inside this interval.

\item[(d)]
	I. This is the only graph that passes through $(2,2)$, $(2,-2)$, $(-2,2)$ and $(-2,-2)$.

\item[(c)]
	IV. The graph for $y$ shows that $y\geq0$. IV is the only graph without a negative $y$-coordinate.

\item[(d)]
	II. This is the only curve with the same $x$ and $y$ intercepts as shown in the graph. 

\end{itemize}

\section*{Problem 4}
\vspace*{-15pt}
\begin{center}
	\includegraphics*[width=0.85\linewidth]{q4.png}
\end{center}

\section*{Problem 7}
\begin{itemize}
\item[(a)]
	Let $t=x$. $-3\leq x\leq2\implies-3\leq t\leq2$, $y=x^2\implies y=t^2$.
	\begin{equation*}
		\boxed{x(t)=t,y(t)=t^2,t\in\interval{-3}{2}}
	\end{equation*}

\item[(b)]
	Let $t=f^{-1}(x)\implies f(t)=x$. $y=f^{-1}(x)\implies y=t$.
	\begin{equation*}
		\boxed{x(t)=t^5+2t+1,y(t)=t}
	\end{equation*}

\item[(c)]
	\phantom{} \\
	$\begin{aligned}
		&x=x_0+(x_1-x_0)t \\
		&y=y_0+(y_1-y_0)t \\
		&0\leq t\leq1
	\end{aligned}\implies\begin{aligned}
		&x=2+(1-2)t \\
		&y=-3+(5-(-3))t \\
		&0\leq t\leq1
	\end{aligned}\implies\fbox{
		\begin{varwidth}{\dimexpr\textwidth-2\fboxsep-2\fboxrule\relax}
			$\begin{aligned}
				&x(t)=2-t \\
				&y(t)=-3+8t \\
				&0\leq t\leq1
			\end{aligned}$
		\end{varwidth}
	}$

\item[(d)]
	\begin{align*}
		x^2+2x+y^2-4y&=4 \\
		(x+1)^2-1+(y-2)^2-4&=4 \\
		(x+1)^2+(y-2)^2&=9
	\end{align*}
	$9=9\cos^2t+9\sin^2t$. Let $9\cos^2t=(x+1)^2,9\sin^2t=(y-2)^2$.
	
	$\begin{aligned}
		&9\cos^2t=(x+1)^2 \\
		&9\sin^2t=(y-2)^2 \\
		&0\leq t\leq2\pi
	\end{aligned}\implies\begin{aligned}
		&3\cos t=x+1 \\
		&3\sin t=y-2 \\
		&0\leq t\leq2\pi
	\end{aligned}\implies\fbox{
		\begin{varwidth}{\dimexpr\textwidth-2\fboxsep-2\fboxrule\relax}
			$\begin{aligned}
				&x(t)=3\cos t-1 \\
				&y(t)=3\sin t+2 \\
				&0\leq t\leq2\pi
			\end{aligned}$
		\end{varwidth}
	}$

\item[(e)]
	Let $\cos^2t=\dfrac{x^2}{4},\sin^2t=\dfrac{y^2}{9}\implies\cos t=\dfrac{x}{2},\sin t=\dfrac{y}{3}$. All points on the left half of the ellipse satisfy $x\leq0\implies\cos t\leq0\implies\dfrac{\pi}{2}\leq t\leq\dfrac{3\pi}{2}$.

	\begin{equation*}
		\boxed{x(t)=2\cos t,y(t)=3\sin t,\frac{\pi}{2}\leq t\leq\frac{3\pi}{2}}
	\end{equation*}

\end{itemize}

\section*{Problem 8}
\begin{itemize}
\item[(a)]
	Let $t=x\implies y=t^2$. $-1\leq x\leq 2\implies-1\leq t\leq 2$. Let $t_1=t+1\implies t=t_1-1,x=t_1-1,y=(t_1-1)^2,0\leq t_1\leq3$.
	\begin{equation*}
		\boxed{x(t)=t-1,y(t)=(t-1)^2,0\leq t\leq3}
	\end{equation*}

\item[(b)]
	First path: $x=0+(3-0)t=3t,y=0+(4-0)t=4t,0\leq t\leq1$.

	Second path: We need to increase $t$ by $1$ so that it can connect to the first path. \\ $x=3+(5-3)(t-1)=3+2t-2=1+2t,y=4+(0-4)(t-1)=4-4t+4=8-4t,1<t\leq2$.
	\begin{equation*}
		\boxed{x(t)=\begin{cases}
			3t & 0\leq t\leq1 \\
			1+2t & 1<t\leq2
		\end{cases},
		y(t)=\begin{cases}
			4t & 0\leq t\leq1 \\
			8-4t & 1<t\leq2
		\end{cases}}
	\end{equation*}

\item[(c)]
	We will reverse the direction of the parametric equation by using $-t$ instead of $t$, and apply a parameter shift of $-\dfrac{\pi}{2}$ to move the starting point to $(0,-1)$.

	$\begin{aligned}
		&x=\cos\left(-\left(t+\frac{\pi}{2}\right)\right) \\
		&y=\sin\left(-\left(t+\frac{\pi}{2}\right)\right) \\
		&0\leq t\leq2\pi
	\end{aligned}\implies\begin{aligned}
		&x=\cos t\cos\frac{\pi}{2}-\sin t\sin\frac{\pi}{2} \\
		&y=-\sin t\cos\frac{\pi}{2}-\cos t\sin\frac{\pi}{2} \\
		&0\leq t\leq2\pi
	\end{aligned}\implies\fbox{
		\begin{varwidth}{\dimexpr\textwidth-2\fboxsep-2\fboxrule\relax}
			$\begin{aligned}
				&x(t)=-\sin t \\
				&y(t)=-\cos t \\
				&0\leq t\leq2\pi
			\end{aligned}$
		\end{varwidth}
	}$

\end{itemize}

\section*{Problem 9}
\begin{itemize}
\item[(a)]
	\phantom{}
	\vspace*{-25pt}
	\begin{center}
		\includegraphics*[width=0.82\linewidth]{q9a.png}
	\end{center}
	Intersections: $(-2.097,1.430),(-3,0)$

\item[(b)]
	A collision point must satisfy $x_1=y_1,x_2=y_2$ at the same value $t$.

	\begin{align*}
		3\sin t&=-3+\cos t \\
		2\cos t&=1+\sin t \\
		\\
		9-9\cos^2t&=\cos^2t-6\cos t+9 \\
		4-4\sin^2t&=\sin^2t+2\sin t+1 \\
		\\
		0&=10\cos^2t-6\cos t \\
		0&=5\sin^2t+2\sin t-3 \\
		\\
		0&=2\cos(t)(5\cos t-3) \\
		t&\in\left\{\frac{\pi}{2},\frac{3\pi}{2},\arccos\frac{3}{5},2\pi-\arccos\frac{3}{5}\right\} \\
		\\
		0&=5\sin^2t+5\sin t-3\sin t-3 \\
		0&=5\sin(t)(\sin t+1)-3(\sin t+1) \\
		0&=(5\sin t-3)(\sin t+1) \\
		t&\in\left\{\arcsin\frac{3}{5},\pi-\arcsin\frac{3}{5},\frac{3\pi}{2}\right\}
	\end{align*}

	A collision occurs at $t=\dfrac{3\pi}{2}$. Thus the one collision point is \boxed{(-3,0)}.

\end{itemize}

\end{document}