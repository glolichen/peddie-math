\documentclass{article}

\usepackage[letterpaper,portrait,top=0.4in, left=0.6in, right=0.6in, bottom=1in]{geometry}

\usepackage{amsmath, amsfonts, amsthm, amssymb}
\usepackage{graphicx, float}
\usepackage{mathtools}
\usepackage{titlesec}
\usepackage{interval}
\usepackage{hyperref}
\usepackage{siunitx}
\usepackage{titling}
\usepackage{vwcol}
\usepackage{setspace}
\usepackage{empheq}
\usepackage{cancel}
\usepackage{esdiff}
\usepackage{multicol}
\usepackage{mdframed}
\usepackage{esdiff}
\usepackage{tikzsymbols}
\usepackage{multicol}
\usepackage{tikz}
\usepackage{varwidth}
\usepackage{pgfplots}

\intervalconfig {
	soft open fences
}

\newcommand{\alignedintertext}[1]{%
  \noalign{%
    \vskip\belowdisplayshortskip
    \vtop{\hsize=\linewidth#1\par
    \expandafter}%
    \expandafter\prevdepth\the\prevdepth
  }%
}

\newtheorem{lemma}{Lemma}

\renewcommand{\qedsymbol}{\Smiley[1.3]}
\newcommand*{\paren}[1]{\ensuremath\left(#1\right)}
\newcommand*{\problem}[1]{\section*{Problem #1}}
\newcommand*{\aps}{\section*{AP Corner}}
\newcommand*{\limit}[2][x]{\ensuremath{\displaystyle\lim_{#1\to#2}}}
\newcommand*{\Limit}[3][x]{\ensuremath{\displaystyle\lim_{#1\to#2}\left[#3\right]}}
\newcommand*{\deriv}[1][x]{\ensuremath{\dfrac{\mathrm{d}}{\mathrm{d}#1}}}
\newcommand*{\Deriv}[2][x]{\ensuremath{\dfrac{\mathrm{d}}{\mathrm{d}#1}\left[#2\right]}}
\newcommand*{\iinteg}[2][x]{\ensuremath{\displaystyle\int #2\;\mathrm{d}#1}}
\newcommand*{\dinteg}[4][x]{\ensuremath{\displaystyle\int_{#2}^{#3}#4\;\mathrm{d}#1}}
\newcommand*{\abs}[1]{\ensuremath{\left|#1\right|}}
\newcommand*{\eps}{\varepsilon}
\newcommand*{\floor}[1]{\ensuremath{\lfloor #1\rfloor}}
\newcommand*{\cbrt}[1]{\ensuremath{\sqrt[3]{#1}}}

\DeclareMathOperator{\DNE}{DNE}

%opening
\title{Problem Set \#63}
\author{Jayden Li}
\date{April 17, 2024}

\allowdisplaybreaks

\begin{document}
\setstretch{1.25}
\fontsize{12pt}{12pt}\selectfont
\setlength{\abovedisplayskip}{0pt}
\maketitle

\problem{1}
\begin{itemize}
	\item[(a)]
	$f$ is a polynomial so it is differentiable on $(1,3)$ and continuous on $[1,3]$.
	\begin{align*}
		f(1)&=5-12(1)+3(1)^2=5-12+3=-4 \\
		f(3)&=5-12(3)+3(3)^2=5-36+27=-4
	\end{align*}
	\begin{align*}
		f'(x)&=0 \\
		-12+6x&=0 \\
		x&=2
	\end{align*}

	\item[(b)]
	$\sqrt{x}$ is continuous on $[0,9]$ and differentiable on $(0,9)$, and $x/3$ is continuous and differentiable on $\mathbb{R}$. So $f$ is continuous on $[0,9]$ and differentiable on $(0,9)$.
	\begin{align*}
		f(0)&=\sqrt{0}-0/3=0 \\
		f(9)&=\sqrt{9}-9/3=0
	\end{align*}
	\begin{align*}
		f'(x)&=0 \\
		\frac{1}{2\sqrt{x}}-\frac{1}{3}&=0 \\
		2\sqrt{x}&=3 \\
		\sqrt{x}&=\frac{3}{2} \\
		x&=\frac{\sqrt{3}}{\sqrt{2}}\cdot\frac{\sqrt{2}}{\sqrt{2}}
		\intertext{Discard negative case since $x\in(0,9)$.}
		x&=\frac{\sqrt{6}}{2}
	\end{align*}
\end{itemize}

\problem{2}
\begin{align*}
	f(x)&=1-\paren{\cbrt{x}}^2 \\
	f(-1)&=1-\paren{\cbrt{-1}}^2=0 \\
	f(1)&=1-\paren{\cbrt{1}}^2=0
\end{align*}
\begin{align*}
	f'(c)&=0 \\
	-\frac{2}{3\cbrt{c}}&=0 \\
	-2&=0\cdot3\cbrt{c}
\end{align*}
Zero multiplied by any real number is $0$, so the above statement is false.

This does not contradict Rolle's Theorem because $f$ is not differentiable at $x=0$, so $f$ is not differentiable on $(-1,1)$, which is one of the conditions of Rolle's Theorem.

\problem{3}
$x\in\{0.9,3.2,4.4,6.1\}$

\problem{4}
\begin{itemize}
	\begin{minipage}[t]{0.5\linewidth}
		\item[(a)]
		\phantom{}

		\centering
		\begin{tikzpicture}
			\begin{axis}[
				% restrict y to domain=0:10,
				samples=100,
				minor tick num=1,
				xmin = 0, xmax = 10,
				ymin = 0, ymax = 10,
				axis lines=middle,
				axis line style={->},
				x label style={anchor=west},
				y label style={anchor=south},
				xlabel={$x$},
				ylabel={$f(x)$},
				unbounded coords=jump]
				\addplot[mark=none, domain=0:10] {x+4/x};
			\end{axis}
		\end{tikzpicture}
	\end{minipage}
	\begin{minipage}[t]{0.5\linewidth}
		\item[(b)]
		\phantom{}

		\centering
		\begin{tikzpicture}
			\begin{axis}[
				% restrict y to domain=0:10,
				samples=100,
				minor tick num=1,
				xmin = 0, xmax = 10,
				ymin = 0, ymax = 10,
				axis lines=middle,
				axis line style={->},
				x label style={anchor=west},
				y label style={anchor=south},
				xlabel={$x$},
				ylabel={$f(x)$},
				unbounded coords=jump]
				\addplot[mark=none, domain=0:10] {x+4/x};
				\addplot[mark=none, domain=0:10, color=red] {0.5*x+4.5};
			\end{axis}
		\end{tikzpicture}
	\end{minipage}

	\item[(c)]
	\begin{align*}
		f'(c)&=\frac{f(8)-f(1)}{8-1} \\
		1+4\paren{-1\cdot x^{-2}}&=\frac{8.5-5}{7} \\
		1-\frac{4}{x^2}&=\frac{3.5}{7} \\
		\frac{4}{x^2}&=\frac{1}{2} \\
		x^2&=8
		\intertext{Ignore negative case since $-\sqrt{8}\not\in[1,8]$.}
		x&=2\sqrt{2}
	\end{align*}
	\begin{figure}[h]
		\centering
		\begin{tikzpicture}
			\begin{axis}[
				% restrict y to domain=0:10,
				samples=100,
				minor tick num=1,
				xmin = 0, xmax = 10,
				ymin = 0, ymax = 10,
				axis lines=middle,
				axis line style={->},
				x label style={anchor=west},
				y label style={anchor=south},
				xlabel={$x$},
				ylabel={$f(x)$},
				unbounded coords=jump]
				\addplot[mark=none, domain=0:10] {x+4/x};
				\addplot[mark=none, domain=0:10, color=red] {0.5*x+4.5};
				\addplot[mark=none, domain=0:10, color=blue] {0.5*x+2.83};
			\end{axis}
		\end{tikzpicture}
	\end{figure}
\end{itemize}

\problem{5}
\begin{itemize}
	\item[(a)]
	$f$ is differentiable on $(-1,1)$ and continuous on $[-1,1]$ because it is a polynomial.
	\begin{align*}
		f'(x)&=\frac{f(1)-f(-1)}{1-(-1)} \\
		6x+2&=\frac{10-6}{2} \\
		6x&=2-2 \\
		x&=0
	\end{align*}

	\item[(b)]
	$\cbrt{x}$ is differentiable at all real numbers except $0$, so it is differentiable on $(0,1)$ and continuous on $[0,1]$.
	\begin{multicols}{2}
		\begin{align*}
			f'(x)&=\frac{\cbrt{1}-\cbrt{0}}{1-0} \\
			\frac{1}{3}x^{-2/3}&=1 \\
			\frac{1}{\paren{\cbrt{x}}^2}&=3 \\
			\paren{\cbrt{x}}^2&=\frac{1}{3} \\
			\cbrt{x}&=\frac{1}{\sqrt{3}}\cdot\frac{\sqrt{3}}{\sqrt{3}} \\
			x&=\frac{\paren{\sqrt{3}}^3}{3^3} \\
			x&=\frac{3\sqrt{3}}{27} \\
			x&=\frac{\sqrt{3}}{9}
		\end{align*}
	\end{multicols}
\end{itemize}

\problem{6}
\begin{align*}
	f(4)-f(1)&=f'(c)(4-1) \\
	(4-3)^{-2}-(1-2)^{-2}&=-2(c-3)^{-3}\cdot1\cdot3 \\
	\frac{1}{1^2}-\frac{1}{(-1)^2}&=-\frac{6}{(c-3)^3} \\
	0\cdot(c-3)^3&=-6
\end{align*}
Which is impossible since any number multiplied by $0$ is $0$.

This does not contradict the MVT because $f$ is undefined at $x=3$, so it is not continuous on $[1,4]$.

\problem{7}


\problem{8}


\problem{9}


\problem{10}


\problem{11}


\problem{12}


\problem{13}


\problem{14}


\problem{15}

\end{document}
