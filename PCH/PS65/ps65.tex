\documentclass{article}

\usepackage[letterpaper,portrait,top=0.4in, left=0.6in, right=0.6in, bottom=1in]{geometry}

\usepackage{amsmath, amsfonts, amsthm, amssymb}
\usepackage{graphicx, float}
\usepackage{mathtools}
\usepackage{titlesec}
\usepackage{interval}
\usepackage{hyperref}
\usepackage{siunitx}
\usepackage{titling}
\usepackage{vwcol}
\usepackage{setspace}
\usepackage{empheq}
\usepackage{cancel}
\usepackage{esdiff}
\usepackage{multicol}
\usepackage{mdframed}
\usepackage{esdiff}
\usepackage{tikzsymbols}
\usepackage{multicol}
\usepackage{tikz}
\usepackage{varwidth}
\usepackage{pgfplots}

\intervalconfig {
	soft open fences
}

\newcommand{\alignedintertext}[1]{%
  \noalign{%
    \vskip\belowdisplayshortskip
    \vtop{\hsize=\linewidth#1\par
    \expandafter}%
    \expandafter\prevdepth\the\prevdepth
  }%
}

\newtheorem{lemma}{Lemma}

\renewcommand{\qedsymbol}{\Smiley[1.3]}
\newcommand*{\paren}[1]{\ensuremath\left(#1\right)}
\newcommand*{\problem}[1]{\section*{Problem #1}}
\newcommand*{\aps}{\section*{AP Corner}}
\newcommand*{\limit}[2][x]{\ensuremath{\displaystyle\lim_{#1\to#2}}}
\newcommand*{\Limit}[3][x]{\ensuremath{\displaystyle\lim_{#1\to#2}\left[#3\right]}}
\newcommand*{\deriv}[1][x]{\ensuremath{\dfrac{\mathrm{d}}{\mathrm{d}#1}}}
\newcommand*{\Deriv}[2][x]{\ensuremath{\dfrac{\mathrm{d}}{\mathrm{d}#1}\left[#2\right]}}
\newcommand*{\iinteg}[2][x]{\ensuremath{\displaystyle\int #2\;\mathrm{d}#1}}
\newcommand*{\dinteg}[4][x]{\ensuremath{\displaystyle\int_{#2}^{#3}#4\;\mathrm{d}#1}}
\newcommand*{\abs}[1]{\ensuremath{\left|#1\right|}}
\newcommand*{\eps}{\varepsilon}
\newcommand*{\floor}[1]{\ensuremath{\lfloor #1\rfloor}}
\newcommand*{\cbrt}[1]{\ensuremath{\sqrt[3]{#1}}}

\DeclareMathOperator{\DNE}{DNE}

%opening
\title{Problem Set \#65}
\author{Jayden Li}
\date{April 23, 2024}

\allowdisplaybreaks

\begin{document}
\setstretch{1.25}
\fontsize{12pt}{12pt}\selectfont
\setlength{\abovedisplayskip}{0pt}
\maketitle

\problem{2}
\begin{itemize}
	\item[(a)]
	At $x=4$, where the $f$ goes from decreasing to increasing. By the first derivative test it is a local minimum.

	\item[(b)]
	We need to compare whether the function decreases more where the derivative is a semicircle, or whether it increases more. We can find out by calculating the area under or above the graph, which represents the quantity of change accumulated by $f'$.
	
	The area above the semicircle is $2^2\cdot\pi/2=2\pi\approx6.28$, so the absolute difference between $f(0)$ and $f(4)$ is approximately $6.28$. Because the function is decreasing, we see that $f(0)$ is approximately $6.28$ higher than $f(4)$.

	The area below the line is the area of a triangle with legs of length $2$ and $6$. Its area is $2\cdot6/2=6$. Therefore $f(10)$ is $6$ higher than $f(4)$.

	It is known that $f(10)=2$. Therefore, we see that $f(4)=2-6=-4$.

	We found earlier that $f(0)-6.28\approx f(4)$, so $f(0)\approx-4+6.28=2.28$.

	Because $2.28>2$, $f$ reaches an absolute maximum at $x=0$.
\end{itemize}

\problem{3}
$R$ has an absolute maximum at $t=2.29$ and $R(2.29)=3.95$. Therefore the fastest rate in which water fills into the rank is $3.95$ gallons per hour.

\problem{5}
\begin{itemize}
	\item[(a)]
	If $(1,-2)$ were a critical point, then the derivative evaluated at $x=1,y=-2$ is zero or not differentiable at that point (definition of a critical point). $2(1)+(-2)=0$, so it is a critical point.

	\item[(b)] 
	We use the second derivative test. We first find the second derivative $\diff[2]{y}{x}$:
	\begin{equation*}
		\diff[2]{y}{x}=\Deriv{2x+y}=2+\diff{y}{x}=2+2x+y
	\end{equation*}
	The second derivative evaluated at $(1,-2)$ is $2+2(1)+(-2)=2>0$. Thus, by the second derivative test the point $(1,-2)$ is a local minimum.
\end{itemize}

\problem{6}
\begin{itemize}
	\item[(a)]
	An inflection point is where the function $f$ changes from concave down to concave up, or from concave up to concave down (definition of inflection point). A point is concave up if the second derivative $f''$ is greater than $0$ at that point, and concave down if $f''$ is less than $0$ (definition of concavity). So we can say that an inflection point is where the second derivative $f''$ changes sign -- in other words, where the first derivative $f'$ changes from increasing to decreasing, or from decreasing to increasing. On the graph of $f'$, we see that this happens at $x=2$ and $x=6$, which are the inflection points of $f$.

	\item[(b)]
	\begin{align*}
		g(x)&=f(x)-x \\
		g'(x)&=f'(x)-1
	\end{align*}
	Decreasing on $(0,5)$, which is where the graph of $f'$ translated down $1$ unit is negative.

	\item[(c)]
	Suppose that $c\in[0,7]$ is a critical point of $g$. Then $g$ is either not differentiable at $c$, or $g'(c)=0$. $f$ is differentiable on $[0,7]$. $x$ is differentiable for all real numbers. Therefore, $g$ is differentiable on $[0,7]$, so there is no point $c$ where $g$ is not differentiable. So all critical points $c$ are such that $g'(c)=0$.
	\begin{align*}
		g'(c)&=0 \\
		f'(x)-1&=0 \\
		f'(c)&=1
	\end{align*}
	By looking at the graph of $f'$, $c$ must equal $5$. So we have a critical point at $x=5$.
	
	By the closed interval method, the absolute minimum is either at the endpoints (which are $x=0$ and $x=7$), or at critical points ($x=5$). If we were to draw the graph of $g'$, it is obvious that the area above the semicircle connected to a line segment is greater than the area under the triangle. So I claim that the absolute minimum occurs at $x=4$.

	It is known that $f(4)=3$, so $g(4)=f(4)-4=3-4=-1$. The absolute maximum of $g$ on the interval $[0,7]$ is $-1$.
\end{itemize}

\problem{7}
\begin{itemize}
	\item[(a)]
	\begin{equation*}
		H'(6)\approx\frac{H(7)-H(5)}{7-5}=\frac{11-6}{2}=\frac{5}{2}
	\end{equation*}
	The rate of growth of the tree $6$ years after being grown, in meters per year.

	\item[(b)]
	Consider the interval $[3,5]$. The average rate of change of $H$ over this interval is given by:
	\begin{equation*}
		\frac{H(5)-H(3)}{5-3}=\frac{6-2}{2}=2
	\end{equation*}
	$H$ is differentiable on $(3,5)$ (given by the problem), so it must be continuous on $[3,5]$. Then there must exist some $c\in(3,5)$ such that $H'(c)=2$ (by the mean value theorem).
\end{itemize}

\problem{8}
\begin{itemize}
	\item[(a)]
	Assuming that $r'$ has a constant rate of change on $[7,10]$, $r''(8.5)$ will equal the average rate of change on $[7,10]$.
	\begin{equation*}
		r''(8.5)=\frac{r'(10)-r'(7)}{10-7}=\frac{-3.8-(4.4)}{3}=\frac{0.6}{3}=0.2
	\end{equation*}

	\item[(b)]
	Yes by the intermediate value theorem.
	
	Because $r$ is twice-differentiable, $r'$ must be differentiable and therefore continuous on $[0,3]$.

	Let $c=-6$. We have that $c\in(r'(0),r'(3))$. By the intermediate value theorem, $r'(t)=c$ for some $t\in(0,3)$.
\end{itemize}

\problem{9}
\begin{itemize}
	\item[(a)]
	$f$ is increasing on $[-6,-2)\cup(2,5)$ because that is where $f'$ is positive.

	\item[(b)]
	Critical points are $x=-2$ and $x=2$. Endpoints are $x=-6$ and $x=5$. $f'$ exists for all values on $[-6,5]$ so $f$ must also be continuous on $[-6,5]$ so we can use the closed interval method. Because I don't want to think of an easier way, we use integration to calculate $f(-6)$, $f(2)$ and $f(5)$.
	\begin{gather*}
		f(-2)=7 \\
		\dinteg{-6}{-2}{f'(x)}=f(-2)-f(-6) \implies 4=7-f(-6) \implies f(-6)=3 \\
		\dinteg{-2}{2}{f'(x)}=f(2)-f(-2) \implies -2\pi=f(2)-7 \implies f(2)=7-2\pi \\
		\dinteg{-2}{5}{f'(x)}=f(5)-f(-2) \implies -2\pi+3=f(5)-7 \implies f(5)=10-2\pi
	\end{gather*}
	The absolute minimum value is $7-2\pi$ because it is less than all the other points.

	\item[(c)]
	$f''(3)$ doesn't exist because $f'$ has a sharp corner at $x=3$ and sharp corners are not differentiable.

	$f''(5)$ doesn't exist. Since $f'$ is continuous on the closed interval $[-6,5]$, $f'$ is differentiable on the open interval $(-6,5)\not\ni5$.
\end{itemize}

\end{document}
