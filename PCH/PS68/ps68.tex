\documentclass[preview, margin=0.6in]{standalone}

\usepackage[letterpaper,portrait,top=0.4in, left=0.6in, right=0.6in, bottom=1in]{geometry}

\usepackage{amsmath, amsfonts, amsthm, amssymb}
\usepackage{graphicx, float}
\usepackage{mathtools}
\usepackage{titlesec}
\usepackage{interval}
\usepackage{hyperref}
\usepackage{siunitx}
\usepackage{titling}
\usepackage{vwcol}
\usepackage{setspace}
\usepackage{empheq}
\usepackage{cancel}
\usepackage{esdiff}
\usepackage{multicol}
\usepackage{mdframed}
\usepackage{esdiff}
\usepackage{tikzsymbols}
\usepackage{multicol}
\usepackage{tikz}
\usepackage{varwidth}
\usepackage{pgfplots}

\intervalconfig {
	soft open fences
}

\newcommand{\alignedintertext}[1]{%
  \noalign{%
    \vskip\belowdisplayshortskip
    \vtop{\hsize=\linewidth#1\par
    \expandafter}%
    \expandafter\prevdepth\the\prevdepth
  }%
}

\newtheorem{lemma}{Lemma}

\renewcommand{\qedsymbol}{\Smiley[1.3]}
\newcommand*{\paren}[1]{\ensuremath\left(#1\right)}
\newcommand*{\problem}[1]{\section*{Problem #1}}
\newcommand*{\aps}{\section*{AP Corner}}
\newcommand*{\limit}[2][x]{\ensuremath{\displaystyle\lim_{#1\to#2}}}
\newcommand*{\Limit}[3][x]{\ensuremath{\displaystyle\lim_{#1\to#2}\left[#3\right]}}
\newcommand*{\deriv}[1][x]{\ensuremath{\dfrac{\mathrm{d}}{\mathrm{d}#1}}}
\newcommand*{\Deriv}[2][x]{\ensuremath{\dfrac{\mathrm{d}}{\mathrm{d}#1}\left[#2\right]}}
\newcommand*{\iinteg}[2][x]{\ensuremath{\displaystyle\int #2\;\mathrm{d}#1}}
\newcommand*{\dinteg}[4][x]{\ensuremath{\displaystyle\int_{#2}^{#3}#4\;\mathrm{d}#1}}
\newcommand*{\abs}[1]{\ensuremath{\left|#1\right|}}
\newcommand*{\eps}{\varepsilon}
\newcommand*{\floor}[1]{\ensuremath{\lfloor #1\rfloor}}
\newcommand*{\cbrt}[1]{\ensuremath{\sqrt[3]{#1}}}
\newcommand*{\lheqzero}{\ensuremath{\underset{\text{L'H}}{\overset{\left[\frac00\right]}{=}}}}
\newcommand*{\lheqinfty}{\ensuremath{\underset{\text{L'H}}{\overset{\left[\frac{\infty}{\infty}\right]}{=}}}}

\DeclareMathOperator{\DNE}{DNE}
\DeclareMathOperator{\sgn}{sgn}

\setlength{\parindent}{0pt}

%opening
\title{\vspace*{-30pt}Problem Set \#68}
\author{Jayden Li}
\date{May 9, 2024}

% \allowdisplaybreaks
\postdisplaypenalty=100000

\begin{document}
\setstretch{1.25}
\fontsize{12pt}{12pt}\selectfont
\setlength{\abovedisplayskip}{0pt}
\maketitle

\problem{1}
\begin{multicols}{2}
	\begin{itemize}
		\item[(a)]
		\begin{minipage}[t]{\linewidth}
			\begin{align*}
				y&=\limit{0^+}(1+\sin4x)^{\cot x} \\
				\ln y&=\limit{0^+}\cot(x)\ln(1+\sin4x) \\
				&=\limit{0^+}\frac{\cos(x)\ln(1+\sin4x)}{\sin x} \\
				\shortintertext{
					\begin{mdframed}
						\begin{align*}
							\limit{0^+}\cos(x)\ln(1+\sin4x)&=0 \\
							\limit{0^+}\sin x&=0
						\end{align*}
					\end{mdframed}
				}
				&\lheqzero\limit{0^+}\frac{-\sin(x)\ln(1+\sin4x)+\frac{4\cos(x)\cos(4x)}{1+\sin 4x}}{\cos x} \\
				&=\frac{0+\frac{4}{1}}{1}
				=4 \implies
				y=\boxed{e^4}
			\end{align*}
		\end{minipage}

		\item[(b)]
		\begin{minipage}[t]{\linewidth}
			\begin{align*}
				y&=\limit{0^+}x^x \implies
				\ln y=\limit{0^+}x\ln x
				=\limit{0^+}\frac{\ln x}{\frac1x}
				\shortintertext{
					\begin{mdframed}
						\begin{align*}
							\limit{0^+}\ln x&=-\infty \\
							\limit{0^+}\frac{1}{x}&=\infty
						\end{align*}
					\end{mdframed}
				}
				&\lheqinfty\limit{0^+}\frac{\frac1x}{-\frac{1}{x^2}}
				=-\limit{0^+}x
				=0 \implies
				y=e^0=\boxed{1}
			\end{align*}
		\end{minipage}
	\end{itemize}
\end{multicols}

\problem{2}
\begin{multicols}{2}
	\begin{itemize}
		\item[(a)]
		\begin{minipage}[t]{\linewidth}
			\begin{align*}
				y&=\limit{0^+}x^{x^2} \\
				\ln y&=\limit{0^+}x^2\ln x
				=\limit{0^+}\frac{\ln x}{\frac{1}{x^2}}
				\shortintertext{
					\begin{mdframed}
						\begin{align*}
							\limit{0^+}\ln x&=-\infty \\
							\limit{0^+}\frac{1}{x^2}&=\infty
						\end{align*}
					\end{mdframed}
				}
				&\lheqinfty\limit{0^+}\frac{\frac1x}{-\frac{1}{x^3}}
				=-\limit{0^+}x^2
				=0 \\
				y&=e^0=\boxed{1}
			\end{align*}
		\end{minipage}

		\item[(b)]
		\begin{minipage}[t]{\linewidth}
			\begin{align*}
				y&=\limit{0}(1-2x)^{\frac1x} \\
				\ln y&=\Limit{0}{\frac1x\ln(1-2x)}
				=\limit{0}\frac{\ln(1-2x)}{x}
				\shortintertext{
					\begin{mdframed}
						\begin{align*}
							\limit{0}\ln(1-2x)&=0 \\
							\limit{0}x&=0
						\end{align*}
					\end{mdframed}
				}
				&\lheqzero\limit{0}\frac{\frac{1}{1-2x}\cdot(-2)}{1}
				=\frac{1}{1-0}\cdot(-2)
				=-2 \\
				y&=e^{-2}=\boxed{-\frac{1}{e^2}}
			\end{align*}
		\end{minipage}

		\item[(c)]
		\begin{minipage}[t]{\linewidth}
			\begin{align*}
				y&=\limit{\infty}\paren{1+\frac3x+\frac{5}{x^2}}^x \\
				\ln y&=\limit{\infty}x\ln\paren{1+\frac3x+\frac{5}{x^2}}
				\shortintertext{Let $k=1/x$. Then $x=1/k$, $3/x=3k$ and $5/x^2=5k^2$. As $x\to\infty$, $k\to0$.}
				&=\limit[k]{0}\frac{\ln\paren{1+3k+5k^2}}{k}
				\shortintertext{
					\begin{mdframed}
						\begin{align*}
							\limit[k]{0}\ln\paren{1+3k+5k^2}&=0 \\
							\limit[k]{0}k&=0
						\end{align*}
					\end{mdframed}
				}
				&\lheqzero\limit[k]{0}\frac{\frac{3+10k}{1+3k+5k^2}}{1}
				=\frac{3+0}{1+0+0}
				=3 \\
				y&=\boxed{e^3}
			\end{align*}
		\end{minipage}

		\item[(d)]
		\begin{minipage}[t]{\linewidth}
			\begin{align*}
				y&=\limit{\infty}x^{\frac1x} \\
				\ln y&=\limit{\infty}\frac{1}{x}\ln x
				=\limit{\infty}\frac{\ln x}{x}
				=0
				\shortintertext{(This miraculous fact is proven in Problem 9.)}
				y&=e^0=\boxed{1}
			\end{align*}
		\end{minipage}

		\item[(e)]
		\begin{minipage}[t]{\linewidth}
			\begin{align*}
				y&=\limit{0^+}(4x+1)^{\cot x} \\
				\ln y&=\limit{0^+}\cot(x)\ln(4x+1) \\
				&=\limit{0^+}\frac{\cos(x)\ln(4x+1)}{\sin x}
				\shortintertext{
					\begin{mdframed}
						\begin{align*}
							\limit{0^+}\cos(x)\ln(4x+1)&=0 \\
							\limit{0^+}\sin x&=0
						\end{align*}
					\end{mdframed}
				}
				&\lheqzero\limit{0^+}\frac{-\cos(x)\ln(4x+1)+\frac{4\cos x}{4x+1}}{\cos x} \\
				&=\frac{-1\cdot0+\frac{4}{1}}{1}
				=4 \implies
				y=\boxed{e^4}
			\end{align*}
		\end{minipage}

		\item[(f)]
		\begin{minipage}[t]{\linewidth}
			\begin{align*}
				y&=\limit{0^+}(\cos x)^{\frac{1}{x^2}} \\
				\ln y&=\limit{0^+}\frac{1}{x^2}\ln(\cos x)
				=\limit{0^+}\frac{\ln(\cos x)}{x^2}
				\shortintertext{
					\begin{mdframed}
						\begin{align*}
							\limit{0^+}\ln(\cos x)&=0 \\
							\limit{0^+}x^2&=0 \\
							\limit{0^+}\sin x&=0 \\
							\limit{0^+}2x\cos x&=0
						\end{align*}
					\end{mdframed}
				}
				&\lheqzero\limit{0^+}\frac{\frac{-\sin x}{\cos x}}{2x}
				=-\limit{0^+}\frac{\sin x}{2x\cos x} \\
				&\lheqzero-\limit{0^+}\frac{\cos x}{2\cos x-2x\sin x}
				=-\frac{1}{2-0}=-\frac{1}{2} \\
				y&=e^{-\frac12}=\frac{1}{\sqrt{e}}=\boxed{\frac{\sqrt{e}}{e}}
			\end{align*}
		\end{minipage}
	\end{itemize}
\end{multicols}

\problem{4}
\begin{align*}
	&\limit{\infty}\frac{x}{\sqrt{x^2+1}}
	\shortintertext{
		\begin{mdframed}
			\begin{align*}
				\limit{\infty}x&=\infty \\
				\limit{\infty}\sqrt{x^2+1}&=\infty
			\end{align*}
		\end{mdframed}
	}
	=\lheqinfty{}&\limit{\infty}\frac{1}{\frac{2x}{2\sqrt{x^2+1}}}
	=\limit{\infty}\frac{\sqrt{x^2+1}}{x}
\end{align*}
Which is the reciprocal of what we started with. If we keep applying LHR it will just go back to the original function.
\begin{equation*}
	\limit{\infty}\frac{x}{\sqrt{x^2+1}}
	=\limit{\infty}\frac{x}{|x|\sqrt{1+\frac{1}{x^2}}}
	=\limit{\infty}\frac{\sgn(x)}{\sqrt{1+\frac{1}{x^2}}}
	=\boxed{1}
\end{equation*}

\problem{6}
First, we change the variable $E$ to $x$ because it is too confusing.
\begin{align*}
	&\limit{0^+}P(x)=\Limit{0^+}{\frac{e^x+e^{-x}}{e^x-e^{-x}}-\frac1x}
	=\Limit{0^+}{\coth x-\frac1x}
	=\Limit{0^+}{\frac{\cosh x}{\sinh x}-\frac1x}
	=\limit{0^+}\frac{x\cosh x-\sinh x}{x\sinh x}
	\shortintertext{
		\begin{mdframed}
			\begin{align*}
				\Limit{0^+}{x\cosh x-\sinh x}&=0 \\
				\limit{0^+}x\sinh x&=0
			\end{align*}
		\end{mdframed}
	}
	=\lheqzero{}&\limit{0^+}\frac{\cosh x+x\sinh x-\cosh x}{\sinh x+x\cosh x}
	=\frac{\cosh0+0\sinh0-\cosh0}{\sinh0+0\cosh0}
	=\frac{0}{1}
	=0
\end{align*}

\problem{7}
\begin{align*}
	A&=\Limit[n]{\infty}{A_0\paren{1+\frac rn}^{nt}} \\
	\ln A&=\ln\paren{A_0}\Limit[n]{\infty}{nt\ln\paren{1+\frac rn}}
	\shortintertext{Let $k=1/n$. Then $n=1/k$ and $r/n=rk$. As $n\to\infty$, $k\to0$.}
	&=\ln\paren{A_0}\limit[k]{0}\frac{t\ln(1+rk)}{k}
	\shortintertext{
		\begin{mdframed}
			\begin{align*}
				\Limit[k]{0}{t\ln(1+rk)}&=0 \\
				\limit[k]{0}k&=0
			\end{align*}
		\end{mdframed}
	}
	&\lheqzero\ln\paren{A_0}\limit[k]{0}\frac{\frac{rt}{1+rk}}{1}
	=rt\ln\paren{A_0} \implies
	A=\exp\paren{rt\ln\paren{A_0}}
	=\exp\paren{\ln\paren{A_0}}\cdot e^{rt}
	=A_0e^{rt}
\end{align*}
\qed

\problem{8}
\begin{align*}
	&\limit{0}\frac{f(2+3x)+f(2+5x)}{x}
	=\limit{0}\frac{f(2+3x)-f(2)+f(2+5x)-f(2)}{x} \\
	={}&\limit{0}\frac{f(2+3x)-f(2)}{x}+\limit{0}\frac{f(2+5x)-f(2)}{x}
	\intertext{Let $h_1=3x$ and $h_2=5x$. Then $x=h_1/3=h_2/5$. As $x\to0$, $h_1,h_2\to0$.}
	={}&\limit[h_1]{0}\frac{f(2+h_1)-f(2)}{\frac{h_1}{3}}+\limit[h_2]{0}\frac{f(2+h_2)-f(2)}{\frac{h_2}{5}} \\
	={}&3\limit[h_1]{0}\frac{f(2+h_1)-f(2)}{h_1}+5\limit[h_2]{0}\frac{f(2+h_2)-f(2)}{h_2} \\
	={}&3f'(2)+5f'(2)
	=3\cdot7+5\cdot7
	=\boxed{56}
\end{align*}

\problem{9}
\begin{mdframed}
	\begin{align*}
		\limit{\infty}\ln x&=\infty \\
		\limit{\infty}x^p&=\infty
	\end{align*}
\end{mdframed}
\begin{align*}
	&\limit{\infty}\frac{\ln x}{x^p}
	\lheqinfty\limit{\infty}\frac{\frac1x}{px^{p-1}}
	=\limit{\infty}\frac{1}{px^p}
	=0
\end{align*}
\qed

\problem{10}
\begin{mdframed}
	\begin{align*}
		\limit{0^+}\ln x&=-\infty \\
		\limit{0^+}x^{-a}&=\limit{0^+}\frac{1}{x^a}=\pm\infty
	\end{align*}
\end{mdframed}
\begin{align*}
	&\limit{0^+}x^a\ln x
	=\limit{0^+}\frac{\ln x}{x^{-a}}
	\lheqinfty\limit{0^+}\frac{\frac1x}{-ax^{-a-1}}
	=-\frac1a\limit{0^+}\frac{1}{x^{-a}}
	=-\frac1a\limit{0^+}x^a
	=\frac1a\cdot0
	=0
\end{align*}
\qed

\end{document}
