\documentclass{article}

\usepackage[letterpaper,portrait,top=0.4in, left=0.6in, right=0.6in, bottom=1in]{geometry}

\usepackage{amsmath, amsfonts, amsthm, amssymb}
\usepackage{graphicx, float}
\usepackage{mathtools}
\usepackage{titlesec}
\usepackage{interval}
\usepackage{hyperref}
\usepackage{siunitx}
\usepackage{titling}
\usepackage{vwcol}
\usepackage{setspace}
\usepackage{empheq}
\usepackage{cancel}
\usepackage{esdiff}
\usepackage{multicol}
\usepackage{mdframed}
\usepackage{esdiff}
\usepackage{multicol}
\usepackage{tikz}
\usepackage{varwidth}

\intervalconfig {
	soft open fences
}

\newcommand{\alignedintertext}[1]{%
	\noalign{%
		\vskip\belowdisplayshortskip
		\vtop{\hsize=\linewidth#1\par
		\expandafter}%
		\expandafter\prevdepth\the\prevdepth
	}%
}

\renewcommand{\Re}{{\mathrm{Re}}}
\renewcommand{\Im}{{\mathrm{Im}}}

% \allowdisplaybreaks

%opening
\title{Problem Set \#53}
\author{Jayden Li}
\date{February 16, 2024}

\allowdisplaybreaks

\begin{document}
\setstretch{1.25}
\fontsize{12pt}{12pt}\selectfont
\setlength{\abovedisplayskip}{0pt}
\maketitle

\section*{Problem 7}
\begin{itemize}
\item[(b)]
	\begin{align*}
		|3-\sqrt{3}i|&=\sqrt{9+3}=\sqrt{12} \\
		\arg(3-\sqrt{3}i)&=\arctan\left(\frac{-\sqrt{3}}{3}\right)=-\frac{\pi}{6} \\
		\\
		(3-\sqrt{3})^5&=\left(\sqrt{12}e^{-\frac{i\pi}{6}}\right)^5 \\
		&=(\sqrt{12})^5\left(\cos\left(-\frac{\pi}{6}\right)+i\sin\left(-\frac{\pi}{6}\right)\right)^5 \\
		&=(\sqrt{12})^4(\sqrt{12})\left(\cos\left(-\frac{5\pi}{6}\right)+i\sin\left(-\frac{5\pi}{6}\right)\right) \\
		&=144(2\sqrt{3})\left(-\frac{\sqrt{3}}{2}-\frac{1}{2}i\right) \\
		&=-\frac{144\cdot2\cdot3}{2}-\frac{288\sqrt{3}}{2}i \\
		&=\boxed{-432-144\sqrt{3}i}
	\end{align*}	
\end{itemize}

\section*{Problem 10}
\begin{itemize}
\item[(a)]
	Let $z=a+bi$. Then $\overline{z}=a-bi$, $\Re(z)=a$ and $\Im(z)=b$.
	\begin{align*}
		\frac{z+\overline{z}}{2}&=\frac{a+bi+a-bi}{2}=\frac{2a}{2}=\Re(z) \\
		\frac{z-\overline{z}}{2i}&=\frac{a+bi-a+bi}{2i}=\frac{2bi}{2i}=b=\Im(z)
	\end{align*}
	\qed

\item[(b)]
	\begin{align}
		e^{i\theta}&=\cos\theta+i\sin\theta \\
		e^{-i\theta}&=\cos(-\theta)+i\sin(-\theta)=\cos\theta-i\sin\theta
	\end{align}
	\begin{minipage}[t]{0.5\linewidth}
		$(1)+(2)$:
		\begin{align*}
			e^{i\theta}+e^{-i\theta}&=2\cos\theta \\
			\frac{1}{2}\left(e^{i\theta}+e^{-i\theta}\right)&=\cos\theta
		\end{align*}
	\end{minipage}
	\begin{minipage}[t]{0.5\linewidth}
		$(1)-(2)$:
		\begin{align*}
			e^{i\theta}-e^{-i\theta}&=2i\sin\theta \\
			\frac{1}{2i}\left(e^{i\theta}-e^{-i\theta}\right)&=\sin\theta
		\end{align*}
	\end{minipage}
	\qed

\item[(c)]
	\begin{minipage}[t]{0.5\linewidth}
		\begin{align*}
			\sin(i)&=\frac{1}{2i}\left(e^{i\cdot i}-e^{-i\cdot i}\right) \\
			&=\frac{1(-i)}{2i(-i)}\left(e^{-1}-e^1\right) \\
			&=\frac{-i\left(e^{-1}-e^1\right)}{2} \\
			&=i\left(\frac{e^1-e^{-1}}{2}\right) \\
			\Aboxed{\sin(i)&=i\sinh1} \\
			&=\frac{i}{2}\left(\frac{e^2}{e}-\frac{1}{e}\right) \\
			\Aboxed{\sin(i)&=\frac{ie^2-i}{2e}}
		\end{align*}
	\end{minipage}
	\begin{minipage}[t]{0.5\linewidth}
		\begin{align*}
			\cos(i)&=\frac{1}{2}\left(e^{i\cdot i}+e^{-i\cdot i}\right) \\
			&=\frac{e^{-1}+e^1}{2} \\
			\Aboxed{\cos(i)&=\cosh1} \\
			&=\frac{1}{2}\left(\frac{1}{e}+\frac{e^2}{e}\right) \\
			\Aboxed{\cos(i)&=\frac{1+e^2}{2e}}
		\end{align*}
	\end{minipage}

\item[(d)]
	Let $\theta=ix$. Then $x=-i\theta$.
	
	\begin{minipage}[t]{0.5\linewidth}
		\begin{align*}
			\cos(ix)&=\frac{e^{i\cdot ix}+e^{-i\cdot ix}}{2} \\
			\cos\theta&=\frac{e^{-x}+e^x}{2} \\
			&=\cosh x \\
			&=\cosh(-i\theta) \\
			\Aboxed{\cos\theta&=\cosh(i\theta)}
		\end{align*}
	\end{minipage}
	\begin{minipage}[t]{0.5\linewidth}
		\begin{align*}
			\sin(ix)&=\frac{e^{i\cdot ix}-e^{-i\cdot ix}}{2i} \\
			\sin\theta&=\frac{e^{-x}-e^x}{2}\cdot\frac{1}{i} \\
			&=\frac{e^x-e^{-x}}{2}\cdot\frac{1}{-i}\left(\frac{i}{i}\right) \\
			&=i\sinh x \\
			&=i\sinh(-i\theta) \\
			\Aboxed{\sin\theta&=-i\sinh(i\theta)}
		\end{align*}
	\end{minipage}

\item[(e)]
	\begin{align*}
		\sin(i)&=i\sinh1 \\
		\cos(i)&=\cosh1
	\end{align*}

\end{itemize}

\end{document}