\documentclass{article}

\usepackage[letterpaper,portrait,top=0.4in, left=0.6in, right=0.6in, bottom=1in]{geometry}

\usepackage{amsmath, amsfonts, amsthm, amssymb}
\usepackage{graphicx, float}
\usepackage{mathtools}
\usepackage{titlesec}
\usepackage{interval}
\usepackage{hyperref}
\usepackage{siunitx}
\usepackage{titling}
\usepackage{vwcol}
\usepackage{setspace}
\usepackage{empheq}
\usepackage{cancel}
\usepackage{esdiff}
\usepackage{multicol}
\usepackage{mdframed}
\usepackage{esdiff}
\usepackage{tikzsymbols}
\usepackage{multicol}
\usepackage{tikz}
\usepackage{varwidth}

\intervalconfig {
	soft open fences
}

\newcommand{\alignedintertext}[1]{%
  \noalign{%
    \vskip\belowdisplayshortskip
    \vtop{\hsize=\linewidth#1\par
    \expandafter}%
    \expandafter\prevdepth\the\prevdepth
  }%
}

\renewcommand\qedsymbol{\Smiley[1.3]}
\newcommand*{\paren}[1]{\ensuremath\left(#1\right)}
\newcommand*{\problem}[1]{\section*{Problem #1}}
\newcommand*{\limit}[2][x]{\ensuremath{\displaystyle\lim_{#1\to#2}}}
\newcommand*{\Limit}[3][x]{\ensuremath{\displaystyle\lim_{#1\to#2}\left[#3\right]}}
\newcommand*{\deriv}[1][x]{\ensuremath{\dfrac{\mathrm{d}}{\mathrm{d}#1}}}
\newcommand*{\Deriv}[2][x]{\ensuremath{\dfrac{\mathrm{d}}{\mathrm{d}#1}\left[#2\right]}}
\newcommand*{\abs}[1]{\ensuremath{\left|#1\right|}}

\newcommand*{\eps}{\varepsilon}

\DeclareMathOperator{\DNE}{DNE}

%opening
\title{Problem Set \#$55\dfrac{1}{2}$}
\author{Jayden Li}
\date{March 22, 2024}

\allowdisplaybreaks

\begin{document}
\setstretch{1.25}
\fontsize{12pt}{12pt}\selectfont
\setlength{\abovedisplayskip}{0pt}
\maketitle

\problem{2}
\begin{align*}
	&|\tan x-1|<0.2 \\
	\implies{}&-0.2<\tan x-1<0.2 \\
	\implies{}&0.8<\tan x<1.2 \\
	\implies{}&\arctan(0.8)<x<\arctan(1.2) \\
	&\abs{x-\frac{\pi}{4}}<\delta \\
	\implies{}&-\delta<x-\frac{\pi}{4}<\delta \\
	\implies{}&\frac{\pi}{4}-\delta<x<\frac{\pi}{4}+\delta
\end{align*}
\begin{align*}
	\delta&=\min\paren{\frac{\pi}{4}+\arctan(0.8),\arctan(1.2)-\frac{\pi}{4}} \\
	&=\boxed{\arctan(1.2)-\frac{\pi}{4}}
\end{align*}

\problem{3}
\begin{itemize}
	\item[(a)]
	\begin{equation*}
		r^2\pi=1000\implies\boxed{r=\sqrt{\frac{1000}{\pi}}\mathrm{cm}}
	\end{equation*}

	\item[(b)]
	\begin{gather*}
		995<r^2\pi<1005 \\
		\frac{995}{\pi}<r^2<\frac{1005}{\pi} \\
		\sqrt{\frac{995}{\pi}}<r<\sqrt{\frac{1005}{\pi}} \\
		\sqrt{\frac{1000}{\pi}}-\delta<r<\sqrt{\frac{1000}{\pi}}+\delta \\
		\delta=\min\paren{\sqrt{\frac{1000}{\pi}}-\sqrt{\frac{995}{\pi}},\sqrt{\frac{1005}{\pi}}-\sqrt{\frac{1000}{\pi}}} \\
		\delta\approx\boxed{0.044547\mathrm{cm}}
	\end{gather*}

	\item[(c)]
	$x$ is the area. $f$ is the function that calculates the radius needed for a circle with radius $x$. $a$ is the desired radius that $x$ tends to. As $x$ tends to $a$, $L$ tends to $f(x)$. $\eps=5\mathrm{cm}$. $\delta\approx0.044547\mathrm{cm}$.
\end{itemize}

\problem{4}
\begin{align*}
	|4x-8|&<\eps \\
	4|x-2|&<\eps \\
	|x-2|&<\frac{\eps}{4}
\end{align*}

Obviously/clearly/trivially $\delta=\dfrac{\eps}{4}$.

\begin{proof}
	Let $\delta=\dfrac{\eps}{4}$. If $|x-2|<\delta$, $4|x-2|<4\cdot\dfrac{\eps}{4}$. Therefore $|4x-8|<\eps$.
\end{proof}

\begin{itemize}
	\item[(a)]
	$\delta=\dfrac{\eps}{4}=\dfrac{0.1}{4}=\boxed{0.025}$

	\item[(b)]
	$\delta=\dfrac{\eps}{4}=\dfrac{0.01}{4}=\boxed{0.0025}$
\end{itemize}

\problem{5}
\begin{itemize}
	\item[(b)]
	\begin{align*}
		\abs{\frac{x^2+x-6}{x-2}-5}<\eps \\
		\abs{\frac{(x+3)\cancel{(x-2)}-5}{\cancel{x-2}}}<\eps \\
		\abs{x+3-5}<\eps \\
		\abs{x-2}<\eps \\
		\abs{x-2}<\delta
	\end{align*}
	I claim that $\delta=\eps$.

	\begin{proof}
		Let $\delta=\eps$.
		\begin{equation*}
			\abs{x-2}<\delta
			\implies\abs{x+3-5}<\eps
			\implies\abs{\frac{(x+3)(x-2)}{x-2}-5}<\eps
			\implies\abs{\frac{x^2+x-6}{x-2}-5}<\eps
		\end{equation*}
	\end{proof}


	\item[(d)]
	Because $x^2$ and $\sqrt{x}$ are always positive:
	\begin{equation*}
		\abs{x^2-0}<\eps
		\implies |x^2|<\eps
		\implies |x-0|<\sqrt{\eps}
	\end{equation*}
	I claim that $\delta=\sqrt{\eps}$.

	\begin{proof}
		Let $\delta=\sqrt{\eps}$.
		\begin{equation*}
			\abs{x-0}<\delta
			\implies\abs{x}<\sqrt{\eps} \\
			\implies\paren{\abs{x}}^2<\eps \\
			\implies\abs{x^2-0}<\eps
		\end{equation*}
	\end{proof}

	\item[(e)]
	\begin{equation*}
		\abs{\abs{x}-0}<\eps
		\implies |x|<\eps
		\implies |x-0|<\eps
	\end{equation*}
	I claim that $\delta=\eps$.

	\begin{proof}
		Let $\delta=\eps$.
		\begin{equation*}
			\abs{x-0}<\delta
			\implies\abs{\abs{x-0}}<\eps
			\implies\abs{\abs{x}-0}<\eps
		\end{equation*}
	\end{proof}

	\item[(f)]
	\begin{equation*}
		\abs{x^2-4x+5-1}<\eps
		\implies\abs{x^2-4x+4}<\eps
		\implies\abs{(x-2)^2}<\eps
		\implies\abs{x-2}<\sqrt{\eps}
	\end{equation*}
	I claim that $\delta=\sqrt{\eps}$.

	\begin{proof}
		Let $\delta=\sqrt{\eps}$.
		\begin{gather*}
			\abs{x-2}<\delta
			\implies\abs{(x-2)^2}<\delta^2
			\implies\abs{x^2-4x+4}<\eps
			\implies\abs{x^2-4x+5-1}<\eps
		\end{gather*}
	\end{proof}

	\item[(g)]
	\begin{equation*}
		\abs{x^2-1-3}<\eps
		\implies\abs{x+2}\abs{x-2}<\eps \tag{1}
	\end{equation*}
	Suppose $\delta=1$. By definition:
	\begin{equation*}
		|x+2|<\delta
		\implies-1<x+2<1
		\implies-5<x-2<-3
		\implies\abs{x-2}<5 \tag{2}
	\end{equation*}
	Substitute (2) into (1):
	\begin{equation*}
		5\abs{x+2}<\eps
		\implies\abs{x+2}<\frac{\eps}{5}
	\end{equation*}
	I claim that $\delta=\min\paren{1,\eps/5}$.
	\begin{proof}
		Let $\delta=\min\paren{1,\eps/5}$. If $\abs{x+2}<\delta$ then all of the following are true:
		\begin{align*}
			\abs{x+2}&<1 \tag{3} \\
			\abs{x+2}&<\frac{\eps}{5} \tag{4}
		\end{align*}

		From (3) we have:
		\begin{equation*}
			\abs{x+2}<1
			\implies-1<x+2<1
			\implies-5<x-2<-3
			\implies\abs{x-2}<5 \tag{5}
		\end{equation*}

		Multiple (3) and (5):
		\begin{equation*}
			\abs{x-2}\abs{x+2}=\abs{x^2-16}<5\cdot\frac{\eps}{5}=\eps
		\end{equation*}

		Thus by the $\eps$-$\delta$ definition $\Limit{-2}{x^2-1}=3$.
	\end{proof}

\end{itemize}

\problem{7}
\textbf{Lemma.} Density of rational and irrational numbers: there exists both a rational and irrational number on the open interval $(x,y)$ where $x<y$. (by obvious intuition)

\begin{proof}
	Suppose not. Let $L=\limit{0}f(x)$ and choose $\eps=0.1$. By the $\eps$-$\delta$ definition there must exist some $\delta>0$ such that $|x|<\delta$. On the open interval $(-\delta,\delta)$, there exists both a rational number $p$ and an irrational number $q$ (by the Lemma).
	
	On the interval $(-\delta,\delta)$, we have $\abs{f(p)-f(q)}=1\not<\eps=0.1$. We have a contradiction and the limit $\limit{0}f(x)$ does not exist.
\end{proof}

\problem{8}
\begin{align*}
	\frac{1}{(x+3)^4}&>10000 \\
	1&>10000(x+3)^4 \\
	\sqrt[4]{1}&>\sqrt[4]{10000(x+3)^4} \\
	1&>10(x+3) \\
	-29&>10x \\
	\Aboxed{x&<-\frac{29}{10}}
\end{align*}

\end{document}
