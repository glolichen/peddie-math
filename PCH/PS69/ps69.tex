\documentclass[preview, margin=0.6in]{standalone}

\usepackage[letterpaper,portrait,top=0.4in, left=0.6in, right=0.6in, bottom=1in]{geometry}

\usepackage{amsmath, amsfonts, amsthm, amssymb}
\usepackage{graphicx, float}
\usepackage{mathtools}
\usepackage{titlesec}
\usepackage{interval}
\usepackage{hyperref}
\usepackage{siunitx}
\usepackage{titling}
\usepackage{vwcol}
\usepackage{setspace}
\usepackage{empheq}
\usepackage{cancel}
\usepackage{esdiff}
\usepackage{multicol}
\usepackage{mdframed}
\usepackage{esdiff}
\usepackage{tikzsymbols}
\usepackage{multicol}
\usepackage{tikz}
\usepackage{varwidth}
\usepackage{pgfplots}

\intervalconfig {
	soft open fences
}

\newcommand{\alignedintertext}[1]{%
  \noalign{%
    \vskip\belowdisplayshortskip
    \vtop{\hsize=\linewidth#1\par
    \expandafter}%
    \expandafter\prevdepth\the\prevdepth
  }%
}

\newtheorem{lemma}{Lemma}

\renewcommand{\qedsymbol}{\Smiley[1.3]}
\newcommand*{\paren}[1]{\ensuremath\left(#1\right)}
\newcommand*{\problem}[1]{\section*{Problem #1}}
\newcommand*{\aps}{\section*{AP Corner}}
\newcommand*{\limit}[2][x]{\ensuremath{\displaystyle\lim_{#1\to#2}}}
\newcommand*{\Limit}[3][x]{\ensuremath{\displaystyle\lim_{#1\to#2}\left[#3\right]}}
\newcommand*{\deriv}[1][x]{\ensuremath{\dfrac{\mathrm{d}}{\mathrm{d}#1}}}
\newcommand*{\Deriv}[2][x]{\ensuremath{\dfrac{\mathrm{d}}{\mathrm{d}#1}\left[#2\right]}}
\newcommand*{\iinteg}[2][x]{\ensuremath{\displaystyle\int #2\;\mathrm{d}#1}}
\newcommand*{\dinteg}[4][x]{\ensuremath{\displaystyle\int_{#2}^{#3}#4\;\mathrm{d}#1}}
\newcommand*{\abs}[1]{\ensuremath{\left|#1\right|}}
\newcommand*{\eps}{\varepsilon}
\newcommand*{\floor}[1]{\ensuremath{\lfloor #1\rfloor}}
\newcommand*{\cbrt}[1]{\ensuremath{\sqrt[3]{#1}}}
\newcommand*{\lheqzero}{\ensuremath{\underset{\text{L'H}}{\overset{\left[\frac00\right]}{=}}}}
\newcommand*{\lheqinfty}{\ensuremath{\underset{\text{L'H}}{\overset{\left[\frac{\infty}{\infty}\right]}{=}}}}

\DeclareMathOperator{\DNE}{DNE}
\DeclareMathOperator{\sgn}{sgn}

\setlength{\parindent}{0pt}

%opening
\title{\vspace*{-30pt}Problem Set \#69}
\author{Jayden Li}
\date{May 12, 2024}

% \allowdisplaybreaks
\postdisplaypenalty=100000

\begin{document}
\setstretch{1.25}
\fontsize{12pt}{12pt}\selectfont
\setlength{\abovedisplayskip}{0pt}
\maketitle

\problem{2}
Let $s$ be the side length of the square.
\begin{align*}
	V&=s^2 \\
	\diff Vt&=2s\diff st
	=2\cdot4\cdot6
	=48
\end{align*}
The area of the square is increasing at a rate of $48\,cm^2/s$.

\problem{4}
\begin{align*}
	y&=x^3+2x \\
	\diff yt&=3x^2\diff xt+2\diff xt
	=3\cdot2^2\cdot5+2\cdot5=\boxed{70}
\end{align*}

\problem{6}
Let $d_1$ be the ground distance between the plane and the radio station and let $d_2$ be the actual distance.
\begin{align*}
	d_2^2&=1^2+d_1^2 \\
	2d_2\diff{d_2}{t}&=2d_1\diff{d_1}{t} \\
	\sqrt{1+2^2}\diff{d_2}{t}&=2\cdot500 \\
	\diff{d_2}{t}&=\frac{1000}{\sqrt{5}}\cdot\frac{\sqrt{5}}{\sqrt{5}} \\
	\diff{d_2}{t}&=\boxed{200\sqrt{5}\,mi/h}
\end{align*}
\textit{This is not a very well worded question -- is the actual distance (accounting for altitude) between the plane and the radio station or the ground distance 2 miles? I took it as the latter, so I had $d_1=2$ instead of $d_2=2$.}

\problem{8}
Let $s$ be the distance traveled by the first car and $w$ the distance traveled by the second.
\begin{align*}
	D^2&=s^2+w^2 \\
	2D\diff Dt&=2s\diff st+2w\diff wt \\
	\sqrt{(2\cdot60^2)+(2\cdot25)^2}\diff Dt&=(2\cdot60)(60)+(2\cdot25)(25) \\
	130\diff Dt&=8450 \\
	\diff Dt&=65
\end{align*}
The cars are moving apart at a rate of $65\,mi/h$.

\problem{10}
Let $h$ be the altitude of the triangle and $b$ be the base.
\begin{align*}
	A&=\frac12hb \\
	\diff At&=\frac12 \diff ht b+\frac12 h \diff bt \\
	2&=\frac12 20+\frac12 10 \diff bt \\
	5\diff bt&=2-10 \\
	\diff bt&=-\frac85
\end{align*}
The base of the triangle is decreasing at a rate of $1.6\,cm/min$.

\problem{12}
Let $h$ be the height of the water. The water in the tank has the shape of a cone with height $h$, and by similar triangles we see that the radius of the cone must be $h/3$.

Let $V$ be the volume of the cone representing the water.
\begin{align*}
	V&=\frac13\pi\paren{\frac h3}^2h \\
	&=\frac{\pi h^3}{27} \\
	\diff Vt&=\frac{3\pi h^2}{27}\diff ht
	=\frac{\pi(200)^2}{9}(20)
	\approx 279252.680
\end{align*}
Let $x$ be the rate at which water is being pumped into the tank.
\begin{equation*}
	x-10000=279252.680 \implies x=289252.680
\end{equation*}
Water is being pumped into the tank at a rate of $289252.680\,cm^3/min$.

\problem{14}
Let $x$ be the diameter and height of the pile of gravel. Then $x/2$ is the radius of the cone.
\begin{align*}
	V&=\frac13\pi\paren{\frac x2}^2x \\
	&=\frac{\pi x^3}{12} \\
	\diff Vt&=\frac{3\pi x^2}{12}\diff xt \\
	30&=\frac{\pi 10^2}{4}\diff xt \\
	\diff xt&=\frac{6}{5\pi}
\end{align*}
The height of the pile is increasing at a rate of $6/5\pi\,cm/min$.

\problem{18}
Let $\sin_d$ and $\cos_d$ be sine and cosine functions where the argument is in degrees. We use $\sin_r$ and $\cos_r$ to denote standard sine and cosine functions. We start by finding the derivatives of $\sin_d$ and $\cos_d$.
\begin{align*}
	\deriv\sin_d x
	&=\deriv\sin_r\paren{\frac{\pi}{180}x}
	=\frac{\pi}{180}\cos_r\paren{\frac{\pi}{180}x}
	=\frac{\pi}{180}\cos_d x \\
	\deriv\cos_d x
	&=\deriv\cos_r\paren{\frac{\pi}{180}x}
	=-\frac{\pi}{180}\sin_r\paren{\frac{\pi}{180}x}
	=-\frac{\pi}{180}\sin_d x
\end{align*}
Let $x$ be the third side of the triangle. By the law of cosines:
\begin{align*}
	x^2&=12^2+15^2-2(12)(15)\cos_d\theta \\
	&=369-360\cos_d\theta \\
	2x\diff xt&=-360\paren{-\frac{\pi}{180}\sin_d\theta}\diff{\theta}{t} \\
	\sqrt{369-360\cos_d(60)}\diff xt&=\pi\sin_d(60)(2) \\
	\diff xt&\approx0.396
\end{align*}
The length of the third side is increasing at a rate of $0.396\,cm/min$.

\problem{19}
Let $h$ be the height of the rocket, let $\theta$ be the rotation of the camera and let $d$ be the distance from the camera to the rocket.

\begin{itemize}
	\item[(a)]
	\begin{align*}
		d^2&=4000^2+h^2 \\
		2d\diff dt&=2h\diff ht \\
		\sqrt{4000^2+3000^2}\diff dt&=3000\cdot600 \\
		\diff dt&=360
	\end{align*}
	The distance between the rocket and the camera is increasing at a rate of $360\,ft/s$.
	
	\item[(b)]
	\begin{align*}
		\tan\theta&=\frac{h}{4000} \implies \theta=\arctan\frac{h}{4000} \\
		\sec^2(\theta)\diff{\theta}{t}&=\frac{1}{4000}\diff ht \\
		\sec^2\paren{\arctan\frac{3000}{4000}}\diff{\theta}{t}&=\frac{600}{4000} \\
		\diff{\theta}{t}&=0.096
	\end{align*}
	The angle of the camera is changing at a rate of $0.096\,rad/s$.
\end{itemize}

\problem{20}
Let $\theta$ be the angle of elevation of the telescope and let $d$ be the distance between the plane and telescope.
\begin{align*}
	\tan\theta&=\frac{5}{d} \implies \frac{1}{d^2}=\paren{\frac{\tan\theta}{5}}^2 \\
	\sec^2(\theta)\diff{\theta}{t}&=-\frac{5}{d^2}\diff dt \\
	\sec^2\paren{\frac{\pi}{3}}\paren{-\frac{\pi}{6}}&=-5\paren{\frac{\tan\frac{\pi}{3}}{5}}^2\diff dt \\
	\frac{4\pi}{6}&=\frac35\diff dt \\
	\diff dt&=\frac{10\pi}{9}
\end{align*}
The plane is traveling at a speed of $10\pi/9\,km/min$.

\problem{21}


\problem{22}

\end{document}
