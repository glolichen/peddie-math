% JUMP TO LINE 60, 75
\documentclass[preview, margin=0.6in]{standalone}
\usepackage[letterpaper,portrait,top=0.4in, left=0.6in, right=0.6in, bottom=1in]{geometry}

\usepackage{amsmath, amsfonts, amsthm, amssymb}
\usepackage{graphicx, float}
\usepackage{mathtools}
\usepackage{titlesec}
\usepackage{interval}
\usepackage{hyperref}
\usepackage{siunitx}
\usepackage{titling}
\usepackage{vwcol}
\usepackage{setspace}
\usepackage{empheq}
\usepackage{cancel}
\usepackage{esdiff}
\usepackage{multicol}
\usepackage{mdframed}
\usepackage{esdiff}
\usepackage{tikzsymbols}
\usepackage{multicol}
\usepackage{tikz}
\usepackage{varwidth}
\usepackage{parskip}
\usepackage{pgfplots}
\pgfplotsset{compat=1.18}
\intervalconfig {
	soft open fences
}

\newcommand{\alignedintertext}[1]{%
  \noalign{%
    \vskip\belowdisplayshortskip
    \vtop{\hsize=\linewidth#1\par
    \expandafter}%
    \expandafter\prevdepth\the\prevdepth
  }%
}

\newtheorem{lemma}{Lemma}

\renewcommand{\qedsymbol}{\Smiley[1.3]}
\newcommand*{\problem}[1]{\section*{Problem #1}}
\newcommand*{\aps}{\section*{AP Corner}}
\newcommand*{\deriv}[1][x]{\ensuremath{\dfrac{\mathrm{d}}{\mathrm{d}#1}}}
\newcommand*{\floor}[1]{\ensuremath{\lfloor #1\rfloor}}
\newcommand*{\lheqzero}{\ensuremath{\underset{\text{L'H}}{\overset{\left[\frac00\right]}{=}}}}
\newcommand*{\lheqinfty}{\ensuremath{\underset{\text{L'H}}{\overset{\left[\frac{\infty}{\infty}\right]}{=}}}}

\DeclareMathOperator{\DNE}{DNE}
\DeclareMathOperator{\sgn}{sgn}

\DeclareMathOperator{\arccsc}{arccsc}
\DeclareMathOperator{\arcsec}{arcsec}
\DeclareMathOperator{\arccot}{arccot}

%opening

\title{\vspace*{-30pt}Problem Set \#38}
\author{Jayden Li}
\date{\today}

% \allowdisplaybreaks
\postdisplaypenalty=100000

\begin{document}
\setstretch{1.25}
\fontsize{12pt}{12pt}\selectfont
\setlength{\abovedisplayskip}{0pt}
\setlength{\parindent}{0pt}
\setlength{\parskip}{2ex plus 0.5ex minus 0.2ex}
\maketitle

\problem{5}
\begin{itemize}
	\item[(k)]
		Let $\displaystyle f(x)=x\sin \left(\frac 1x\right)$. Then $a_n=f(n)$.
		\begin{equation*}
		    \lim_{x\to\infty}f(x)=\lim_{x\to\infty}x\sin \left(\frac 1x\right)=\lim_{y\to 0^+}\frac{\sin y}{y}=1
		\end{equation*}
		Because the limit $\lim_{x\to\infty}f(x)$ exists, $\{a_n\}$ converges.

	\item[(l)]
		Let $\displaystyle f(x)=\left(1+\frac{2}{x}\right)^x$. Then $a_n=f(n)$.
		\begin{align*}
			\ln \lim_{x\to\infty}f(x)&=\lim_{x\to\infty}x\ln\left(1+\frac 2x\right)=\lim_{y\to 0^+}\frac{\ln(1+2y)}{y}=\lim_{y\to0^+}\frac{2}{1+2y}=2 \\
			\lim_{x\to\infty}f(x)&=e^2
		\end{align*}
		Because the limit $\lim_{x\to\infty}f(x)$ exists, $\{a_n\}$ converges.

	\item[(m)]
		Let $\displaystyle f(x)=\ln \left(2x^2+1\right)-\ln\left(x^2+1\right)$. Then $a_n=f(n)$.
		\begin{align*}
			\exp \lim_{x\to\infty}f(x)&=\lim_{x\to\infty}\frac{2x^2+1}{x^2+1}=\lim_{x\to\infty}\frac{4\cancel x}{2\cancel x}=2 \\
			\lim_{x\to\infty}f(x)&=\ln2
		\end{align*}
		Because the limit $\lim_{x\to\infty}f(x)$ exists, $\{a_n\}$ converges.

	\item[(n)]
		$\{a_n\}$ diverges, pick $0<\varepsilon<1$ and epsilon-delta definition will fail.

	\item[(o)]
		\begin{equation*}
			a_{\infty}=\lim_{n\to\infty}\frac{n!}{2^n}=\lim_{n\to\infty}\frac{n\times(n-1)\times\ldots\times3\times2\times1}{2\times2\times\ldots\times2\times2}=\lim_{n\to\infty}\frac{n}{2}\frac{n-1}{2}\ldots \frac{3}{2}\frac{1}{2}
		\end{equation*}
		$n/2$ tends to infinity as $n$ tends to infinity. Assuming the rest are finite the product tends to infinity. So $\{a_n\}$ diverges.
\end{itemize}

\problem{6}
\begin{itemize}
	\item[(a)]
		\begin{align*}
			a_1&=1060 \\ 
			a_2&=1123.6 \\
			a_3&=1191.016 \\
			a_4&\approx1262.477 \\
			a_5&\approx1338.226
		\end{align*}

	\item[(b)]
		Divergent. $1.06^n$ tends to infinity as $n$ tends to infinity.
\end{itemize}

\problem{7}
\begin{itemize}
	\item[(a)]
		$-1$ create an alternating sequence between $-1$ and $1$ which does not converge, other negative values greater than $-1$ will still alternate but tend to $0$. Values greater than $1$ grow without bound. Converges when $-1<r\leq1$.
	\item[(b)]
		$-1<r<1$.
\end{itemize}

\end{document}
