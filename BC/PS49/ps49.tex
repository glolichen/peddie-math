% JUMP TO LINE 60, 75
\documentclass[preview, margin=0.6in]{standalone}
\usepackage[letterpaper,portrait,top=0.4in, left=0.6in, right=0.6in, bottom=1in]{geometry}

\usepackage{amsmath, amsfonts, amsthm, amssymb}
\usepackage{graphicx, float}
\usepackage{mathtools}
\usepackage{titlesec}
\usepackage{interval}
\usepackage{hyperref}
\usepackage{siunitx}
\usepackage{titling}
\usepackage{vwcol}
\usepackage{setspace}
\usepackage{empheq}
\usepackage{cancel}
\usepackage{esdiff}
\usepackage{multicol}
\usepackage{mdframed}
\usepackage{esdiff}
\usepackage{tikzsymbols}
\usepackage{multicol}
\usepackage{tikz}
\usepackage{varwidth}
\usepackage{parskip}
\usepackage{pgfplots}
\pgfplotsset{compat=1.18}
\intervalconfig {
	soft open fences
}

\newcommand{\alignedintertext}[1]{%
  \noalign{%
    \vskip\belowdisplayshortskip
    \vtop{\hsize=\linewidth#1\par
    \expandafter}%
    \expandafter\prevdepth\the\prevdepth
  }%
}

\newtheorem{lemma}{Lemma}

\renewcommand{\qedsymbol}{\Smiley[1.3]}
\newcommand*{\problem}[1]{\section*{Problem #1}}
\newcommand*{\aps}{\section*{AP Corner}}
\newcommand*{\deriv}[1][x]{\ensuremath{\dfrac{\mathrm{d}}{\mathrm{d}#1}}}
\newcommand*{\floor}[1]{\ensuremath{\lfloor #1\rfloor}}
\newcommand*{\lheqzero}{\ensuremath{\underset{\text{L'H}}{\overset{\left[\frac00\right]}{=}}}}
\newcommand*{\lheqinfty}{\ensuremath{\underset{\text{L'H}}{\overset{\left[\frac{\infty}{\infty}\right]}{=}}}}

\DeclareMathOperator{\DNE}{DNE}
\DeclareMathOperator{\sgn}{sgn}

\DeclareMathOperator{\arccsc}{arccsc}
\DeclareMathOperator{\arcsec}{arcsec}
\DeclareMathOperator{\arccot}{arccot}

%opening

\title{\vspace*{-40pt}Problem Set \#49}
\author{Jayden Li}
\date{\today}
% \allowdisplaybreaks
\postdisplaypenalty=100000

\begin{document}
\setstretch{1.25}
\fontsize{12pt}{12pt}\selectfont
\setlength{\abovedisplayskip}{\abovedisplayskip/2}
\setlength{\belowdisplayskip}{\belowdisplayskip/2}
\setlength{\parindent}{0pt}
\setlength{\parskip}{2ex plus 0.5ex minus 0.2ex}
\maketitle

\problem{1}
\begin{itemize}
\item[(a)]
	$\begin{aligned}[t]
		L
		&=\lim_{n\to\infty}\left|\frac{a_{n+1}}{a_n}\right|
		=\lim_{n\to\infty}\left|\frac{\cos(n+1)}{(n+1)^2}\frac{n^2}{\cos n}\right|
		=\lim_{n\to\infty}\frac{n^2}{n^2+2n+1}\frac{\cos n\cos 1-\sin n \sin 1}{\cos n} \\
		&=\lim_{n\to\infty}\frac{n^2}{n^2 \left(1+\frac 2n+\frac{1}{n^2}\right)}\left(\cos1-\tan n\sin 1\right)
		=\cos 1-\sin(1)\lim_{n\to\infty}\tan(n)
		\text{ does not exist.}
	\end{aligned}$

	Not sure how would ratio test, but series is clearly absolutely convergent by direct comparison to $1/n^2$ as $\left|\cos n\right|<1$.

\item[(b)]
	$\begin{aligned}[t]
		L 
		&=\lim_{n\to\infty}\left|\frac{a_{n+1}}{a_n}\right|
		=\lim_{n\to\infty}\left|\frac{n+1}{5^{n+1}}\frac{5^n}{n}\right|
		=\lim_{n\to\infty}\frac{n \left(1+\frac1n\right)}{5n}
		=\frac15<1
	\end{aligned}$

	We have $L<1$ so the series is absolutely convergent by the ratio test.

\item[(d)]
	$\begin{aligned}[t]
		L
		&=\lim_{n\to\infty}\left|\frac{a_{n+1}}{a_n}\right|
		=\lim_{n\to\infty}\left|\frac{(k+1)\left(\frac23\right)^{k+1}}{k \left(\frac23\right)^k}\right|
		=\lim_{n\to\infty}\frac{k+1}{k}\frac23
		=\frac23<1
	\end{aligned}$

	The series is absolutely convergent by the ratio test.

\item[(e)]
	$\begin{aligned}[t]
		L
		&=\lim_{n\to\infty}\left|\frac{a_{n+1}}{a_n}\right|
		=\lim_{n\to\infty}\left|\frac{(-1)^{n+1}(1.1)^{n+1}}{(n+1)^4}\frac{n^4}{(-1)^n(1.1)^n}\right|
		=\lim_{n\to\infty}\frac{n^4}{(n+1)^4}\left|(-1)(1.1)\right|
		=1.1>1
	\end{aligned}$

	The series is divergent by the ratio test.

\item[(h)]
	$\begin{aligned}[t]
		L
		&=\lim_{n\to\infty}\left|\frac{a_{n+1}}{a_n}\right|
		=\lim_{n\to\infty}\left|\frac{(-1)^{n+1}}{\ln(n+1)}\frac{\ln n}{(-1)^n}\right|
		=\lim_{n\to\infty}\frac{\ln n}{\ln (n+1)}
		\lheqinfty \lim_{n\to\infty}\frac{\frac1n}{\frac{1}{n+1}}
		=\lim_{n\to\infty}\frac{n+1}{n}
		=1
	\end{aligned}$

	Ratio test is inconclusive.

	However notice that $\sum 1/\ln n$ is divergent (trivial) but also $1/\ln n=\left|a_n\right|$ is decreasing so by the absolute convergence test the series is conditionally convergent.

\item[(i)]
	$\begin{aligned}[t]
		L
		&=\lim_{n\to\infty}\left|\frac{a_{n+1}}{a_n}\right|
		=\lim_{n\to\infty}\left|\frac{\cos\left(\frac{\pi n}{3}+\frac{\pi}{3}\right)}{(n+1)!}\frac{n!}{\cos(\pi n/3)}\right|
		=\lim_{n\to\infty}\left|\frac{\cos \frac{\pi n}{3}\cos \frac{\pi}{3}-\sin \frac{\pi n}{3}\sin \frac{\pi}{3}}{(n+1)\cos \frac{\pi n}{3}}\right|
		=0<1
	\end{aligned}$

	The series is absolutely convergent by the ratio test.

\item[(j)]
	$\begin{aligned}[t]
		L 
		&=\lim_{n\to\infty}\sqrt[n]{\left|a_n\right|}
		=\lim_{n\to\infty}\sqrt[n]{\left(\frac{n^2+1}{2n^2+1}\right)^n}
		=\lim_{n\to\infty}\frac{n^2+1}{2n^2+1}
		=\frac12 
		<1
	\end{aligned}$

	The series is absolutely convergent by the root test.

\item[(l)]
	$\begin{aligned}[t]
		L
		&=\lim_{n\to\infty}\left|\frac{a_{n+1}}{a_n}\right|
		=\lim_{n\to\infty}\left|\frac{(n+1)^{100}\cdot 100^{n+1}}{(n+1)!}\frac{n!}{n^{100}\cdot 100^n}\right|
		=\lim_{n\to\infty}\frac{(n+1)^{100}}{n^{100}}\frac{100}{n+1}
		=1\cdot0=0<1
	\end{aligned}$

	The series is absolutely convergent by the ratio test.
\end{itemize}

\problem{2}
\begin{itemize}
	\item[(a)] Convergent with $L=0$.
	\item[(b)] Convergent with $L=2/3$.
\end{itemize}

\problem{3}
\begin{equation*}
	L 
	=\lim_{n\to\infty}\left|\frac{a_{n+1}}{a_n}\right|
	=\lim_{n\to\infty}\left|\frac{5n+1}{4n+3}a_n\cdot \frac{1}{a_n}\right|
	=\lim_{n\to\infty}\frac{5n+1}{4n+3}
	=\lim_{n\to\infty}\frac{5n \left(1+\frac{1}{5n}\right)}{4n \left(1+\frac{3}{4n}\right)}
	=\frac54>1
\end{equation*}
By the ratio test the series is divergent.

\problem{4}
Absolutely convergent (see PS48 last question).

\problem{5}
\begin{itemize}
	\item[(a)]
		\begin{equation*}
		    L 
			=\lim_{n\to\infty}\left|\frac{a_{n+1}}{a_n}\right|
			=\lim_{n\to\infty}\left|\frac{1}{(n+1)^3}\frac{n^3}{1}\right|
			\lim_{n\to\infty}\frac{n^3}{(n+1)^3}
			=1
		\end{equation*}
		Ratio test is inconclusive.

	\item[(b)]
		\begin{equation*}
		    L 
			=\lim_{n\to\infty}\left|\frac{a_{n+1}}{a_n}\right|
			=\lim_{n\to\infty}\left|\frac{(-3)^{n}}{\sqrt{n+1}}\frac{\sqrt{n}}{(-3)^{n-1}}\right|
			=\lim_{n\to\infty}\frac{3 \sqrt{n}}{\sqrt{n+1}}
			=3>1
		\end{equation*}
		The series is divergent by the ratio test.
\end{itemize}

\problem{6}
\begin{itemize}
\item[(a)]
	\begin{equation*}
		L 
		=\lim_{n\to\infty}\left|\frac{a_{n+1}}{a_n}\right|
		=\lim_{n\to\infty}\left|\frac{x^{n+1}}{(n+1)!}\frac{n!}{x^n}\right|
		=\lim_{n\to\infty}\frac{x}{n+1}
		=0<1
	\end{equation*}

	Therefore for all $x\in\mathbb R$ the series is convergent by the ratio test.

\item[(b)]
	Yeah did that.
\end{itemize}

\problem{7}
\begin{itemize}
	\item[(a)] Written on the messy previous submission.
	\item[(b)] A lot.
\end{itemize}

\end{document}
