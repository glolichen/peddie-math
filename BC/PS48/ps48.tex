% JUMP TO LINE 60, 75
\documentclass[preview, margin=0.6in]{standalone}
\usepackage[letterpaper,portrait,top=0.4in, left=0.6in, right=0.6in, bottom=1in]{geometry}

\usepackage{amsmath, amsfonts, amsthm, amssymb}
\usepackage{graphicx, float}
\usepackage{mathtools}
\usepackage{titlesec}
\usepackage{interval}
\usepackage{hyperref}
\usepackage{siunitx}
\usepackage{titling}
\usepackage{vwcol}
\usepackage{setspace}
\usepackage{empheq}
\usepackage{cancel}
\usepackage{esdiff}
\usepackage{multicol}
\usepackage{mdframed}
\usepackage{esdiff}
\usepackage{tikzsymbols}
\usepackage{multicol}
\usepackage{tikz}
\usepackage{varwidth}
\usepackage{parskip}
\usepackage{pgfplots}
\pgfplotsset{compat=1.18}
\intervalconfig {
	soft open fences
}

\newcommand{\alignedintertext}[1]{%
  \noalign{%
    % \vskip\belowdisplayshortskip
    \vtop{\hsize=\linewidth#1\par
    \expandafter}%
    \expandafter\prevdepth\the\prevdepth
  }%
}

\newtheorem{lemma}{Lemma}

\renewcommand{\qedsymbol}{\Smiley[1.3]}
\newcommand*{\problem}[1]{\section*{Problem #1}}
\newcommand*{\aps}{\section*{AP Corner}}
\newcommand*{\deriv}[1][x]{\ensuremath{\dfrac{\mathrm{d}}{\mathrm{d}#1}}}
\newcommand*{\floor}[1]{\ensuremath{\lfloor #1\rfloor}}
\newcommand*{\lheqzero}{\ensuremath{\underset{\text{L'H}}{\overset{\left[\frac00\right]}{=}}}}
\newcommand*{\lheqinfty}{\ensuremath{\underset{\text{L'H}}{\overset{\left[\frac{\infty}{\infty}\right]}{=}}}}

\DeclareMathOperator{\DNE}{DNE}
\DeclareMathOperator{\sgn}{sgn}

\DeclareMathOperator{\arccsc}{arccsc}
\DeclareMathOperator{\arcsec}{arcsec}
\DeclareMathOperator{\arccot}{arccot}

%opening

\title{\vspace*{-40pt}Problem Set \#48}
\author{Jayden Li}
\date{\today}
% \allowdisplaybreaks
\postdisplaypenalty=100000

\begin{document}
\setstretch{1.25}
\fontsize{12pt}{12pt}\selectfont
\setlength{\abovedisplayskip}{\abovedisplayskip/2}
\setlength{\belowdisplayskip}{\belowdisplayskip/2}
\setlength{\parindent}{0pt}
\setlength{\parskip}{2ex plus 0.5ex minus 0.2ex}
\maketitle

\problem{1}
\begin{itemize}
	\item[(a)]
		\begin{gather*}
			a_n=\frac{(-1)^n}{\sqrt[3]{x}^2} \\ 
			\sum_{n=1}^{\infty}|a_n|
			=\sum_{n=1}^{\infty}\frac{1}{x^{2/3}} \text{ diverges by the $p$-series test as $p=2/3\leq1$.} \\
			\lim_{n\to\infty}a_n=0 \\
			\sqrt[3]{x+1}>\sqrt[3]{x}
			\implies \sqrt[3]{x+1}^2>\sqrt[3]{x}^2
			\implies \frac{1}{\sqrt[3]{x+1}^2}=a_{n+1}<a_n=\frac{1}{\sqrt[3]{x}^2}
		\end{gather*}
		The series is not absolutely convergent since $\sum |a_n|$ diverges, but $\sum a_n$ converges by the alternating series test. So $\sum a_n$ is conditionally convergent.
	
	\item[(b)]
		\begin{gather*}
		    a_n=\frac{\cos n}{n^2} \\
			\forall n\in \mathbb Z^+:
			\left|\cos n\right|\leq1
			\implies \frac{\left|\cos n\right|}{n^2}\leq\frac{1}{n^2}
			\implies \left|\frac{\cos n}{n^2}\right|\leq\frac{1}{n^2}
		\end{gather*}
		$\sum 1/n^2$ converges by the $p$-series test since $p=2>1$. Then $\sum \cos \left|n/n^2\right|$ converges by the direct comparison test, so $\sum a_n$ is absolutely convergent.
	
	\item[(c)]
		\begin{gather*}
		    a_n=\frac{(-1)^n}{\sqrt{n^2-1}} \\
			\sum_{n=2}^{\infty}|a_n|
			=\sum_{n=2}^{\infty}\frac{1}{\sqrt{n^2-1}}
			\text{ diverges by limit comparison to $\sum \frac1n$.} \\
			\lim_{n\to\infty}\frac{1}{\sqrt{n^2-1}}=0 \\
			\begin{aligned}
				\forall \mathbb N \ni n \geq 2 :
				2n\geq -1
				&\implies n^2+2n\geq n^2-1
				\implies (n+1)^2-1\geq n^2-1 \\
				&\implies \sqrt{(n+1)^2-1}\geq \sqrt{n^2-1} \\
				&\implies \frac{1}{\sqrt{(n+1)^2-1}}=a_{n+1}\leq a_n=\frac{1}{\sqrt{n^2-1}}
			\end{aligned}
		\end{gather*}
		$\sum a_n$ converges by the alternating series test so the series conditionally converges.
\end{itemize}

\problem{2}
\begin{gather*}
    a_n=\frac{b_n^n\cos(\pi n)}{n},
	|a_n|=\frac{b_n^n}{n} \\
	\lim_{n\to\infty}\sqrt[n]{|a_n|}
	=\lim_{n\to\infty}\sqrt[n]{\frac{b_n^n}{n}}
	=\lim_{n\to\infty} \frac{b_n}{\sqrt[n]{n}}
	=\frac12 \lim_{n\to\infty}\frac{1}{n^{1/n}}
	=\frac12 \lim_{n\to\infty} n^{-1/n}=\frac12
\end{gather*}
\begin{mdframed}
	\begin{equation*}
		\ln\lim_{n\to\infty}n^{-1/n}
		=\lim_{n\to\infty}\left[-\frac1n \ln n\right]
		=-\lim_{n\to \infty}\frac{\ln n}{n}
		=0
		\implies \lim_{n\to\infty}n^{-1/n}
		=e^0=1
	\end{equation*}
\end{mdframed}

The series is absolutely convergent by the root test.

\end{document}
