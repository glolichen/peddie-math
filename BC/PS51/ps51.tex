% JUMP TO LINE 60, 75
\documentclass[preview, margin=0.6in]{standalone}
\usepackage[letterpaper,portrait,top=0.4in, left=0.6in, right=0.6in, bottom=1in]{geometry}

\usepackage{amsmath, amsfonts, amsthm, amssymb}
\usepackage{graphicx, float}
\usepackage{mathtools}
\usepackage{titlesec}
\usepackage{interval}
\usepackage{hyperref}
\usepackage{siunitx}
\usepackage{titling}
\usepackage{vwcol}
\usepackage{setspace}
\usepackage{empheq}
\usepackage{cancel}
\usepackage{esdiff}
\usepackage{multicol}
\usepackage{mdframed}
\usepackage{esdiff}
\usepackage{tikzsymbols}
\usepackage{multicol}
\usepackage{tikz}
\usepackage{varwidth}
\usepackage{parskip}
\usepackage{pgfplots}
\pgfplotsset{compat=1.18}
\intervalconfig {
	soft open fences
}

\newcommand{\alignedintertext}[1]{%
  \noalign{%
    \vtop{\hsize=\linewidth#1\par
    \expandafter}%
    \expandafter\prevdepth\the\prevdepth
  }%
}

\newtheorem{lemma}{Lemma}

\renewcommand{\qedsymbol}{\Smiley[1.3]}
\newcommand*{\problem}[1]{\section*{Problem #1}}
\newcommand*{\aps}{\section*{AP Corner}}
\newcommand*{\deriv}[1][x]{\ensuremath{\dfrac{\mathrm{d}}{\mathrm{d}#1}}}
\newcommand*{\floor}[1]{\ensuremath{\lfloor #1\rfloor}}
\newcommand*{\lheqzero}{\ensuremath{\underset{\text{L'H}}{\overset{\left[\frac00\right]}{=}}}}
\newcommand*{\lheqinfty}{\ensuremath{\underset{\text{L'H}}{\overset{\left[\frac{\infty}{\infty}\right]}{=}}}}

\DeclareMathOperator{\DNE}{DNE}
\DeclareMathOperator{\sgn}{sgn}

\DeclareMathOperator{\arccsc}{arccsc}
\DeclareMathOperator{\arcsec}{arcsec}
\DeclareMathOperator{\arccot}{arccot}

%opening

\title{\vspace*{-40pt}Problem Set \#51}
\author{Jayden Li}
\date{\today}
% \allowdisplaybreaks
\postdisplaypenalty=100000

\begin{document}
\setstretch{1.25}
\fontsize{12pt}{12pt}\selectfont
\setlength{\abovedisplayskip}{\abovedisplayskip/2}
\setlength{\belowdisplayskip}{\belowdisplayskip/2}
\setlength{\parindent}{0pt}
\setlength{\parskip}{2ex plus 0.5ex minus 0.2ex}
\maketitle

\problem{1}
\begin{itemize}
	\item[(a)]
		$\begin{aligned}[t]
			f(x)
			&=\frac{1}{1+x}
			=\frac{1}{1-(-x)}
			=\sum_{n=0}^{\infty}(-x)^n
			=\boxed{\sum_{n=0}^{\infty}(-1)^n x^n}
		\end{aligned}$

		Interval of convergence: $(-1,1)$.

	\item[(c)]
		$\begin{aligned}[t]
		    f(x)
			&=\frac{x}{9+x^2}
			=\frac{x}{9}\frac{1}{1-\left(-\frac{x^2}{9}\right)}
			=\frac{x}{9}\sum_{n=1}^{\infty}\left(-\frac{x^2}{9}\right)^n
			=\frac{x}{9}\sum_{n=1}^{\infty}(-1)^n \frac{x^{2n}}{9^n}
			=\boxed{\sum_{n=1}^{\infty}(-1)^n \frac{x^{2n+1}}{9^{n+1}}}
		\end{aligned}$
		
		Converges when $\displaystyle \left|-\frac{x^2}{9}\right|<1 \iff x^2<9 \iff \left|x\right|<3$.

		Interval of convergence: $(-3,3)$.

\end{itemize}

\problem{6}
\begin{itemize}
\item[(b)]
\phantom{}
\begin{mdframed}
	First, we calculate the power series expansion of the arctangent function.

	$\begin{aligned}[t]
		\deriv[x]\arctan x
		&=\frac{1}{1+x^2}
		=\frac{1}{1-\underbrace{\left(-x^2\right)}_{\left|-x^2\right|<1 \iff \left|x\right|<1}}
		=\sum_{n=0}^{\infty}\left(-x^2\right)^n
		=\sum_{n=0}^{\infty}(-1)^n x^{2n} \\
		\arctan x
		&=\int \deriv[x]\arctan x\,\mathrm{d}x
		=\int \sum_{n=0}^{\infty}(-1)^n x^{2n}\,\mathrm{d}x
		=\sum_{n=1}^{\infty}\frac{(-1)^n x^{2n+1}}{2n+1}+C \\
		\alignedintertext{
			\begin{equation*}
				\arctan 0
				=0
				=\sum_{n=1}^{\infty}\frac{(-1)^n 0^{2n+1}}{2n+1}+C
				=C
				\implies C=0
			\end{equation*}
		}
		&=\sum_{n=1}^{\infty}\frac{(-1)^n x^{2n+1}}{2n+1}
		=\boxed{\sum_{n=0}^{\infty}\frac{(-1)^n x^{2n+1}}{2n+1}}
	\end{aligned}$
\end{mdframed}

$\begin{aligned}[t]
	\int \frac{x-\arctan x}{x^3}\,\mathrm{d}x
	&=\int\left(\frac{1}{x^2}-\frac{1}{x^3}\sum_{n=0}^{\infty}\frac{(-1)^n x^{2n+1}}{2n+1}\right)\mathrm{d}x
	=-\frac1x-\int \sum_{n=0}^{\infty}\frac{(-1)^n x^{2n-2}}{2n+1}\,\mathrm{d}x \\
	&=-\frac1x-\sum_{n=0}^{\infty}\frac{(-1)^n x^{2n-1}}{(2n+1)(2n-1)}+C \\
	\alignedintertext{\begin{mdframed}
		\begin{align*}
			\text{LHS}
			&=\lim_{n\to0}\left[-\frac1x-\sum_{n=0}^{\infty}\frac{(-1)^n x^{2n-1}}{(2n+1)(2n-1)}+C\right]
			=\lim_{x\to0}\left[-\frac1x-x^{-1}+C\right]
			=C \\
			\text{RHS}
			&=\lim_{x\to0}\frac{x-\arctan x}{x^3}
			\lheqzero \lim_{x\to0}\frac{1-\frac{1}{1+x^2}}{3x^2}\frac{1+x^2}{1+x^2}
			=\lim_{x\to0} \frac{1+x^2-1}{3x^2 \left(1+x^2\right)}
			=\lim_{x\to0} \frac{x^2}{3x^2+3x^4} \\
			&\lheqzero \lim_{x\to0} \frac{2x}{6x+12x^3}
			\lheqzero \lim_{x\to0} \frac{2}{6+36x^2}
			=\frac13
		\end{align*}

		Therefore, $C=1/3$.
	\end{mdframed}}
	&=\boxed{-\frac1x-\sum_{n=0}^{\infty}\frac{(-1)^n x^{2n-1}}{(2n+1)(2n-1)}+\frac13}
\end{aligned}$

Radius of convergence would be the same as that of arctangent power series above, so $R=1$.

\end{itemize}

\problem{7}
\begin{itemize}
	\item[(a)]
		$\begin{aligned}[t]
		    \int_{0}^{0.2}\frac{1}{1+x^5}\,\mathrm{d}x
			&=\int_{0}^{0.2}\frac{1}{1-\left(-x^5\right)}\,\mathrm{d}x
			=\int_{0}^{0.2}\sum_{n=0}^{\infty}(-1)^n x^{5n}\,\mathrm{d}x
			=\left[\sum_{n=0}^{\infty}\frac{(-1)^n x^{5n+1}}{5n+1}\right]_{0}^{0.2} \\
			&=\sum_{n=0}^{\infty}\frac{(-1)^n (0.2)^{5n+1}}{5n+1}-\cancel{\sum_{n=0}^{\infty}\frac{(-1)^n (0)^{5n+1}}{5n+1}}
			\approx \boxed{0.199989}
		\end{aligned}$

		Interval of convergence is $(-1,1)$, both bounds $0,0.2\in(-1,1)$.

	\item[(b)]
		$\begin{aligned}[t]
		    \int_{0}^{0.1}x\arctan(3x)\,\mathrm{d}x
			&=\int_{0}^{0.1}x \sum_{n=0}^{\infty}\frac{(-1)^n (3x)^{2n+1}}{2n+1}\,\mathrm{d}x
			=\int_{0}^{0.1} \sum_{n=0}^{\infty}\frac{(-1)^n 3^{2n+1} x^{2n+2}}{2n+1}\,\mathrm{d}x \\
			&=\left[\sum_{n=0}^{\infty}\frac{(-1)^n 3^{2n+1} x^{2n+3}}{(2n+1)(2n+3)}\right]_{0}^{0.1}
			=\sum_{n=0}^{\infty}\frac{(-1)^n 3^{2n+1} (0.1)^{2n+3}}{(2n+1)(2n+3)}
			\approx \boxed{0.000983}
		\end{aligned}$

		Radius of convergence is one third of that of arctangent, so interval of convergence is $(-1/3,1/3)$. Bounds are in the interval of convergence.

\end{itemize}

\problem{8}
\begin{itemize}
	\item[(a)]
		$\begin{aligned}[t]
		    f'(x)
			=\deriv \sum_{n=0}^{\infty}\frac{x^n}{n!}
			=\sum_{n=1}^{\infty}\frac{nx^{n-1}}{n(n-1)!}
			=\sum_{n=1}^{\infty}\frac{x^{n-1}}{(n-1)!}
			=\sum_{n=0}^{\infty}\frac{x^{n}}{n!}
			=f(x)
		\end{aligned}$

	\item[(b)]
		\phantom{}
		\begin{mdframed}
			Let $k$ be a positive integer.

			$\begin{aligned}[t]
			    \lim_{n\to\infty}\frac{n!}{(n-k)!n^k}
				=\lim_{n\to\infty}\frac{\overbrace{n(n-1)(n-2)\times\ldots}^{\text{$k$ terms}} \times \cancel{(n-k)!}}{\underbrace{n\times n\times\ldots\times n}_{\text{$k$ terms}} \cancel{(n-k)!}}
				=\lim_{n\to\infty}\frac{n}{n}\frac{n-1}{n}\frac{n-2}{n}\ldots \frac{n-k+1}{n}
				=1
			\end{aligned}$
		\end{mdframed}
		$\begin{aligned}[t]
		    e^x
			&=\lim_{n\to\infty}\left(1+\frac{x}{n}\right)^n
			=\lim_{n\to\infty}\sum_{k=0}^{n}\binom nk \left(\frac{x}{n}\right)^k
			=\lim_{n\to\infty}\sum_{k=0}^{n}\frac{\cancel{n!}}{k!\cancel{(n-k)!}} \frac{x^k}{\cancel{n^k}}
			=\lim_{n\to\infty}\sum_{k=0}^{n}\frac{x^k}{k!}
			=\sum_{n=0}^{\infty}\frac{x^n}{n!}
		\end{aligned}$

\end{itemize}

\problem{9}

We know that:
\begin{equation*}
    \tan \frac{\pi}{6}
	=\frac{\sin \frac{\pi}{6}}{\cos \frac{\pi}{6}}
	=\frac{\frac{1}{2}}{\frac{\sqrt{3}}{2}}
	=\frac{1}{\sqrt{3}}
	\iff
	\arctan\left(\tan \frac{\pi}{6}\right)
	=\arctan\left(\frac{1}{\sqrt{3}}\right)
	\iff
	\frac{\pi}{6}
	=\arctan\left(\frac{1}{\sqrt{3}}\right)
\end{equation*}
\begin{align*}
	\pi
	&=6 \arctan\left(\frac{1}{\sqrt{3}}\right)
	=6 \sum_{n=0}^{\infty}\frac{(-1)^n \left(\frac{1}{\sqrt{3}}\right)^{2n+1}}{2n+1}
	=6 \sum_{n=0}^{\infty}\frac{(-1)^n 3^{-\frac{2n+1}2}}{2n+1}
	=6 \sum_{n=0}^{\infty}\frac{(-1)^n}{(2n+1) 3^{n+1/2}} \\
	&=\frac{6}{3^{1/2}}\frac{\sqrt{3}}{\sqrt{3}} \sum_{n=0}^{\infty}\frac{(-1)^n}{(2n+1) 3^{n}}
	=\frac{6\sqrt 3}{3}\sum_{n=0}^{\infty}\frac{(-1)^n}{(2n+1) 3^{n}}
	=2 \sqrt{3}\sum_{n=0}^{\infty}\frac{(-1)^n}{(2n+1) 3^{n}}
\end{align*}

\end{document}
