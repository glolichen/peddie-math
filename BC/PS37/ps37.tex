% JUMP TO LINE 60, 75
\documentclass[preview, margin=0.6in]{standalone}
\usepackage[letterpaper,portrait,top=0.4in, left=0.6in, right=0.6in, bottom=1in]{geometry}

\usepackage{amsmath, amsfonts, amsthm, amssymb}
\usepackage{graphicx, float}
\usepackage{mathtools}
\usepackage{titlesec}
\usepackage{interval}
\usepackage{hyperref}
\usepackage{siunitx}
\usepackage{titling}
\usepackage{vwcol}
\usepackage{setspace}
\usepackage{empheq}
\usepackage{cancel}
\usepackage{esdiff}
\usepackage{multicol}
\usepackage{mdframed}
\usepackage{esdiff}
\usepackage{tikzsymbols}
\usepackage{multicol}
\usepackage{tikz}
\usepackage{varwidth}
\usepackage{parskip}
\usepackage{pgfplots}
\pgfplotsset{compat=1.18}
\intervalconfig {
	soft open fences
}

\newcommand{\alignedintertext}[1]{%
  \noalign{%
    \vskip\belowdisplayshortskip
    \vtop{\hsize=\linewidth#1\par
    \expandafter}%
    \expandafter\prevdepth\the\prevdepth
  }%
}

\newtheorem{lemma}{Lemma}

\renewcommand{\qedsymbol}{\Smiley[1.3]}
\newcommand*{\problem}[1]{\section*{Problem #1}}
\newcommand*{\aps}{\section*{AP Corner}}
\newcommand*{\deriv}[1][x]{\ensuremath{\dfrac{\mathrm{d}}{\mathrm{d}#1}}}
\newcommand*{\floor}[1]{\ensuremath{\lfloor #1\rfloor}}
\newcommand*{\lheqzero}{\ensuremath{\underset{\text{L'H}}{\overset{\left[\frac00\right]}{=}}}}
\newcommand*{\lheqinfty}{\ensuremath{\underset{\text{L'H}}{\overset{\left[\frac{\infty}{\infty}\right]}{=}}}}

\DeclareMathOperator{\DNE}{DNE}
\DeclareMathOperator{\sgn}{sgn}

\DeclareMathOperator{\arccsc}{arccsc}
\DeclareMathOperator{\arcsec}{arcsec}
\DeclareMathOperator{\arccot}{arccot}

%opening

\title{\vspace*{-30pt}Problem Set \#37}
\author{Jayden Li}
\date{\today}

% \allowdisplaybreaks
\postdisplaypenalty=100000

\begin{document}
\setstretch{1.25}
\fontsize{12pt}{12pt}\selectfont
\setlength{\abovedisplayskip}{0pt}
\setlength{\parindent}{0pt}
\setlength{\parskip}{2ex plus 0.5ex minus 0.2ex}
\maketitle

\problem{3}
\begin{itemize}
	\item[(a)]
		At any time, the distance traveled by the dog is equal to the distance traveled by the rabbit. Also note that because the dog always runs in a straight line to the rabbit, the tangent line to $f(x)$ passes through the rabbit.

		The distance traveled by the dog and rabbit can be calculated using the arc length formula. This is the $y$-coordinate of the rabbit. The $y$-coordinate of the dog equals $f(x)$. The $x$-difference between the animals is always $x$. Therefore:

		$\begin{aligned}[t]
			&f'(x)=\frac1x\left(f(x)-\int_{x}^{L}\sqrt{1+\left(f'(t)\right)^2}\,\mathrm{d}t\right)
			\implies xf'(x)=f(x)-\int_{x}^{L}\sqrt{1+\left(f'(t)\right)^2}\,\mathrm{d}t \\
			\implies{}& \deriv[x]\left[f(x)-xf'(x)\right]=\deriv[x]\int_{x}^{L}\sqrt{1+\left(f'(t)\right)^2}\,\mathrm{d}t
			\implies f'(x)-f'(x)-xf''(x)=-\sqrt{1+(f'(x))^2} \\
			\implies{}& xf''(x)=\sqrt{1+(f'(x))^2}
			\implies \boxed{x\diff[2]yx=\sqrt{1+\left(\diff yx\right)^2}}
		\end{aligned}$

	\item[(b)]
		$\begin{aligned}[t]
			x\diff[2]yx=\sqrt{1+\left(\diff yx\right)^2}
			&\implies \left[\begin{aligned}
					z&=\diff yx \\
					\diff zx&=\diff[2]yx
			\end{aligned}\right]
			x\diff zx=\sqrt{1+z^2}
			\implies \frac{1}{\sqrt{1+z^2}}\diff zx=\frac1x \\
			&\implies \int\frac{1}{\sqrt{1+z^2}}\,\mathrm{d}z=\int \frac1x\,\mathrm{d}x=\ln|x| \\
			\alignedintertext{
			\begin{mdframed}
				$\begin{aligned}[t]
				    \int \frac{1}{\sqrt{1+z^2}}\,\mathrm{d}z
					&=\left[\begin{aligned}
							z&=\tan\theta \\
							\mathrm{d}z&=\sec^2\theta
					\end{aligned}\right]
					\int \frac{\sec^2\theta}{\sqrt{1+\tan^2\theta}}\,\mathrm{d}\theta
					=\int \sec\theta\,\mathrm{d}\theta
					=\ln|\sec\theta+\tan\theta|+C_0 \\
					&=\ln \left|\sqrt{z^2+1}+z\right|+C_0
				\end{aligned}$
			\end{mdframed}
			}
			&\implies \ln \left|\sqrt{z^2+1}+z\right|+C_0=\ln|x|
			\implies C_0\sqrt{z^2+1}+C_0z=x \\
			&\implies C_0^2\left(z^2+1\right)=x^2-2C_0xz+C_0^2z^2
			\implies C_0^2=x^2-2C_0xz \\
			&\implies 2C_0xz=x^2-C_0^2
			\implies z=\frac{x^2-C_0^2}{2C_0x} \\
			z(L)=0
			&\implies\frac{L^2-C_0^2}{2C_0L}=0
			\implies L^2-C_0^2=0
			\implies C_0^2=L^2
			\implies C_0=\pm L \\
			z=\frac{x^2-L^2}{2Lx}
			&\implies \begin{aligned}[t]
				y&=\int \frac{x^2-L^2}{2Lx}\,\mathrm{d}x
				=\frac12\int\left(\frac{x}{L}-\frac{L}{x}\right)\mathrm{d}x
				=\frac12 \left(\frac{x^2}{2L}-L\ln|x|\right)+C_1
			\end{aligned} \\
			y(0)=0 &\implies\frac12 \left(\frac{0}{2L}-L\right)+C_1
			=-\frac{L}{2}+C_1=0
			\implies C_1=\frac{L}{2} \\
		   &\implies y=\frac12 \left(\frac{x^2}{2L}-L\ln|x|\right)+\frac{L}2
		   =\boxed{\frac{x^2}{4L}-\frac{L\ln|x|}{2}+\frac{L}{2}}
		\end{aligned}$

	\item[(c)] No. The above function is not defined at $x=0$, and because the $x$-component of the rabbit's position always equals $0$, the dog will never reach it.

\end{itemize}

\end{document}
