% JUMP TO LINE 60, 75
\documentclass[preview, margin=0.6in]{standalone}
\usepackage[letterpaper,portrait,top=0.4in, left=0.6in, right=0.6in, bottom=1in]{geometry}

\usepackage{amsmath, amsfonts, amsthm, amssymb}
\usepackage{graphicx, float}
\usepackage{mathtools}
\usepackage{titlesec}
\usepackage{interval}
\usepackage{hyperref}
\usepackage{siunitx}
\usepackage{titling}
\usepackage{vwcol}
\usepackage{setspace}
\usepackage{empheq}
\usepackage{cancel}
\usepackage{esdiff}
\usepackage{multicol}
\usepackage{mdframed}
\usepackage{esdiff}
\usepackage{tikzsymbols}
\usepackage{multicol}
\usepackage{tikz}
\usepackage{varwidth}
\usepackage{parskip}
\usepackage{pgfplots}
\pgfplotsset{compat=1.18}
\intervalconfig {
	soft open fences
}

\newcommand{\alignedintertext}[1]{%
  \noalign{%
    \vtop{\hsize=\linewidth#1\par
    \expandafter}%
    \expandafter\prevdepth\the\prevdepth
  }%
}

\newtheorem{lemma}{Lemma}

\renewcommand{\qedsymbol}{\Smiley[1.3]}
\newcommand*{\problem}[1]{\section*{Problem #1}}
\newcommand*{\aps}{\section*{AP Corner}}
\newcommand*{\deriv}[1][x]{\ensuremath{\dfrac{\mathrm{d}}{\mathrm{d}#1}}}
\newcommand*{\floor}[1]{\ensuremath{\lfloor #1\rfloor}}
\newcommand*{\lheqzero}{\ensuremath{\underset{\text{L'H}}{\overset{\left[\frac00\right]}{=}}}}
\newcommand*{\lheqinfty}{\ensuremath{\underset{\text{L'H}}{\overset{\left[\frac{\infty}{\infty}\right]}{=}}}}

\DeclareMathOperator{\DNE}{DNE}
\DeclareMathOperator{\sgn}{sgn}

\DeclareMathOperator{\arccsc}{arccsc}
\DeclareMathOperator{\arcsec}{arcsec}
\DeclareMathOperator{\arccot}{arccot}

%opening

\title{\vspace*{-40pt}Problem Set \#58, Problem 9}
\author{Jayden Li}
\date{\today}
% \allowdisplaybreaks
\postdisplaypenalty=100000

\begin{document}
\setstretch{1.25}
\fontsize{12pt}{12pt}\selectfont
\setlength{\abovedisplayskip}{\abovedisplayskip/2}
\setlength{\belowdisplayskip}{\belowdisplayskip/2}
\setlength{\parindent}{0pt}
\setlength{\parskip}{2ex plus 0.5ex minus 0.2ex}
\maketitle

\problem{9}
\begin{align*}
	&\left\{\begin{aligned}
		x(\theta)&=R\cos \theta+\theta R\sin \theta \\
		y(\theta)&=R\sin \theta-\theta R\cos \theta
	\end{aligned}\right.
	\implies
	x'(\theta)
	=-R\sin \theta+R\sin \theta+\theta R\cos \theta
	=\theta R\sin \theta
\end{align*}

The left half of the involute is a semicircle containing the point:
\begin{equation*}
	(x(\pi),y(\pi))
	=\left(R\cos \pi+\pi R\sin \pi, R\sin \pi-\pi R\cos \pi\right)
	=\left(-R, -R\pi\right)
\end{equation*}

The semicircle is centered at $(-R, 0)$. Thus the radius of the semicircle is $R\pi$ and its area $A_1=\pi \left(R\pi\right)^2/2=R^2\pi^3/2$.

The other areas are $A_2=A_3$, where:

$\begin{aligned}[t]
    A_2
	&=\int_{0}^{\pi}\left(R\sin \theta-\theta R\cos\theta\right)\theta R\cos \theta\,\mathrm{d}\theta
	=\int_{0}^{\pi}\theta R^2\sin \theta\cos \theta\,\mathrm{d}\theta-\int_{0}^{\pi}\theta^2 R^2\cos^2\theta\,\mathrm{d}\theta \\
	&=R^2\left(\int_{0}^{\pi}\theta \sin \theta\cos \theta\,\mathrm{d}\theta-\int_{0}^{\pi}\theta^2 \cos^2\theta\,\mathrm{d}\theta\right) \\
	\alignedintertext{
	\begin{mdframed}
		$\begin{aligned}[t]
			I
			&=\int \theta\sin \theta\cos \theta\,\mathrm{d}\theta
			=\left[\begin{aligned}
					u&=\theta\sin\theta & \mathrm{d}v&=\cos \theta \,\mathrm{d}\theta \\
				\mathrm{d}u&=\left(\sin\theta+\theta\cos \theta\right)\mathrm{d}\theta & v&=\sin \theta
			\end{aligned}\right] \\
			&=\theta \sin^2\theta-\int \sin(\theta)\left(\sin \theta+\theta\cos \theta\right)\,\mathrm{d}\theta
			=\theta\sin^2\theta-\int \sin^2\theta\,\mathrm{d}\theta-\int \theta\sin\theta\cos\theta\,\mathrm{d}\theta \\
			&=\theta\sin^2\theta-\int \frac{1-\cos2\theta}{2}\,\mathrm{d}\theta-I
			=\theta\sin^2\theta-\frac12 \left(\theta-\frac{\sin2\theta}{2}\right)-I \\
			&=\theta\sin^2\theta-\frac{\theta}{2}+\frac{\sin2\theta}{4}-I
			\implies I=\frac{\theta\sin^2\theta}{2}-\frac{\theta}{4}+\frac{\sin2\theta}{8}
		\end{aligned}$
	\end{mdframed}
	\begin{mdframed}
		$\begin{aligned}[t]
		    \int \theta^2\cos^2\theta\,\mathrm{d}\theta
			&=\int \theta^2 \frac{1+\cos 2\theta}{2}\,\mathrm{d}\theta
			=\frac12 \int \left(\theta^2+\theta^2\cos 2\theta\right)\mathrm{d}\theta
			=\frac{1}{2} \left(\frac{\theta^3}{3}+\int \theta^2\cos 2\theta\,\mathrm{d}\theta\right) \\
			&=\left[\begin{aligned}
				u&=\theta^2 & \mathrm{d}v&=\cos 2\theta \,\mathrm{d}\theta \\
				\mathrm{d}u&=2\theta \,\mathrm{d}\theta & v&=\frac12\sin 2\theta
			\end{aligned}\right]\frac{\theta^3}{6}+\frac{1}{2}\left(\frac{1}{2}\theta^2\sin 2\theta-\int \theta\sin 2\theta\,\mathrm{d}\theta\right) \\
			&=\left[\begin{aligned}
				u&=\theta & \mathrm{d}v&=\sin 2\theta \,\mathrm{d}\theta \\
				\mathrm{d}u&=\mathrm{d}\theta & v&=-\frac12 \cos 2\theta
			\end{aligned}\right]\frac{\theta^3}{6}+\frac{1}{2}\left(\frac{1}{2}\theta^2\sin 2\theta-\left(-\frac12\theta\cos2\theta+\int\frac12 \cos 2\theta\mathrm{d}\theta\right)\right) \\
			&=\frac{\theta^3}{6}+\frac{1}{2}\left(\frac{1}{2}\theta^2\sin 2\theta-\left(-\frac12\theta\cos2\theta+\frac14 \sin 2\theta\right)\right)+C \\
			&=\frac{\theta^3}{6}+\frac12 \left(\frac12 \theta^2\sin 2\theta+\frac12 \theta\cos 2\theta-\frac14 \sin 2\theta\right)+C \\
			&=\frac{\theta^3}{6}+\frac{\theta^2\sin 2\theta}{4}+\frac{\theta\cos 2\theta}{4}-\frac{\sin 2\theta}{8}+C
		\end{aligned}$
	\end{mdframed}
	}
	&=R^2 \left(\left[\frac{\theta\sin ^2\theta}{2}-\frac{\theta}{4}+\frac{\sin 2\theta}{8}\right]_{0}^{\pi}-\left[\frac{\theta^3}{6}+\frac{\theta^2\sin 2\theta}{4}+\frac{\theta\cos 2\theta}{4}-\frac{\sin 2\theta}{8}\right]_{0}^{\pi}\right) \\
	&=R^2 \left(0-\frac{\pi}{4}\right)-R^2 \left(\frac{\pi^3}{6}+\frac{\pi}{4}\right)
	=-\frac{\pi R^2}{4}-\frac{R^2\pi^3}{6}-\frac{\pi R^2}{4}
	=-\frac{R^2\pi^3}{6}-\frac{\pi R^2}{2}
\end{aligned}$

Then we need to subtract the area of the fence, which is a circle of radius $R$. Its area is $A_4=\pi R^2$.

The total area is:

$\begin{aligned}[t]
	A
	&=\left|A_1\right|+\left|A_2\right|+\left|A_3\right|-\left|A_4\right|
	=\frac{R^2\pi^3}{2}+2 \left|-\frac{R^2\pi^3}{6}-\frac{\pi R^2}{2}\right|-\pi R^2
	=\frac{R^2\pi^3}{2}+\frac{R^2\pi^3}{3}+\pi R^2-\pi R^2 \\
	&=\frac{3R^2\pi^3}{6}+\frac{2R^2\pi^3}{6}
	=\boxed{\frac{5R^2\pi^3}{6}}
\end{aligned}$

\end{document}
