% JUMP TO LINE 60, 73
\documentclass[preview, margin=0.6in]{standalone}
\usepackage[letterpaper,portrait,top=0.4in, left=0.6in, right=0.6in, bottom=1in]{geometry}

\usepackage{amsmath, amsfonts, amsthm, amssymb}
\usepackage{graphicx, float}
\usepackage{mathtools}
\usepackage{titlesec}
\usepackage{interval}
\usepackage{hyperref}
\usepackage{siunitx}
\usepackage{titling}
\usepackage{vwcol}
\usepackage{setspace}
\usepackage{empheq}
\usepackage{gensymb}
\usepackage{textcomp}
\usepackage{cancel}
\usepackage{esdiff}
\usepackage{multicol}
\usepackage{mdframed}
\usepackage{esdiff}
\usepackage{tikzsymbols}
\usepackage{multicol}
\usepackage{tikz}
\usepackage{varwidth}
\usepackage{pgfplots}
\pgfplotsset{compat=1.18}
\intervalconfig {
	soft open fences
}

\newcommand{\alignedintertext}[1]{%
  \noalign{%
    \vskip\belowdisplayshortskip
    \vtop{\hsize=\linewidth#1\par
    \expandafter}%
    \expandafter\prevdepth\the\prevdepth
  }%
}

\newtheorem{lemma}{Lemma}

\renewcommand{\qedsymbol}{\Smiley[1.3]}
\newcommand*{\problem}[1]{\section*{Problem #1}}
\newcommand*{\aps}{\section*{AP Corner}}
\newcommand*{\deriv}[1][x]{\ensuremath{\dfrac{\mathrm{d}}{\mathrm{d}#1}}}
\newcommand*{\floor}[1]{\ensuremath{\lfloor #1\rfloor}}
\newcommand*{\lheqzero}{\ensuremath{\underset{\text{L'H}}{\overset{\left[\frac00\right]}{=}}}}
\newcommand*{\lheqinfty}{\ensuremath{\underset{\text{L'H}}{\overset{\left[\frac{\infty}{\infty}\right]}{=}}}}

\DeclareMathOperator{\DNE}{DNE}
\DeclareMathOperator{\sgn}{sgn}

\DeclareMathOperator{\arccsc}{arccsc}
\DeclareMathOperator{\arcsec}{arcsec}
\DeclareMathOperator{\arccot}{arccot}

\setlength{\parindent}{0pt}

%opening
\title{\vspace*{-30pt}Problem Set \#12}
\author{Jayden Li}
\date{\today}

% \allowdisplaybreaks
\postdisplaypenalty=100000

\begin{document}
\setstretch{1.25}
\fontsize{12pt}{12pt}\selectfont
\setlength{\abovedisplayskip}{0pt}
\maketitle
\problem{3}
Let $a$ be the length of the edges of the cube. Then $a=30,\mathrm da=0.1$.
\begin{itemize}
	\item[(a)]
	\begin{align*}
	    V&=a^3 \\
		\text{Maximum error}&=\mathrm dV
		=3a^2\cdot\mathrm da
		=3(30)^2\cdot0.1
		=\boxed{270\text{ cm}^3} \\
		\text{Relative error}&=\frac{270}{V}=\frac{270}{27000}=\boxed{0.01} \\
		\text{Percentage error}&=0.01\cdot100=\boxed{1\%}
	\end{align*}
	
	\item[(b)]
	\begin{align*}
	    A&=6a^2 \\
		\text{Maximum error}&=\mathrm dA
		=12a\cdot\mathrm da
		=12(30)\cdot0.1
		=\boxed{36\text{ cm}^2} \\
		\text{Relative error}&=\frac{36}{A}=\frac{36}{5400}=\boxed{0.00667} \\
		\text{Percentage error}&=0.00667\cdot100=\boxed{0.667\%}
	\end{align*}
\end{itemize}

\problem{6}
\begin{mdframed}
	\textbf{Lemma.} Let $\tan_d(x)$ be tangent in terms of degrees. We calculate the derivative of $\tan_d$.
	\begin{equation*}
		\deriv[x]\tan_d(x)
		=\deriv[x]\tan\left(\frac{\pi x}{180}\right)
		=\frac{\pi}{180}\sec^2(x)
	\end{equation*}
\end{mdframed}

We calculate the linearization of tangent around $45\degree$.
\begin{align*}
	L_{\tan}(x)&=\tan_d'\left(45\degree\right)\left(x-45\degree\right)+\tan\left(45\degree\right)
	=\frac{\pi}{180}\sec^2\left(\frac{\pi}{4}\right)\left(x-45\degree\right)+1 \\
	&=\frac{\pi}{180}\left(\frac{1}{\frac{\sqrt{2}}{2}}\right)^{2}\left(x-45\degree\right)+1
	=\frac{\pi}{90}\left(x-45\degree\right)+1 \\
	L_{\tan}(44\degree)&=\frac{\pi}{90}\left(44\degree-45\degree\right)+1
	=1-\frac{\pi}{90}=\boxed{0.965}
\end{align*}

\problem{8}
\begin{align*}
	L_f(x)&=f'(1)(x-1)+f(1)
	=2(x-1)+5
	=2x+3
\end{align*}
\begin{itemize}
	\item[(a)]
	\begin{align*}
		f(0.9)&\approx L_f(0.9)=2(0.9)+3=\boxed{4.8} \\
		f(1.1)&\approx L_f(1.1)=2(1.1)+3=\boxed{5.2}
	\end{align*}

	\item[(b)] $f$ is concave down because $f'$ is decreasing. So estimates in part (a) are \boxed{\text{overestimates}}.
\end{itemize}


\end{document}
