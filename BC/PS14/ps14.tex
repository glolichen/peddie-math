% JUMP TO LINE 60, 73

\documentclass[preview, margin=0.6in]{standalone}
\usepackage[letterpaper,portrait,top=0.4in, left=0.6in, right=0.6in, bottom=1in]{geometry}

\usepackage{amsmath, amsfonts, amsthm, amssymb}
\usepackage{graphicx, float}
\usepackage{mathtools}
\usepackage{titlesec}
\usepackage{interval}
\usepackage{hyperref}
\usepackage{siunitx}
\usepackage{titling}
\usepackage{vwcol}
\usepackage{setspace}
\usepackage{empheq}
\usepackage{cancel}
\usepackage{esdiff}
\usepackage{multicol}
\usepackage{mdframed}
\usepackage{esdiff}
\usepackage{tikzsymbols}
\usepackage{multicol}
\usepackage{tikz}
\usepackage{varwidth}
\usepackage{pgfplots}
\pgfplotsset{compat=1.18}
\intervalconfig {
	soft open fences
}

\newcommand{\alignedintertext}[1]{%
  \noalign{%
    \vskip\belowdisplayshortskip
    \vtop{\hsize=\linewidth#1\par
    \expandafter}%
    \expandafter\prevdepth\the\prevdepth
  }%
}

\newtheorem{lemma}{Lemma}

\renewcommand{\qedsymbol}{\Smiley[1.3]}
\newcommand*{\problem}[1]{\section*{Problem #1}}
\newcommand*{\aps}{\section*{AP Corner}}
\newcommand*{\deriv}[1][x]{\ensuremath{\dfrac{\mathrm{d}}{\mathrm{d}#1}}}
\newcommand*{\floor}[1]{\ensuremath{\lfloor #1\rfloor}}
\newcommand*{\lheqzero}{\ensuremath{\underset{\text{L'H}}{\overset{\left[\frac00\right]}{=}}}}
\newcommand*{\lheqinfty}{\ensuremath{\underset{\text{L'H}}{\overset{\left[\frac{\infty}{\infty}\right]}{=}}}}

\DeclareMathOperator{\DNE}{DNE}
\DeclareMathOperator{\sgn}{sgn}

\DeclareMathOperator{\arccsc}{arccsc}
\DeclareMathOperator{\arcsec}{arcsec}
\DeclareMathOperator{\arccot}{arccot}

\setlength{\parindent}{0pt}

%opening
\title{\vspace*{-30pt}Problem Set \#14}
\author{Jayden Li}
\date{\today}

% \allowdisplaybreaks
\postdisplaypenalty=100000

\begin{document}
\setstretch{1.25}
\fontsize{12pt}{12pt}\selectfont
% \setlength{\abovedisplayskip}{0pt}
% \setlength{\abovedisplayskip}{5pt}
% \setlength{\belowdisplayskip}{5pt}
% \setlength{\abovedisplayshortskip}{-80pt}
% \setlength{\belowdisplayshortskip}{100pt}
\maketitle
\problem{1}
\begin{itemize}
	\item[(e)]
		$\begin{aligned}[t]
		    \int p^5\ln p\,\mathrm{d}p
			&=\left[\begin{alignedat}{2}
				u&=\ln p &\quad \mathrm{d}u&=\frac{\mathrm{d}p}{p} \\
				\mathrm{d}v&=p^5 \,\mathrm{d}p & v&=\frac{p^6}{6}
			\end{alignedat}\right]
			\frac{p^6\ln p}{6}-\int \frac{p^6}{6p}\,\mathrm{d}p
			=\boxed{\frac{p^6\ln p}{6}-\frac{p^6}{36}+C}
		\end{aligned}$

	\item[(f)]
		$\begin{aligned}[t]
			\int_{0}^{1}t\cosh t\,\mathrm{d}t
			&=\left[\begin{alignedat}{2}
				u&=t &\quad \mathrm{d}u&=\mathrm{d}t \\
				\mathrm{d}v&=\cosh t & v&=\sinh t
			\end{alignedat}\right]
			\left[t\sinh t-\int \sinh t\,\mathrm{d}t\right]_{0}^{1}
			=\left[t\sinh t-\cosh t\right]_{0}^{1} \\
			&=1\sinh1-\cosh1-0+\cosh0
			=\frac{e^1-e^{-1}}{2}-\frac{e^1+e^{-1}}{2}+\frac{e^0+e^{-0}}{2} \\
			&=\frac{e^1-e^{-1}-e^1-e^{-1}}{2}+1
			=\boxed{-\frac{1}{e}+1}
		\end{aligned}$

	\item[(g)]
		$\begin{aligned}[t]
		    \int_{1}^{2}\frac{\ln x}{x^2}\,\mathrm{d}x
			&=\left[\begin{alignedat}{2}
				u&=\ln x &\quad \mathrm{d}u&=\frac{\mathrm{d}x}{x} \\
				\mathrm{d}v&=\frac{\mathrm{d}x}{x^2} & v&=-\frac{1}{x}
			\end{alignedat}\right]
			\left[-\frac{\ln x}{x}\right]_{1}^{2}+\int_{1}^{2}\frac{1}{x^2}\,\mathrm{d}x
			=-\frac{\ln2}{2}+0+\left[-\frac{1}{x}\right]_{1}^{2} \\
			&=-\frac{\ln2}{2}+\left(-\frac12+1\right)
			=\frac{-\ln2}{2}+\frac12=\boxed{\frac{1-\ln2}{2}}
		\end{aligned}$
\end{itemize}

\problem{2}
\boxed{\text{Average}=\displaystyle \frac{81\ln3-26}{18}} (obvious.)

\problem{8}
\begin{itemize}
	\item[(a)]
		$\begin{aligned}[t]
		    \int \sec x\,\mathrm{d}x
			&=\int \frac{1}{\cos x}\,\mathrm{d}x
			=\int \frac{\cos x}{1-\sin^2x}\,\mathrm{d}x
			=\left[ \begin{aligned}
				u&=\sin x \\				
				\mathrm{d}u&=\cos(x)\,\mathrm{d}x
			\end{aligned} \right]
			\int \frac{1}{1-u^2}\,\mathrm{d}u
			=\int \frac{1}{(1+u)(1-u)}\,\mathrm{d}u \\
			\alignedintertext{We can use partial fraction decomposition here. We need to find $A,B$ such that	\begin{align*}
				\frac{A}{1+u}+\frac{B}{1-u}&=\frac{1}{(1+u)(1-u)} \\
				\frac{A-Au+B+Bu}{(1+u)(1-u)}&=\frac{1}{(1+u)(1-u)} \\
				\frac{A+B-(A-B)u}{(1+u)(1-u)}&=\frac{1}{(1+u)(1-u)}
			\end{align*}
			We know that $(A-B)u=0u\implies A=B$. But also $A+B=1$, so $A,B=1/2$.}
			&=\int\left(\frac{1/2}{1+u}+\frac{1/2}{1-u}\right)\mathrm{d}u
			=\frac12 \left(\int \frac{1}{1+u}\,\mathrm{d}u+\int \frac{1}{1-u}\,\mathrm{d}u\right) \\
			&=\frac12 \left(\ln|1+u|-\ln|1-u|\right)
			=\boxed{\frac12 \ln \left|\frac{1+\sin x}{1-\sin x}\right|+C}
		\end{aligned}$
		(I did look up some hints for this one, such as substituting $u=\sin x$ and using partial fractions.)
		
	\item[(b)]
		$\begin{aligned}[t]
		    \int \sec^3x\,\mathrm{d}x
			&=\int \frac{1}{\cos^3x}\,\mathrm{d}x
			=\int \frac{\cos x}{\cos^4x}\,\mathrm{d}x
			=\int \frac{\cos x}{\left(1-\sin^2x\right)^2}\,\mathrm{d}x
			=\left[ \begin{aligned}
				u&=\sin x \\				
				\mathrm{d}u&=\cos(x)\,\mathrm{d}x
			\end{aligned} \right]
			\int \frac{1}{(1-u^2)^2}\,\mathrm{d}u \\
			&=\int \left(\frac{1}{(1-u)(1+u)}\right)^2\,\mathrm{d}u
			=\int\left(\frac12\left(\frac{1}{1+u}+\frac{1}{1-u}\right)\right)^2\mathrm{d}u \\
			&=\frac14 \int \frac{1}{(1+u)^2}\,\mathrm{d}u+\frac12 \int \frac{1}{(1+u)(1-u)}\,\mathrm{d}u+\frac14 \int \frac{1}{(1-u)^2}\,\mathrm{d}u \\
			&=-\frac{1}{4(1+u)^2}+\frac{1}{4(1-u)^2}+\frac14 \int\left(\frac{1}{1+u}+\frac{1}{1-u}\right)\mathrm{d}u \\
			&=-\frac{1}{4(1+u)^2}+\frac{1}{4(1-u)^2}+\frac14(\ln|1+u|-\ln|1-u|) \\
			&=\boxed{-\frac{1}{4(1+\sin x)}+\frac{1}{4(1-\sin x)}+\frac14\ln \left|\frac{1+\sin x}{1-\sin x}\right|+C}
		\end{aligned}$

\end{itemize}

\end{document}
