% JUMP TO LINE 60, 75
\documentclass[preview, margin=0.6in]{standalone}
\usepackage[letterpaper,portrait,top=0.4in, left=0.6in, right=0.6in, bottom=1in]{geometry}

\usepackage{amsmath, amsfonts, amsthm, amssymb}
\usepackage{graphicx, float}
\usepackage{mathtools}
\usepackage{titlesec}
\usepackage{interval}
\usepackage{hyperref}
\usepackage{siunitx}
\usepackage{titling}
\usepackage{vwcol}
\usepackage{setspace}
\usepackage{empheq}
\usepackage{cancel}
\usepackage{esdiff}
\usepackage{multicol}
\usepackage{mdframed}
\usepackage{esdiff}
\usepackage{tikzsymbols}
\usepackage{multicol}
\usepackage{tikz}
\usepackage{varwidth}
\usepackage{parskip}
\usepackage{pgfplots}
\pgfplotsset{compat=1.18}
\intervalconfig {
	soft open fences
}

\newcommand{\alignedintertext}[1]{%
  \noalign{%
    \vskip\belowdisplayshortskip
    \vtop{\hsize=\linewidth#1\par
    \expandafter}%
    \expandafter\prevdepth\the\prevdepth
  }%
}

\newtheorem{lemma}{Lemma}

\newcommand*{\problem}[1]{\section*{Problem #1}}
\newcommand*{\aps}{\section*{AP Corner}}
\newcommand*{\deriv}[1][x]{\ensuremath{\dfrac{\mathrm{d}}{\mathrm{d}#1}}}
\newcommand*{\floor}[1]{\ensuremath{\lfloor #1\rfloor}}
\newcommand*{\lheqzero}{\ensuremath{\underset{\text{L'H}}{\overset{\left[\frac00\right]}{=}}}}
\newcommand*{\lheqinfty}{\ensuremath{\underset{\text{L'H}}{\overset{\left[\frac{\infty}{\infty}\right]}{=}}}}

\DeclareMathOperator{\DNE}{DNE}
\DeclareMathOperator{\sgn}{sgn}

\DeclareMathOperator{\arccsc}{arccsc}
\DeclareMathOperator{\arcsec}{arcsec}
\DeclareMathOperator{\arccot}{arccot}

%opening

\title{\vspace*{-40pt} Harmonic Series is Divergent}
\author{Jayden Li}
\date{\today}

% \allowdisplaybreaks
\postdisplaypenalty=100000

\begin{document}
\setstretch{1.25}
\fontsize{12pt}{12pt}\selectfont
\setlength{\abovedisplayskip}{\abovedisplayskip/2}
\setlength{\belowdisplayskip}{\belowdisplayskip/2}
\setlength{\parindent}{0pt}
\setlength{\parskip}{2ex plus 0.5ex minus 0.2ex}
\maketitle

\begin{proof}
Let $\displaystyle a_n=\frac 1n$ and $\displaystyle s_n=\sum_{k=1}^{n}a_n=\sum_{k=1}^{n}\frac1k$.

Define the sequence $\{b_n\}$ where the $n$th term is the reciprocal of the smallest power of two that is larger than $n$. For all integer $m\geq1$, the $k$th term $b_k=1/2^m$ if and only if $2^{m-1}+1\leq k\leq 2^m$. There are exactly $2^m-2^{m-1}-1+1=2^m-2^{m-1}=2^{m-1}(2-1)=2^m/2$ values of $k$ satisfying this inequality.

Note that the reciprocal of every term $b_k$ is larger than or equal to $k$.
\begin{equation*}
	(b_k)^{-1}\geq k
	\implies b_k\leq \frac{1}{k}=a_k
\end{equation*}

Let $\{u_n\}$ be the $n$th partial sum of $\{b_n\}$. Then for all $k$, $u_k\leq s_k$ because every term of the sum $u_k$ is less than equal to the corresponding term in $s_k$.

Let us calculate the value of $u_k$.
\begin{equation*}
	u_n=1+\sum_{k=1}^{\log_2n} \frac{1}{2^k}\cdot \frac{2^k}{2}
	=1+\sum_{k=1}^{\log_2n}\frac12
	=1+\frac{\log_2n}{2}
\end{equation*}
(this diverges.)

\begin{mdframed}
	Formal definition for the divergence of a series is that it grows arbitrarily:
	\begin{equation*}
		\lim_{n\to\infty}s_n=\infty
		\iff
		(\forall M>0) (\exists c>0) : \mathbb N\ni n>c \implies s_n>M \tag{1}
	\end{equation*}
\end{mdframed}

Because we know $u_n\geq s_n$:
\begin{align*}
	\lim_{n\to\infty}s_n=\infty
	&\iff
	(\forall M>0) (\exists c>0) : \mathbb N\ni n>c \implies s_n\geq u_n>M
	\intertext{If $u_n>M$, it is necessarily true that $s_n>M$. Therefore, the following implies statement (1):}
	&\impliedby (\forall M>0) (\exists c>0) : \mathbb N\ni n>c \implies u_n>M \\
	&\iff (\forall M>0) (\exists c>0) : \mathbb N\ni n>c \implies 1+\frac{\log_2n}{2}>M \\
	&\iff (\forall M>0) (\exists c>0) : \mathbb N\ni n>c \implies 2+\log_2n>2M \\
	&\iff (\forall M>0) (\exists c>0) : \mathbb N\ni n>c \implies \log_2n>2M-2 \\
	&\iff (\forall M>0) (\exists c>0) : \mathbb N\ni n>c \implies n>2^{2M-2}
\end{align*}

Let $c=2^{2M-2}$. Then, for all integer $M>0$, $c$ must also be an integer. If some integer $n>c$:
\begin{align*}
	n>c 
	&\implies n>2^{2M-2} \\
	&\implies \log_2n>2M-2 & \text{(logarithm of all bases is increasing)} \\
	&\implies 2+\log_2n>2M \\
	&\implies 1+\frac{\log_2n}{2}>M
\end{align*}
Therefore, by definition (1), the sequence $\left\{1+\dfrac{\log_2n}{2}\right\}=\{u_n\}$ diverges. The divergence of $\{u_n\}$ implies the divergence of $\{s_n\}$, so the harmonic series diverges.
\end{proof}


\end{document}
