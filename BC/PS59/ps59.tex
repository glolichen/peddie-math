% JUMP TO LINE 60, 75
\documentclass[preview, margin=0.6in]{standalone}
\usepackage[letterpaper,portrait,top=0.4in, left=0.6in, right=0.6in, bottom=1in]{geometry}

\usepackage{amsmath, amsfonts, amsthm, amssymb}
\usepackage{graphicx, float}
\usepackage{mathtools}
\usepackage{titlesec}
\usepackage{interval}
\usepackage{hyperref}
\usepackage{siunitx}
\usepackage{titling}
\usepackage{vwcol}
\usepackage{setspace}
\usepackage{empheq}
\usepackage{cancel}
\usepackage{esdiff}
\usepackage{multicol}
\usepackage{mdframed}
\usepackage{esdiff}
\usepackage{tikzsymbols}
\usepackage{multicol}
\usepackage{tikz}
\usepackage{varwidth}
\usepackage{parskip}
\usepackage{pgfplots}
\pgfplotsset{compat=1.18}
\intervalconfig {
	soft open fences
}

\newcommand{\alignedintertext}[1]{%
  \noalign{%
    \vtop{\hsize=\linewidth#1\par
    \expandafter}%
    \expandafter\prevdepth\the\prevdepth
  }%
}

\newtheorem{lemma}{Lemma}

\renewcommand{\qedsymbol}{\Smiley[1.3]}
\newcommand*{\problem}[1]{\section*{Problem #1}}
\newcommand*{\aps}{\section*{AP Corner}}
\newcommand*{\deriv}[1][x]{\ensuremath{\dfrac{\mathrm{d}}{\mathrm{d}#1}}}
\newcommand*{\floor}[1]{\ensuremath{\lfloor #1\rfloor}}
\newcommand*{\lheqzero}{\ensuremath{\underset{\text{L'H}}{\overset{\left[\frac00\right]}{=}}}}
\newcommand*{\lheqinfty}{\ensuremath{\underset{\text{L'H}}{\overset{\left[\frac{\infty}{\infty}\right]}{=}}}}

\DeclareMathOperator{\DNE}{DNE}
\DeclareMathOperator{\sgn}{sgn}

\DeclareMathOperator{\arccsc}{arccsc}
\DeclareMathOperator{\arcsec}{arcsec}
\DeclareMathOperator{\arccot}{arccot}

%opening

\title{\vspace*{-40pt}Problem Set \#59}
\author{Jayden Li}
\date{\today}
% \allowdisplaybreaks
\postdisplaypenalty=100000

\begin{document}
\setstretch{1.25}
\fontsize{12pt}{12pt}\selectfont
\setlength{\abovedisplayskip}{\abovedisplayskip/2}
\setlength{\belowdisplayskip}{\belowdisplayskip/2}
\setlength{\parindent}{0pt}
\setlength{\parskip}{2ex plus 0.5ex minus 0.2ex}
\maketitle

\problem{6}
\begin{itemize}
	\item[(a)]
		$\begin{aligned}[t]
		    r=2
			&\implies \sqrt{x^2+y^2}=2
			\implies x^2+y^2=4
		\end{aligned}$

		Circle
	
	\item[(b)]
		$\begin{aligned}[t]
			r=3\sin \theta
			&\implies r^2=3r\sin\theta
			\implies x^2+y^2=3y
		\end{aligned}$

		Circle
	
	\item[(b)]
		$\begin{aligned}[t]
		    r=\csc\theta
			&\implies r=\frac{1}{\sin\theta}
			\implies r\sin\theta=1
			\implies y=1
		\end{aligned}$

		Horizontal line
\end{itemize}

\problem{7}
\begin{itemize}
	\item[(a)]
		$\begin{aligned}[t]
		    x=3
			&\implies r\cos\theta=3
			\implies \boxed{r=3\sec\theta}
		\end{aligned}$
	
	\item[(b)]
		$\begin{aligned}[t]
		    x=-y^2
			&\implies r\cos \theta=- \left(r\sin\theta\right)^2
			\implies r\cos\theta=-r^2\sin^2\theta
			\implies -\cos\theta=r\sin^2\theta \\
			&\implies \boxed{r=-\cos\theta \csc^2\theta}
		\end{aligned}$
	
	\item[(b)]
		$\begin{aligned}[t]
			x^2+y^2=2cx
			&\implies r^2=2cr\cos \theta
			\implies \boxed{r=2c\cos\theta}
		\end{aligned}$
\end{itemize}

\problem{10}
\begin{itemize}
	\item[(a)] V

	\item[(b)] II

	\item[(c)] VI

	\item[(d)] III

	\item[(e)] I

	\item[(f)] IV

\end{itemize}

\end{document}
