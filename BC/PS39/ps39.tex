% JUMP TO LINE 60, 75
\documentclass[preview, margin=0.6in]{standalone}
\usepackage[letterpaper,portrait,top=0.4in, left=0.6in, right=0.6in, bottom=1in]{geometry}

\usepackage{amsmath, amsfonts, amsthm, amssymb}
\usepackage{graphicx, float}
\usepackage{mathtools}
\usepackage{titlesec}
\usepackage{interval}
\usepackage{hyperref}
\usepackage{siunitx}
\usepackage{titling}
\usepackage{vwcol}
\usepackage{setspace}
\usepackage{empheq}
\usepackage{cancel}
\usepackage{esdiff}
\usepackage{multicol}
\usepackage{mdframed}
\usepackage{esdiff}
\usepackage{tikzsymbols}
\usepackage{multicol}
\usepackage{tikz}
\usepackage{varwidth}
\usepackage{parskip}
\usepackage{pgfplots}
\pgfplotsset{compat=1.18}
\intervalconfig {
	soft open fences
}

\newcommand{\alignedintertext}[1]{%
  \noalign{%
    \vskip\belowdisplayshortskip
    \vtop{\hsize=\linewidth#1\par
    \expandafter}%
    \expandafter\prevdepth\the\prevdepth
  }%
}

\newtheorem{lemma}{Lemma}

\newcommand*{\problem}[1]{\section*{Problem #1}}
\newcommand*{\aps}{\section*{AP Corner}}
\newcommand*{\deriv}[1][x]{\ensuremath{\dfrac{\mathrm{d}}{\mathrm{d}#1}}}
\newcommand*{\floor}[1]{\ensuremath{\lfloor #1\rfloor}}
\newcommand*{\lheqzero}{\ensuremath{\underset{\text{L'H}}{\overset{\left[\frac00\right]}{=}}}}
\newcommand*{\lheqinfty}{\ensuremath{\underset{\text{L'H}}{\overset{\left[\frac{\infty}{\infty}\right]}{=}}}}

\DeclareMathOperator{\DNE}{DNE}
\DeclareMathOperator{\sgn}{sgn}

\DeclareMathOperator{\arccsc}{arccsc}
\DeclareMathOperator{\arcsec}{arcsec}
\DeclareMathOperator{\arccot}{arccot}

%opening

\title{\vspace*{-30pt}Problem Set \#39}
\author{Jayden Li}
\date{\today}

% \allowdisplaybreaks
\postdisplaypenalty=100000

\begin{document}
\setstretch{1.25}
\fontsize{12pt}{12pt}\selectfont
\setlength{\abovedisplayskip}{\abovedisplayskip / 2}
\setlength{\belowdisplayskip}{\belowdisplayskip / 2}
\setlength{\parindent}{0pt}
\setlength{\parskip}{2ex plus 0.5ex minus 0.2ex}
\maketitle

\problem{1}
For all $k\geq 1$:
\begin{align*}
	a_k>a_{k+1}
	&\iff \frac{k+1}{k}>\frac{k+2}{k+1} \\
	&\iff k^2+2k+1>k^2+2k
	& \text{(multiplying positive $k(k+1)$)} \\
	&\iff k^2+1>k^2
\end{align*}
Clearly, the above must be true. Thus, $\{a_n\}$ is an decreasing sequence. \qed

\problem{2}
\begin{itemize}
\item[(a)] Done in class
\item[(b)] Done in class (yes because sine function is between -1 and 1)

\item[(c)]
$\displaystyle a_n=\left\{\sqrt{2a_{n-1}}\right\},a_1=\sqrt{2}$.

\textbf{\textit{Proof increasing}}

\underline{Base case.} When $n=1$, $a_n=\sqrt{2}$ and $a_{n+1}=\sqrt{2 \sqrt{2}}$. Because square function is increasing for positive values, $a_1<a_2 \iff \sqrt{2} < \sqrt{2 \sqrt{2}} \iff 2 < 2 \sqrt{2}$. Clearly true because $ \sqrt{2}>1$.

\underline{Hypothesis.} Suppose $\displaystyle a_k<a_{k+1}\iff a_k< \sqrt{2a_k}$ when $k\geq1$.

\underline{Inductive Step.}
Want to show the statement holds for $k+1$, that is, $a_{k+1}<a_{k+2}$.
\begin{align*}
	a_{k+1}<a_{k+2}
	&\iff \sqrt{2a_k}< \sqrt{2 \sqrt{2 a_k}} \\
	&\iff 2a_k < 2 \sqrt{2a_k} & \text{(square function increasing for positive values)} \\
	&\iff a_k< \sqrt{2a_k}
\end{align*}
By the principle of mathematical induction, the statement $a_k<a_{k+1}$ is true for all $k\geq 1$. \qed

Because the sequence is increasing, it is clearly bounded at the lowest possible value, which is $a_1= \sqrt{2}$. So $\{a_n\}$ is bounded below.

\textbf{\textit{Proof bounded above}}

I claim that the sequence is bounded above at $2$. That is, $a_k<2$ for all $k\geq1$.

\underline{Base case.} When $n=1$, $a_n=\sqrt{2}<2$.

\underline{Hypothesis.} Suppose $\displaystyle a_k<2$ when $k\geq1$.

\underline{Inductive Step.}
Want to show the statement holds for $k+1$, that is, $a_{k+1}<2$.
\begin{align*}
	a_{k+1}<2
	&\iff \sqrt{2a_k}<2
	\iff 2a_k<4
	\iff a_k<2
\end{align*}
True from hypothesis.

By the principle of mathematical induction, the statement $a_k<\sqrt{2}$ is true for all $k\geq 1$. \qed

Thus, because $\{a_n\}$ is bounded above and below, it is bounded. \qed

\end{itemize}

\problem{3}
\begin{itemize}
	\item[(a)] Done in class
	\item[(b)] Alternating sequence, so not monotonic. Clearly not bounded as $a_{\infty}$ tends to infinity.
	\item[(c)] Let $\displaystyle f(x)=\frac{x}{x^2+1}$. Then $f(n)=a_n$ for all $n\in\mathbb N$.
	\begin{align*}
	    f'(x)<0
		&\iff \frac{\left(x^2+1\right)-x(2x)}{\left(x^2+1\right)^2}<0 \\
		&\iff x^2+1-2x^2<0 & \text{(denominator always positive)} \\
		&\iff x^2>1 \iff x\in (-\infty,-1)\cup (1,\infty)
	\end{align*}
	For all $x>1$, $f'(x)<0$, so $f$ is decreasing for all $x>1$. So $\{a_n\}$ is decreasing for all $n>1$. But $a_1=1/2>2/5=a_2$, so it is decreasing for all $n\geq1$. Therefore $\{a_n\}$ is decreasing and monotonic.

	Clearly it is also bounded because the range of $f$ on the positive real numbers is $(0,\infty)$ (too lazy).
\end{itemize}

\problem{4}
\begin{itemize}
	\item[(a)] Done in class
	\item[(b)] Same proof as 2c, but with 6 instead of 2. Or:

$\displaystyle a_n=\left\{\sqrt{6a_{n-1}}\right\},a_1=\sqrt{6}$.

\textbf{\textit{Proof increasing}}

\underline{Base case.} When $n=1$, $a_n=\sqrt{6}$ and $a_{n+1}=\sqrt{6 \sqrt{6}}$. Because square function is increasing for positive values, $a_1<a_6 \iff \sqrt{6} < \sqrt{6 \sqrt{6}} \iff 6 < 6 \sqrt{6}$. Clearly true because $ \sqrt{6}>1$.

\underline{Hypothesis.} Suppose $\displaystyle a_k<a_{k+1}\iff a_k< \sqrt{6a_k}$ when $k\geq1$.

\underline{Inductive Step.}
Want to show the statement holds for $k+1$, that is, $a_{k+1}<a_{k+2}$.
\begin{align*}
	a_{k+1}<a_{k+2}
	&\iff \sqrt{6a_k}< \sqrt{6 \sqrt{6 a_k}} \\
	&\iff 6a_k < 6 \sqrt{6a_k} & \text{(square function increasing for positive values)} \\
	&\iff a_k< \sqrt{6a_k}
\end{align*}
By the principle of mathematical induction, the statement $a_k<a_{k+1}$ is true for all $k\geq 1$. \qed

Because the sequence is increasing, it is clearly bounded at the lowest possible value, which is $a_1= \sqrt{6}$. So $\{a_n\}$ is bounded below.

\textbf{\textit{Proof bounded above}}

I claim that the sequence is bounded above at $6$. That is, $a_k<6$ for all $k\geq1$.

\underline{Base case.} When $n=1$, $a_n=\sqrt{6}<6$.

\underline{Hypothesis.} Suppose $\displaystyle a_k<6$ when $k\geq1$.

\underline{Inductive Step.}
Want to show the statement holds for $k+1$, that is, $a_{k+1}<6$.
\begin{align*}
	a_{k+1}<6
	&\iff \sqrt{6a_k}<6
	\iff 6a_k<36
	\iff a_k<6
\end{align*}
True from hypothesis.

By the principle of mathematical induction, the statement $a_k<\sqrt{6}$ is true for all $k\geq 1$. \qed

Thus, because $\{a_n\}$ is bounded above and below, it is bounded. \qed

\end{itemize}

\problem{5}
Done in class

\problem{6}
\begin{itemize}
\item[(a)] At the start, there is one pair. That pair takes 2 months to produce another pair, so $f_1=f_2=1$. At any time $n\in\mathbb N,n>2$, no rabbits die, so every rabbit at the previous time of $n-1$ are also at time $n$. There are $f_{n-1}$ such rabbits. Rabbits born two months ago at time $n-2$ now become productive and produce new pairs; there are $f_{n-2}$ of these. So the number of rabbits at time $n$ is $f_n=f_{n-1}+f_{n-2}$.

\item[(b)]
\begin{align*}
    a_{n-1}=1+\frac{1}{a_{n-2}}
	&\iff \frac{f_n}{f_{n-1}}=1+\left(\frac{f_{n-1}}{f_{n-2}}\right)^{-1}
	\iff \frac{f_n}{f_{n-1}}=\frac{f_{n-1}}{f_{n-1}}+\frac{f_{n-2}}{f_{n-1}}
	\iff \frac{f_n}{f_{n-1}}=\frac{f_n}{f_{n-1}} \\
	&\iff a_n=1+\frac{1}{a_{n-1}}
\end{align*}

Let $\displaystyle L= \lim_{n\to\infty}a_n$, then $L=\lim_{n\to \infty}a_{n-1}$. From above, we have:
\begin{align*}
    a_n=1+\frac{1}{a_{n-1}}
	&\implies \lim_{n\to\infty}a_n=\lim_{n\to\infty}\left[1+\frac{1}{a_{n-1}}\right]
	\implies L=1+\frac 1L
	\implies L^2-L-1=0 \\
	&\implies \left(L-\frac12\right)^2-\frac{1}{4}-\frac44=0
	\implies L-\frac12=\pm \sqrt{\frac54}
	\implies \boxed{L=\frac{1\pm\sqrt{5}}{2}}
\end{align*}
The golden ratio $\varphi$ is defined as the positive case.

\end{itemize}

\end{document}
