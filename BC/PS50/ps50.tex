% JUMP TO LINE 60, 75
\documentclass[preview, margin=0.6in]{standalone}
% \documentclass{article}
\usepackage[letterpaper,portrait,top=0.4in, left=0.6in, right=0.6in, bottom=1in]{geometry}

\usepackage{amsmath, amsfonts, amsthm, amssymb}
\usepackage{graphicx, float}
\usepackage{mathtools}
\usepackage{titlesec}
\usepackage{interval}
\usepackage{hyperref}
\usepackage{siunitx}
\usepackage{titling}
\usepackage{vwcol}
\usepackage{setspace}
\usepackage{empheq}
\usepackage{cancel}
\usepackage{esdiff}
\usepackage{multicol}
\usepackage{mdframed}
\usepackage{esdiff}
\usepackage{tikzsymbols}
\usepackage{multicol}
\usepackage{tikz}
\usepackage{varwidth}
\usepackage{parskip}
\usepackage{pgfplots}
\pgfplotsset{compat=1.18}
\intervalconfig {
	soft open fences
}

\newcommand{\alignedintertext}[1]{%
  \noalign{%
    \vskip\belowdisplayshortskip
    \vtop{\hsize=\linewidth#1\par
    \expandafter}%
    \expandafter\prevdepth\the\prevdepth
  }%
}

\newtheorem{lemma}{Lemma}

\renewcommand{\qedsymbol}{\Smiley[1.3]}
\newcommand*{\problem}[1]{\section*{Problem #1}}
\newcommand*{\aps}{\section*{AP Corner}}
\newcommand*{\deriv}[1][x]{\ensuremath{\dfrac{\mathrm{d}}{\mathrm{d}#1}}}
\newcommand*{\floor}[1]{\ensuremath{\lfloor #1\rfloor}}
\newcommand*{\lheqzero}{\ensuremath{\underset{\text{L'H}}{\overset{\left[\frac00\right]}{=}}}}
\newcommand*{\lheqinfty}{\ensuremath{\underset{\text{L'H}}{\overset{\left[\frac{\infty}{\infty}\right]}{=}}}}

\DeclareMathOperator{\DNE}{DNE}
\DeclareMathOperator{\sgn}{sgn}

\DeclareMathOperator{\arccsc}{arccsc}
\DeclareMathOperator{\arcsec}{arcsec}
\DeclareMathOperator{\arccot}{arccot}

%opening

\title{\vspace*{-40pt}Problem Set \#50}
\author{Jayden Li}
\date{\today}
% \allowdisplaybreaks
\postdisplaypenalty=100000

\begin{document}
\setstretch{1.25}
\fontsize{12pt}{12pt}\selectfont
\setlength{\abovedisplayskip}{\abovedisplayskip/2}
\setlength{\belowdisplayskip}{\belowdisplayskip/2}
\setlength{\parindent}{0pt}
\setlength{\parskip}{2ex plus 0.5ex minus 0.2ex}
\maketitle

\problem{1}
\begin{itemize}
\item[(a)]
	Center of convergence is $x=0$.

	$\begin{aligned}[t]
		L<1
		&\iff 
		\begin{aligned}[t]
			\lim_{n\to\infty}\left|\frac{a_{n+1}}{a_n}\right|
			&=\lim_{n\to\infty}\left|\frac{|x|^{n+1}}{\sqrt{n+1}}\frac{\sqrt{n}}{|x|^n}\right|
			=\lim_{n\to\infty}\frac{|x| \sqrt n}{\sqrt{n+1}}
			=\lim_{n\to\infty}\frac{|x| \sqrt{n}}{\sqrt{n}\sqrt{1+\frac1n}} \\
			&=\lim_{n\to\infty}\frac{|x|}{\sqrt{1+\cancelto{0}{1/n}}}
			=|x|<1
		\end{aligned}
	\end{aligned}$

	$\begin{aligned}[t]
		x=1 &: \sum a_n=\sum \frac{1}{n^{1/2}}\text{ diverges by the $p$-series test: $p=1/2\leq1$.} \\
		x=-1&: \sum a_n=\sum \frac{(-1)^n}{n^{1/2}}\text{ converges by the AST since $\lim_{n\to\infty}a_n=0$ and $|a_n|$ is decreasing.}
	\end{aligned}$

	Radius of convergence: $1$, Interval of convergence: $[0+(-1), 0+1)=\boxed{[-1,1)}$.

\item[(b)]
	Center of convergence is $x=0$.

	$\begin{aligned}[t]
		L<1
		&\iff 
		\begin{aligned}[t]
			\lim_{n\to\infty}\left|\frac{a_{n+1}}{a_n}\right|
			&=\lim_{n\to\infty}\left|\frac{(-1)^{n+1}x^{n+1}}{(n+1)^3} \frac{n^3}{(-1)^n x^n}\right|
			=\lim_{n\to\infty}\left|\frac{(-1)xn^3}{(n+1)^3}\right| \\
			&=\lim_{n\to\infty}|x|\frac{n^3}{n^3+3n^2+3n+1}
			=\lim_{n\to\infty}|x|\frac{n^3}{n^3 \left(1+\frac{3}{n}+\frac{3}{n^2}+\frac{1}{n^3}\right)}
			=|x|<1
		\end{aligned}
	\end{aligned}$

	$\begin{aligned}[t]
		x=1 &: \sum a_n=\sum \frac{(-1)^n}{n^3}\text{ converges by the absolute convergence test.} \\
		x=-1&: \sum a_n=\sum \frac{(-1)^n(-1)^n}{n^3}=\sum \frac{1}{n^3} \text{ converges by the $p$-series test.}
	\end{aligned}$

	Radius of convergence: $1$, Interval of convergence: $\boxed{[-1,1]}$.

\item[(c)]
	Center of convergence is $x=0$.

	$\begin{aligned}[t]
		L<1
		&\iff 
		\begin{aligned}[t]
			\lim_{n\to\infty}\left|\frac{a_{n+1}}{a_n}\right|
			\lim_{n\to\infty}\left|\frac{x^{n+1}}{(n+1)!}\frac{n!}{x^n}\right|
			=\lim_{n\to\infty}\frac{|x|}{n+1}
			=0<1
		\end{aligned}
	\end{aligned}$

	Interval of convergence: $\boxed{(-\infty,\infty)}$.

\item[(d)]
	Center of convergence is $x=0$.

	$\begin{aligned}[t]
		L<1
		&\iff 
		\begin{aligned}[t]
			\lim_{n\to\infty}\left|\frac{a_{n+1}}{a_n}\right|
			&\lim_{n\to\infty}\left|\frac{(-1)^{n+1}(n+1)^2x^{n+1}}{2^{n+1}}\frac{2^n}{(-1)^nn^2x^n}\right|
			=\lim_{n\to\infty}\left|\frac{(-1)(n+1)^2x}{2n^2}\right| \\
			&=\lim_{n\to\infty}\frac{|x|(n+1)^2}{2n^2}
			=\frac{|x|}{2}<1
			\iff |x|<2
		\end{aligned}
	\end{aligned}$

	$\begin{aligned}[t]
		x=2 &: \sum a_n=\sum (-1)^n n^2 \text{ diverges by the $n$th term divergence test.} \\
		x=-1&: \sum a_n=\sum \frac{(-1)^n n^2 (-2)^n}{2^n}=\sum \frac{n^2 2^n}{2^n} \text{ diverges by the $n$th term divergence test.}
	\end{aligned}$

	Radius of convergence: $2$, Interval of convergence: $\boxed{(-2,2)}$.

\item[(e)]
	Center of convergence is $x=0$.

	$\begin{aligned}[t]
		L<1
		&\iff 
		\begin{aligned}[t]
			\lim_{n\to\infty}\left|\frac{a_{n+1}}{a_n}\right|
			&=\lim_{n\to\infty}\left|\frac{(-2)^{n+1}x^{n+1}}{\sqrt[4]{n+1}}\frac{\sqrt[4]{n}}{(-2)^nx^n}\right|
			=\lim_{n\to\infty}\left|\frac{(-2)x \sqrt[4]{n}}{\sqrt[4]{n+1}}\right| \\
			&=2 \lim_{n\to\infty}|x| \frac{n^{1/4}}{(n+1)^{1/4}}
			=2|x|<1
			\iff x<\frac12
		\end{aligned}
	\end{aligned}$

	$\displaystyle x=\frac12: \displaystyle \sum a_n=\sum \frac{(-2)^n \left(\frac12\right)^n}{\sqrt[4]{n}}=\sum \frac{(-1)^n}{\sqrt[4]{n}}$ converges by the alternating series test since $\lim_{n\to\infty}a_n=0$ and $|a_n$ is decreasing.

	$\displaystyle x=-\frac12: \displaystyle \sum a_n=\sum \frac{(-2)^n \left(-\frac12\right)^n}{\sqrt[4]{n}}=\sum \frac{1}{\sqrt[4]{n}}$ diverges by the $p$-series test: $p=1/4\leq1$.

	Radius of convergence: $1/2$, Interval of convergence: $\boxed{(-1/2,1/2]}$.

\item[(f)]
	Center of convergence: $x=0$.

	$\begin{aligned}[t]
		L<1
		&\iff 
		\begin{aligned}[t]
			\lim_{n\to\infty}\left|\frac{a_{n+1}}{a_n}\right|
			&=\lim_{n\to\infty}\left|\frac{(-1)^{n+1}x^{n+1}}{4^{n+1} \ln (n+1)}\frac{4^n \ln n}{(-1)^n x^n}\right|
			=\lim_{n\to\infty} \left|\frac{(-1)x\ln n}{4\ln(n+1)}\right| \\
			&=\frac14\lim_{n\to\infty}|x| \frac{\ln n}{\ln(n+1)}
			=\frac{|x|}{4}<1
			\iff |x|<4
		\end{aligned}
	\end{aligned}$

	$\displaystyle x=4: \sum \frac{(-1)^n 4^n}{4^n \ln n}=\sum \frac{(-1)^n}{\ln n}$ converges by the alternating series test since $\lim_{n\to\infty}a_n=0$ and $|a_n|$ is decreasing when $x=4$.

	$\displaystyle x=-4: \sum \frac{(-1)^n (-4)^n}{4^n \ln n}=\frac{1}{\ln n}$ diverges by comparison to $\dfrac 1n$.

	Radius of convergence: $4$, Interval of convergence: \boxed{(-4,4]}.

\item[(g)]
	Center of convergence: $x=2$.

	$\begin{aligned}[t]
		L<1
		&\iff 
		\begin{aligned}[t]
			\lim_{n\to\infty}\left|\frac{a_{n+1}}{a_n}\right|
			&=\lim_{n\to\infty}\left|\frac{(x-2)^{n+1}}{(n+1)^2+1}\frac{n^2+1}{(x-2)^n}\right|
			=\lim_{n\to\infty}\left|\frac{(x-2) \left(n^2+1\right)}{(n+1)^2+1}\right| \\
			&=\lim_{n\to\infty}\left|x-2\right|\frac{n^2+1}{n^2+2n+2}
			=|x-2|<1
		\end{aligned}
	\end{aligned}$

	$\displaystyle x=3: \sum \frac{1}{n^2+1}$ converges by comparison to $\dfrac{1}{n^2}$.

	$\displaystyle x=1: \sum \frac{(-1)^n}{n^2+1}$ converges by the absolute convergence test.

	Radius of convergence: $1$, Interval of convergence: \boxed{[1,3]}.

\item[(h)]
	Center of convergence: $x=-4$.

	$\begin{aligned}[t]
		L<1
		&\iff 
		\begin{aligned}[t]
			\lim_{n\to\infty}\left|\frac{a_{n+1}}{a_n}\right|
			&=\lim_{n\to\infty}\left|\frac{3^{n+1}(x+4)^{n+1}}{\sqrt{n+1}}\frac{\sqrt{n}}{3^n(x+4)^n}\right|
			=\lim_{n\to\infty}\frac{\left|3(x+4)\right|\sqrt{n}}{\sqrt{n+1}} \\
			&=\left|3(x+4)\right|<1
			\iff \left|x+4\right|<\frac13
		\end{aligned}
	\end{aligned}$

	$\displaystyle x=-\frac{11}{3}: \sum \frac{3^n \left(\frac13\right)^n}{\sqrt{n}}=\sum \frac{1}{n^{1/2}}$ diverges by the $p$-series test.

	$\displaystyle x=-\frac{13}{3}: \sum \frac{3^n \left(-\frac13\right)^n}{\sqrt{n}}=\sum \frac{(-1)^n}{\sqrt{n}}$ converges by the alternating series test.

	Radius of convergence: $1/2$, Interval of convergence: \boxed{[-13/3,-11/3)}.

\item[(i)]
	Center of convergence: $x=2$.

	$\begin{aligned}[t]
		L<1
		&\iff 
		\begin{aligned}[t]
			\lim_{n\to\infty}\left|\frac{a_{n+1}}{a_n}\right|
			&=\lim_{n\to\infty}\left|\frac{(x-2)^{n+1}}{(n+1)^{n+1}}\frac{n^n}{(x-2)^n}\right|
			=\lim_{n\to\infty}\left|x-2\right|\frac{n^n}{(n+1)^n}\frac{1}{(n+1)^n} \\
			&=\lim_{n\to\infty}\left|x-2\right|\frac{1}{(n+1)^n}
			=0<1
		\end{aligned}
	\end{aligned}$

	Interval of convergence: \boxed{(-\infty,\infty)}.

\item[(j)]
	Center of convergence: $x=a$.

	$\begin{aligned}[t]
		L<1
		&\iff 
		\begin{aligned}[t]
			\lim_{n\to\infty}\left|\frac{a_{n+1}}{a_n}\right|
			&=\lim_{n\to\infty}\left|\frac{(n+1)(x-a)^{n+1}}{b^{n+1}}\frac{b^n}{n(x-a)^n}\right|
			=\lim_{n\to\infty}\left|\frac{(n+1)(x-a)}{bn}\right| \\
			&=\lim_{n\to\infty}\left|\frac{n+1}{n}\frac{x-a}{b}\right|
			=\left|\frac{x-a}{b}\right|<1
			\iff \left|x-a\right|<b
		\end{aligned}
	\end{aligned}$

	$\displaystyle x=a+b: \sum \frac{n}{b^n}(b+a-a)^n=\sum \frac{n}{b^n}b^n=\sum n$ diverges by the $n$th term divergence test.

	$\displaystyle x=a-b: \sum \frac{n}{b^n}(a-b+a)^n=\sum (-1)^n n$ diverges by the $n$th term divergence test.

	Radius of convergence: $b$, Interval of convergence: \boxed{(a-b,a+b)}.

\item[(k)]
	Center of convergence: $x=1/2$.

	$\begin{aligned}[t]
		L<1
		&\iff 
		\begin{aligned}[t]
			\lim_{n\to\infty}\left|\frac{a_{n+1}}{a_n}\right|
			&=\lim_{n\to\infty}\left|\frac{(n+1)!(2x-1)^{n+1}}{n!(2x-1)^n}\right|
			=\lim_{n\to\infty}(n+1)\left|2x-1\right|
			=\infty<1
		\end{aligned}
	\end{aligned}$

	This is clearly not possible.

	$\displaystyle x=\frac12: \sum n! 0^n = \sum 0$ is convergent.

	Radius of convergence: $0$, Interval of convergence: \boxed{[1/2,1/2]}.

\item[(l)]
	Center of convergence: $x=-1/4$.

	$\begin{aligned}[t]
		L<1
		&\iff 
		\begin{aligned}[t]
			\lim_{n\to\infty}\left|\frac{a_{n+1}}{a_n}\right|
			&=\lim_{n\to\infty}\left|\frac{(4x+1)^{n+1}}{(n+1)^2}\frac{n^2}{(4x+1)^n}\right|
			=\lim_{n\to\infty}\left|4x+1\right|\frac{n^2}{(n+1)^2} \\
			&=|4x+1|<1
		\end{aligned}
	\end{aligned}$

	$\displaystyle x=0: \sum \frac{1}{n^2}$ is convergent by the $p$-series test.

	$\displaystyle x=-1/2: \sum \frac{(-1)^n}{n^2}$ is convergent by the absolute convergence test.

	Radius of convergence: $1/4$, Interval of convergence: \boxed{[-1/2,0]}.

\item[(m)]
	Center of convergence: $x=0$.

	$\begin{aligned}[t]
		L<1
		&\iff 
		\begin{aligned}[t]
			\lim_{n\to\infty}\left|\frac{a_{n+1}}{a_n}\right|
			&=\lim_{n\to\infty}\left|\frac{x^{n+1}}{1\cdot3\cdot5\cdot\ldots\cdot(2n-1)(2n+1)}\frac{1\cdot3\cdot5\cdot\ldots\cdot(2n-1)}{x^n}\right| \\
			&=\lim_{n\to\infty}\left|\frac{x}{2n+1}\right|
			=0<1
		\end{aligned}
	\end{aligned}$

	Interval of convergence: \boxed{(-\infty,\infty)}.
\end{itemize}

\problem{2}
By the ratio test, if $\displaystyle \sum_{n=0}^{\infty}c_n4^n$ is convergent and $L=\displaystyle \lim_{n\to\infty}\left|\frac{a_{n+1}}{a_n}\right|\leq1$. (\textit{If $L<1$, then the series must converge, and if $L>1$, the series must diverge. But if $L=1$ then the test is inconclusive, so $L$ could be $1$ even when series is convergent.}) Therefore:
\begin{equation*}
	L
	=\lim_{n\to\infty}\left|\frac{c_{n+1}4^{n+1}}{c_n4^n}\right|
	=\lim_{n\to\infty}4 \left|\frac{c_{n+1}}{c_n}\right|\leq1
	\implies \lim_{n\to\infty}\left|\frac{c_{n+1}}{c_n}\right|\leq \frac14
\end{equation*}

\begin{itemize}
	\item[(a)]
		\begin{equation*}
			L=\lim_{n\to\infty}\left|\frac{c_{n+1}(-2)^{n+1}}{c_n(-2)^n}\right|
			=2\lim_{n\to\infty}\left|\frac{c_{n+1}}{c_n}\right|
			\leq 2\cdot\frac14
			=\frac12
		\end{equation*}
		Therefore the series converges by the ratio test as $L=1/2<1$.
	
	\item[(b)]
		\begin{equation*}
			L=\lim_{n\to\infty}\left|\frac{c_{n+1}(-4)^{n+1}}{c_n(-4)^n}\right|
			=4\lim_{n\to\infty}\left|\frac{c_{n+1}}{c_n}\right|
			\leq 4\cdot\frac14
			=1
		\end{equation*}
		The ratio test is inconclusive since $L=1$, so the series could either converge or diverge. It \underline{does not} follow that the series must be convergent.

\end{itemize}

\problem{3}
Center of convergence: $x=0$.

$\begin{aligned}[t]
    L<1 \iff
	\begin{aligned}[t]
		\lim_{n\to\infty}\left|\frac{a_{n+1}}{a_n}\right|
		&=\lim_{n\to\infty}\left|\frac{((n+1)!)^k x^{n+1}}{(k(n+1))!}\frac{(kn)!}{(n!)^k x^n}\right|
		=\lim_{n\to\infty}\left|\frac{(n!)^k (n+1)^k x^n x (kn)!}{(kn+k)!(n!)^k x^n}\right| \\
		&=\lim_{n\to\infty}\left|\frac{(n+1)^k x (kn)!}{(kn+k)!}\right|
		=\lim_{n\to\infty}|x| \frac{(n+1)^k (kn)!}{(kn)!(kn+1)(kn+2)\ldots(kn+k)} \\
		&=|x|\lim_{n\to\infty}\frac{(n+1)^k}{\underbrace{(kn+1)(kn+2)\ldots(kn+k)}_{\text{product with $k$ terms}}} \\
		&=|x|\lim_{n\to\infty}\frac{n^k+\overbrace{\ldots\ldots\ldots\ldots\ldots\ldots}^{\text{terms of lower power}}}{k^kn^k+\underbrace{\ldots\ldots\ldots\ldots\ldots\ldots}_{\text{terms of lower power}}}
		=|x|\lim_{n\to\infty}\frac{1}{k^k}
		=\frac{|x|}{k^k}<1
		\iff |x|<k^k
	\end{aligned}
\end{aligned}$

Radius of convergence: \boxed{k^k}.

\problem{4}
No. Suppose we are able to find such a series with radius $R$ and center of convergence $c$. Then, we must have that $c=R$ so that the distance to $0$ is $R$. However, $c$ cannot be a finite distance $R$ from infinity, so this series cannot exist.

\problem{5}
$\begin{aligned}[t]
	f(x)
	&=1+2x+x^2+2x^3+x^4+2x^5+\ldots 
	=\left(1+x+x^2+x^3+x^4+x^5\right)+\left(x+x^3+x^5+\ldots\right) \\
	&=\sum_{n=0}^{\infty}x^n+\sum_{n=0}^{\infty}x^{2n+1}
	=\sum_{n=0}^{\infty}x^n+x\sum_{n=0}^{\infty}\left(x^2\right)^n
	=\boxed{\frac{1}{1-x}+\frac{x}{1-x^2}}
\end{aligned}$

Notice that $f(x)$ equals the sum of two geometric series with common ratios $x$ and $x^2$. By the geometric series test, we must have:
\begin{equation*}
	\left\{\begin{gathered}
		\left|x\right|<1 \\
		\left|x^2\right|<1 \iff \left|x\right|<1
	\end{gathered}\right.
	\iff x\in(-1,1)
\end{equation*}
Therefore, the interval of convergence is \boxed{(-1,1)}.

\problem{6}

By the ratio test, $\displaystyle \sum a_n$ converges if $L=\displaystyle \lim_{n\to\infty}\sqrt[n]{\left|a_n\right|}<1$. We also have $c=\displaystyle \lim_{n\to\infty} \sqrt[n]{\left|c_n\right|}\neq0$.
\begin{equation*}
	L<1
	\iff \lim_{n\to\infty}\sqrt [n]{\left|a_n\right|}
	=\lim_{n\to\infty} \sqrt[n]{c_n x^n}
	=\lim_{n\to\infty} \left|x\right|\sqrt[n]{c_n}
	=c \left|x\right|<1
	\iff \left|x\right|<\frac1c
\end{equation*}
So we have the radius of convergence is $1/c$. \qed

\problem{7}
Radius of convergens is $\min(2,3)=2$.

\end{document}
