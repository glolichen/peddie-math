% JUMP TO LINE 60, 75
\documentclass[preview, margin=0.6in]{standalone}
\usepackage[letterpaper,portrait,top=0.4in, left=0.6in, right=0.6in, bottom=1in]{geometry}

\usepackage{amsmath, amsfonts, amsthm, amssymb}
\usepackage{graphicx, float}
\usepackage{mathtools}
\usepackage{titlesec}
\usepackage{interval}
\usepackage{hyperref}
\usepackage{siunitx}
\usepackage{titling}
\usepackage{vwcol}
\usepackage{setspace}
\usepackage{empheq}
\usepackage{cancel}
\usepackage{esdiff}
\usepackage{multicol}
\usepackage{mdframed}
\usepackage{esdiff}
\usepackage{tikzsymbols}
\usepackage{multicol}
\usepackage{tikz}
\usepackage{varwidth}
\usepackage{parskip}
\usepackage{pgfplots}
\pgfplotsset{compat=1.18}
\intervalconfig {
	soft open fences
}

\newcommand{\alignedintertext}[1]{%
  \noalign{%
    \vtop{\hsize=\linewidth#1\par
    \expandafter}%
    \expandafter\prevdepth\the\prevdepth
  }%
}

\newtheorem{lemma}{Lemma}

\renewcommand{\qedsymbol}{\Smiley[1.3]}
\newcommand*{\problem}[1]{\section*{Problem #1}}
\newcommand*{\aps}{\section*{AP Corner}}
\newcommand*{\deriv}[1][x]{\ensuremath{\dfrac{\mathrm{d}}{\mathrm{d}#1}}}
\newcommand*{\floor}[1]{\ensuremath{\lfloor #1\rfloor}}
\newcommand*{\lheqzero}{\ensuremath{\underset{\text{L'H}}{\overset{\left[\frac00\right]}{=}}}}
\newcommand*{\lheqinfty}{\ensuremath{\underset{\text{L'H}}{\overset{\left[\frac{\infty}{\infty}\right]}{=}}}}

\DeclareMathOperator{\DNE}{DNE}
\DeclareMathOperator{\sgn}{sgn}

\DeclareMathOperator{\arccsc}{arccsc}
\DeclareMathOperator{\arcsec}{arcsec}
\DeclareMathOperator{\arccot}{arccot}

%opening

\title{\vspace*{-40pt}Problem Set \#60}
\author{Jayden Li}
\date{\today}
% \allowdisplaybreaks
\postdisplaypenalty=100000

\begin{document}
\setstretch{1.25}
\fontsize{12pt}{12pt}\selectfont
\setlength{\abovedisplayskip}{\abovedisplayskip/2}
\setlength{\belowdisplayskip}{\belowdisplayskip/2}
\setlength{\parindent}{0pt}
\setlength{\parskip}{2ex plus 0.5ex minus 0.2ex}
\maketitle

\problem{1}
\begin{itemize}
\item[(a)]
$\begin{aligned}[t]
	\left.\diff yx\right|_{\theta=\pi/6}^{}
	&=\left.\frac{2\cos\theta \sin \theta+2\sin \theta\cos \theta}{2\cos^2\theta-2\sin^2\theta}\right|_{\theta=\pi/6}
	=\frac{\frac{\sqrt{3}}{2}\frac{1}{2} \cdot 2}{\left(\frac{\sqrt{3}}{2}\right)^2-\left(\frac{1}{2}\right)^2}
	=\frac{\frac{\sqrt{3}}{2}}{\frac{3}{4}-\frac{1}{4}}
	=\sqrt{3} \\
	x_0
	&=r\cos\frac{\pi}{6}
	=2\sin \frac{\pi}{6}\cos \frac{\pi}{6}
	=2 \frac{\sqrt{3}}{2}\frac{1}{2}
	=\frac{\sqrt{3}}{2} \\
	y_0
	&=r\sin \frac{\pi}{6}
	=2\sin \frac{\pi}{6} \sin \frac{\pi}{6}
	=2 \frac{1}{2}\frac{1}{2}
	=\frac{1}{2} \\
	\Aboxed{y-\frac{1}{2}&=\sqrt{3}\left(x-\frac{\sqrt{3}}{2}\right)}
\end{aligned}$

\item[(b)]
$\begin{aligned}[t]
	\left. \diff yx \right|_{\theta=\pi}
	&=\left.\frac{-\frac{1}{\theta^2}\sin\theta+\frac{1}{\theta}\cos\theta}{-\frac{1}{\theta^2}\cos\theta-\frac{1}{\theta}\sin \theta}\right|_{\theta=\pi}
	=\frac{-\frac{1}{\pi^2}\cdot0+\frac{1}{\pi}(-1)}{-\frac{1}{\pi^2}(-1)-\frac{1}{\pi}\cdot0}
	=\frac{-\frac{1}{\pi}}{\frac{1}{\pi^2}}\cdot \frac{\pi^2}{\pi^2}
	=-\pi \\
	x_0
	&=r\cos\pi
	=\frac{1}{\pi}\cos\pi
	=\frac{1}{\pi}(-1)
	=-\frac{1}{\pi} \\
	y_0
	&=r\sin\pi 
	=\frac{1}{\pi}\cdot0 
	=0 \\
	\Aboxed{y&=-\pi \left(x+\frac{1}{\pi}\right)}
\end{aligned}$

\item[(c)]
$\begin{aligned}[t]
    \left. \diff yx \right|_{\theta=\pi/4}
	&=\left.\frac{-2\sin2\theta \sin\theta+\cos2\theta\cos\theta}{-2\sin2\theta\cos\theta-\cos2\theta\sin\theta}\right|_{\theta=\pi/4}
	=\frac{-2\cdot 1\cdot \frac{\sqrt{2}}{2}+0}{-2\cdot1\cdot \frac{\sqrt{2}}{2}-0}
	=1 \\
	x_0
	&=r\cos \frac{\pi}{4}
	=\cos \frac{\pi}{2} \cos \frac{\pi}{4}
	=0 \\
	y_0
	&=r\sin \frac{\pi}{4}
	=\cos \frac{\pi}{2} \sin \frac{\pi}{4}
	=0 \\
	\Aboxed{y&=x}
\end{aligned}$
\end{itemize}

\problem{2}
\begin{mdframed}
	For some polar graph $r=f(\theta)$, we have:
	\begin{align*}
		x(\theta)&=r\cos(\theta)=f(\theta)\cos(\theta) \\
		y(\theta)&=r\sin(\theta)=f(\theta)\sin(\theta)
	\end{align*}
	From here, we can calculate $\mathrm{d}y/\mathrm{d}x$ and $\mathrm{d}x/\mathrm{d}y$.
	\begin{align*}
		\diff yx
		&=\frac{\mathrm{d}y/\mathrm{d}\theta}{\mathrm{d}x/\mathrm{d}\theta}
		=\frac{f'(\theta)\sin\theta+f(\theta)\sin\theta}{f'(\theta)\cos\theta-f(\theta)\sin\theta} \\
		\diff xy
		&=\frac{\mathrm{d}x/\mathrm{d}\theta}{\mathrm{d}y/\mathrm{d}\theta}
		=\frac{f'(\theta)\cos\theta-f(\theta)\sin\theta}{f'(\theta)\sin\theta+f(\theta)\sin\theta}
		=\left(\diff yx\right)^{-1}
	\end{align*}
\end{mdframed}
\begin{itemize}
\item[(a)]
$\begin{aligned}[t]
	&{}\diff yx
	=\frac{-3\sin\theta\sin\theta+3\cos\theta\cos\theta}{-3\sin\theta\cos\theta-3\cos\theta\sin\theta}
	=\frac{-\sin^2\theta+\cos^2\theta}{-\sin\theta\cos\theta-\cos\theta\sin\theta}
	=\frac{\sin^2\theta-\cos^2\theta}{2\sin\theta\cos\theta}=0 \\
	\implies{}&\sin^2\theta-\cos^2\theta=0
	\implies \sin^2\theta=\cos^2\theta
	\implies \cos\theta=\sin\theta=\pm \frac{\sqrt{2}}{2} \\
	\implies{}& x=r\cos\theta
	\implies x=3\cos\theta\cos\theta
	=3\cos^2\theta
	=3 \frac{1}{2}
	=\frac{3}{2} \\
	\implies{}& y=r\sin\theta
	\implies y=3\cos\theta\sin\theta
	=\pm \frac{3}{2}
	% \diff xy
	% &= \left(\diff yx\right)^{-1}
	% =\frac{2\sin\theta\cos\theta}{\sin^2\theta-\cos^2\theta} \\
\end{aligned}$

Horizontal tangent line: $\displaystyle \left(\frac32,\pm\frac32\right)$.

$\begin{aligned}[t]
	&\diff xy= \frac{2\sin\theta\cos\theta}{\sin^2\theta-\cos^2\theta}=0\implies \theta=\frac{n\pi}{2},n\in\mathbb Z \\
	\alignedintertext{Notice: $\cos\theta,\sin\theta\in \left\{0,1\right\}$.}
	\implies{}& x=r\cos\theta=3\cos^2\theta\in\{0,1\} \\
	\implies{}& y=r\sin\theta=3\cos\theta\sin\theta=0
\end{aligned}$

Vertical tangent line: $(0,0),(1,0)$.

\item[(b)]
\item[(c)]
\end{itemize}

\problem{3}
\begin{align*}
	r&=a\sin\theta+b\cos\theta \\
	r^2&=ar\sin\theta+br\cos\theta \\
	x^2+y^2&=ay+bx \\
	x^2-bx+y^2-ay&=0 \\
	\left(x-\frac{b}{2}\right)^2-\frac{b^2}{4}+\left(y-\frac{a}{2}\right)^2-\frac{a^2}{4}&=0 \\
	\left(x-\frac{b}{2}\right)^2+\left(y-\frac{a}{2}\right)^2&=\frac{b^2}{4}+\frac{a^2}{4}
\end{align*}
Center: $\displaystyle \left(\frac b2, \frac a2\right)$. Radius: $\displaystyle \frac12\sqrt{a^2+b^2}$.

\problem{4}


\end{document}
