% JUMP TO LINE 60, 73
\documentclass[preview, margin=0.6in]{standalone}
\usepackage[letterpaper,portrait,top=0.4in, left=0.6in, right=0.6in, bottom=1in]{geometry}

\usepackage{amsmath, amsfonts, amsthm, amssymb}
\usepackage{graphicx, float}
\usepackage{mathtools}
\usepackage{titlesec}
\usepackage{interval}
\usepackage{hyperref}
\usepackage{siunitx}
\usepackage{titling}
\usepackage{vwcol}
\usepackage{setspace}
\usepackage{empheq}
\usepackage{cancel}
\usepackage{esdiff}
\usepackage{multicol}
\usepackage{mdframed}
\usepackage{esdiff}
\usepackage{tikzsymbols}
\usepackage{multicol}
\usepackage{tikz}
\usepackage{varwidth}
\usepackage{pgfplots}
\pgfplotsset{compat=1.18}
\intervalconfig {
	soft open fences
}

\newcommand{\alignedintertext}[1]{%
  \noalign{%
    \vskip\belowdisplayshortskip
    \vtop{\hsize=\linewidth#1\par
    \expandafter}%
    \expandafter\prevdepth\the\prevdepth
  }%
}

\newtheorem{lemma}{Lemma}

\renewcommand{\qedsymbol}{\Smiley[1.3]}
\newcommand*{\problem}[1]{\section*{Problem #1}}
\newcommand*{\aps}{\section*{AP Corner}}
\newcommand*{\deriv}[1][x]{\ensuremath{\dfrac{\mathrm{d}}{\mathrm{d}#1}}}
\newcommand*{\floor}[1]{\ensuremath{\lfloor #1\rfloor}}
\newcommand*{\lheqzero}{\ensuremath{\underset{\text{L'H}}{\overset{\left[\frac00\right]}{=}}}}
\newcommand*{\lheqinfty}{\ensuremath{\underset{\text{L'H}}{\overset{\left[\frac{\infty}{\infty}\right]}{=}}}}

\DeclareMathOperator{\DNE}{DNE}
\DeclareMathOperator{\sgn}{sgn}

\DeclareMathOperator{\arccsc}{arccsc}
\DeclareMathOperator{\arcsec}{arcsec}
\DeclareMathOperator{\arccot}{arccot}

\setlength{\parindent}{0pt}

%opening
\title{\vspace*{-30pt}Problem Set \#9}
\author{Jayden Li}
\date{\today}

% \allowdisplaybreaks
\postdisplaypenalty=100000

\begin{document}
\setstretch{1.25}
\fontsize{12pt}{12pt}\selectfont
\setlength{\abovedisplayskip}{0pt}
\maketitle
\problem{2}
\textit{Is there a mistake in this question? Is the caption wrong? It says the left is $f'(x)$ and the right is for graphing $f(x)$, but the question says the left is $f(x)$ and the right is for graphing $F(x)$. The question as described in the label would be above our level as we would be finding second antiderivatives.}
\begin{itemize}
	\item[(a)] increasing on $(0,2)\cup(5,7)$, decreasing on $(2,5)$
	\item[(b)] concave up on $(0,1)\cup(4,6)$, concave down on $(1,3)\cup(6,7)$, neither on $(\infty,0]\cup\{1\}\cup[3,4]\cup\{6\}\cup[7,\infty)$
	\item[(c)] relative minimum at $\displaystyle \left(0,0\right), \left(5,\frac{\pi}{4}-\frac{3}{2}\right)$, relative maximum at $\displaystyle \left(2,\frac{1}{2}+\frac{\pi}{4}\right), \left(7, \frac{3\pi}{4}-\frac{3}{2}\right)$
	\item[(d)]
		\begin{align*}
		    F(-1)&=1 \\
			F(1)&=\frac32 \\
			F(2)&=\frac32+\frac{\pi}{4} \\
			F(3)&=1+\frac{\pi}{4} \\
			F(4)&=\frac{\pi}{4} \\
			F(5)&=\frac{\pi}{4}-\frac{1}{2} \\ 
			F(6)&=\frac{\pi}{2}-\frac{1}{2} \\
			F(7)&=\frac{3\pi}{4}-\frac{1}{2} \\
			F(8)&=\frac{3\pi}{4}-\frac{1}{2}
		\end{align*}

	\item[(e)] Other files.

	\item[(f)] Graph is shifted 1 unit down. (a) unchanged. (b) unchanged. $x$-coordinates for (c) are unchanged by $y$-coordinates shifted down by 1. Every value in (d) is subtracted by 1.
\end{itemize}

\problem{5}
\begin{itemize}
	\item[(b)]
		\begin{equation*}
			\text{Average}
			=\frac{1}{2-(-1)}\int_{-1}^{2}x^2\,\mathrm{d}x
			=\frac13\left(\frac{2^3}{3}-\frac{(-1)^3}{3}\right)
			=\frac13\cdot \frac{8+1}{3}
			=\frac13\cdot3
			=1
		\end{equation*}
		Average value occurs at $x=-1$ and $x=1$.
	\item[(c)]
		\begin{align*}
			\text{Average}
			&=\frac{1}{2-0}\int_{0}^{2}\left(1-\cos \frac{\pi t}{2}\right)\mathrm{d}t
			=\frac12 \left[t-\frac{2}{\pi}\sin \frac{\pi t}{2}\right]_{0}^{2}
			=\frac12 \left(2-\frac{2}{\pi}\sin(\pi)-0+\frac{2}{\pi}\sin0\right) 
			=2
		\end{align*}
		Average volume of air in lungs is 2 pints between 0 and 2 seconds.
\end{itemize}

\problem{6}
\begin{itemize}
	\item[(i)]
	\begin{itemize}
		\item[(a)]
			\begin{align*}
			    \frac{1}{5-2}\int_{2}^{5}(x-3)^2\,\mathrm{d}x
				&=\frac13 \left[\frac{(x-3)^3}{3}\right]_{2}^{5}
				=\frac13\left(\frac83-(-1)\frac{1}{3}\right)
				=\boxed{3}
			\end{align*}
		\item[(b)]
			\begin{align*}
				(x-3)^2&=3 \\ 
				x-3&=\pm\sqrt3 \\ 
				\Aboxed{x&=3\pm \sqrt{3}}
			\end{align*}
		\item[(c)] Other files.
	\end{itemize}

	\item[(ii)]
	\begin{itemize}
		\item[(a)]
			\begin{align*}
				\frac{1}{\pi-0}\int_{0}^{\pi}\left(2\sin x-\sin2x\right)\,\mathrm{d}x
				&=\frac{1}{\pi}\left[-2\cos x+\frac12\cos2x\right]_{0}^{\pi} \\
				&=\frac{1}{\pi}\left(-2(-1)+\frac12(1)-\left(-2(1)+\frac12(1)\right)\right) \\
				&=\frac{1}{\pi}\left(\frac42+\frac12+\frac42-\frac12\right)
				=\frac{1}{\pi}\cdot4
				=\boxed{\frac{4}{\pi}}
			\end{align*}
		\item[(b)]
			\begin{align*}
				2\sin x-\sin2x&=\frac{4}{\pi} \\ 
				\Aboxed{x&\in\left\{1.238, 2.808\right\}} \tag{proof by calculator}
			\end{align*}
		\item[(c)] Other files.
	\end{itemize}
\end{itemize}

\problem{7}
By the Mean Value Theorem for Integrals, it is known that there exists some $c\in[1,3]$ such that
\begin{equation*}
    f(c)=\frac{1}{3-1}\int_{1}^{3}f(x)\,\mathrm{d}x=\frac12\cdot8=4
\end{equation*}
So, $f(x)=4$ for some $x\in[1,3]$.\qed

\problem{8}
\begin{align*}
	\frac1b \int_{0}^{b}\left(2+6x-3x^2\right)\,\mathrm{d}x&=3 \\
	\frac1b \left[2x+3x^2-x^3\right]_{0}^{b}&=3 \\ 
	\frac1b \left(2b+3b^2-b^3\right)&=3 \\ 
	2+3b-b^2&=3 \\
	b^2-3b+1&=0 \\ 
	\Aboxed{b&=\frac{3\pm \sqrt{5}}{2}}
\end{align*}

\problem{9}
\begin{align*}
    \frac{1}{50-20}\int_{20}^{50}f(x)\,\mathrm{d}x
	&\approx \frac{1}{30}\cdot5\left(40+34.5+30+32+41.5+54\right)
	=\frac16\cdot232=\boxed{\frac{116}{3}}
\end{align*}

\problem{10}
\begin{itemize}
	\item[(a)]
		\begin{align*}
		    \frac{1}{4-1}\int_{1}^{4}14\pi x^2\,\mathrm{d}x
			&=\frac13 \left[\frac{14\pi x^3}{3}\right]_{1}^{4}
			=\frac13 \left(\frac{896\pi}{3}-\frac{14\pi}{3}\right)
			=\boxed{98\pi}
		\end{align*}
	\item[(b)]
		\begin{align*}
			\frac{1}{k}\int_{0}^{k}k^2 \sin\left(\frac{\pi x}{2k}\right)\,\mathrm{d}x&=98\pi \\
			k\left[-\frac{2k}{\pi}\cos\left(\frac{\pi x}{2k}\right)\right]_0^k&=98\pi \\
			k\left(-\frac{2k}{\pi}\cos\left(\frac{\pi k}{2k}\right)-\left(-\frac{2k}{\pi}\cos(0)\right)\right)&=98\pi \\
			k\left(\frac{2k}{\pi}\right)&=98\pi \\
			2k^2&=98\pi^2 \\ 
			\Aboxed{k&=7\pi}
		\end{align*}
\end{itemize}


\end{document}
