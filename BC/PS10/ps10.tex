% JUMP TO LINE 60, 73
\documentclass[preview, margin=0.6in]{standalone}
\usepackage[letterpaper,portrait,top=0.4in, left=0.6in, right=0.6in, bottom=1in]{geometry}

\usepackage{amsmath, amsfonts, amsthm, amssymb}
\usepackage{graphicx, float}
\usepackage{mathtools}
\usepackage{titlesec}
\usepackage{interval}
\usepackage{hyperref}
\usepackage{siunitx}
\usepackage{titling}
\usepackage{vwcol}
\usepackage{setspace}
\usepackage{empheq}
\usepackage{cancel}
\usepackage{esdiff}
\usepackage{multicol}
\usepackage{mdframed}
\usepackage{esdiff}
\usepackage{tikzsymbols}
\usepackage{multicol}
\usepackage{tikz}
\usepackage{varwidth}
\usepackage{pgfplots}
\pgfplotsset{compat=1.18}
\intervalconfig {
	soft open fences
}

\newcommand{\alignedintertext}[1]{%
  \noalign{%
    \vskip\belowdisplayshortskip
    \vtop{\hsize=\linewidth#1\par
    \expandafter}%
    \expandafter\prevdepth\the\prevdepth
  }%
}

\newtheorem{lemma}{Lemma}

\renewcommand{\qedsymbol}{\Smiley[1.3]}
\newcommand*{\problem}[1]{\section*{Problem #1}}
\newcommand*{\aps}{\section*{AP Corner}}
\newcommand*{\deriv}[1][x]{\ensuremath{\dfrac{\mathrm{d}}{\mathrm{d}#1}}}
\newcommand*{\floor}[1]{\ensuremath{\lfloor #1\rfloor}}
\newcommand*{\lheqzero}{\ensuremath{\underset{\text{L'H}}{\overset{\left[\frac00\right]}{=}}}}
\newcommand*{\lheqinfty}{\ensuremath{\underset{\text{L'H}}{\overset{\left[\frac{\infty}{\infty}\right]}{=}}}}

\DeclareMathOperator{\DNE}{DNE}
\DeclareMathOperator{\sgn}{sgn}

\DeclareMathOperator{\arccsc}{arccsc}
\DeclareMathOperator{\arcsec}{arcsec}
\DeclareMathOperator{\arccot}{arccot}

\setlength{\parindent}{0pt}

%opening
\title{\vspace*{-30pt}Problem Set \#10}
\author{Jayden Li}
\date{\today}

% \allowdisplaybreaks
\postdisplaypenalty=100000

\begin{document}
\setstretch{1.25}
\fontsize{12pt}{12pt}\selectfont
\setlength{\abovedisplayskip}{0pt}
\maketitle
\problem{1}
\begin{itemize}
	\item[(a)]
		\begin{equation*}
			L_3=5.000
		\end{equation*}
	\item[(b)]
		\begin{align*}
			R_3&=14.000 \\ 
			M_3&=8.750
		\end{align*}
	\item[(c)]
		\begin{equation*}
		    \int_{0}^{3}x^2\,\mathrm{d}x=\left.\frac{x^2}{2}\right|_{0}^{3}=\frac{9}{2}
		\end{equation*}
	\item[(d)]
		\begin{align*}
			\text{Error }L_3&=-4.000 \\
			\text{Error }R_3&=5.000 \\
			\text{Error }M_3&=-0.250 \\
		\end{align*}
	\item[(e)]
		\begin{equation*}
			\text{Area}=\frac{\left(b_1+b_2\right)h}{2}
		\end{equation*}
	\item[(f)]
		\begin{align*}
		    \int_{0}^{3}x^2\,\mathrm{d}x
			&\approx \frac{(0+1)1}{2}+\frac{(1+4)1}{2}+\frac{(4+9)1}{2}
			=\frac{1}{2}+\frac{5}{2}+\frac{13}{2}
			=\frac{19}{2}
		\end{align*}
\end{itemize}

\problem{2}
\begin{itemize}
	\item[(a)]
		\begin{align*}
		    \int_{1}^{2}\frac{1}{x^2}\,\mathrm{d}x
			&=\left[-\frac{1}{x}\right]_{1}^{2}
			=-\frac{1}{2}-(-1)
			=\frac{1}{2}
		\end{align*}
	\item[(b)]
		\begin{align*}
			T_4&=0.509 \\
			M_4&=0.496 \\
			T_8&=0.502 \\
			M_8&=0.499
		\end{align*}
	\item[(c)]
		\begin{align*}
			E_{T_4}&=0.509-\frac{1}{2}=0.009 \\
			E_{M_4}&=0.496-\frac{1}{2}=-0.004 \\
			E_{T_8}&=0.502-\frac{1}{2}=0.002 \\
			E_{M_8}&=0.499-\frac{1}{2}=-0.001
		\end{align*}
	\item[(d)] Trapezoidal rule overestimates, midpoint rule underestimates
	\item[(e)] Concave up on $[1,2]$
\end{itemize}

\problem{3}
\begin{itemize}
	\item[(a)]
		\begin{align*}
		    \int_{0}^{1}\left(1-x^2\right)\mathrm{d}x
			&=\left[x-\frac{x^3}{3}\right]_{0}^{1}
			=\frac{3}{3}-\frac{1}{3}-0
			=\frac{2}{3}
		\end{align*}
	\item[(b)]
		\begin{align*}
			T_4&=0.656 \\
			M_4&=0.672 \\
			T_8&=0.664 \\
			M_8&=0.668
		\end{align*}
	\item[(c)]
		\begin{align*}
			E_{T_4}&=0.656-0.666=-0.010 \\
			E_{M_4}&=0.672-0.666=0.006 \\
			E_{T_8}&=0.664-0.666=-0.002 \\
			E_{M_8}&=0.668-0.666=0.002
		\end{align*}
	\item[(d)]Midpoint rule overestimates, trapezoidal rule underestimates
	\item[(e)] Concave down on $[0,1]$
	\item[(f)] Error from midpoint and trapezoid has opposite sign, and absolute error from midpoint is approximately half of absolute error from trapezoidal
\end{itemize}

\problem{4}
\begin{itemize}
	\item[(a)] Uploaded in other files
	\item[(b)]
		\begin{equation*}
			\text{Distance}
			=\int_{0}^{1.8}|v(t)|\,\mathrm{d}t
			=\int_{0}^{1.8}v(t)\,\mathrm{d}t
		\end{equation*}
	\item[(c)]
		\begin{align*}
			L_3&=0.6(100+96+80)=165.6 \\
			R_3&=0.6(96+80+0)=105.6 \\
			T_3&=(L_3+R_3)/2=135.6 \\
		\end{align*}
	\item[(d)]
		\begin{equation*}
		    M_3=0.6(99+90+50)=143.4
		\end{equation*}
	\item[(e)]
		\begin{equation*}
			S_6=\frac{2M_3+T_3}{3}=\frac{2\cdot 71.7+67.8}{3}=140.8
		\end{equation*}
	\item[(f)] \underline{$140.8$ feet} is the best estimate because it is known that Simpson's Rule has the lowest error
\end{itemize}

\problem{5}
\begin{itemize}
	\item[(a)]
		Graphs in other files
		\begin{gather*}
			L_{1_f}=L_{1_g}=L{1_h}=2 \\
			R_{1_f}=R_{1_g}=R{1_h}=1
		\end{gather*}
		$L_1$ and $R_1$ are the same for each function

	\item[(b)]
		\begin{align*}
		    \int_{0}^{1}f(x)\,\mathrm{d}x&\approx M_{1_f}=\frac{7}{4} \\
			\int_{0}^{1}g(x)\,\mathrm{d}x&\approx M_{1_g}=\frac{15}{8} \\
			\int_{0}^{1}h(x)\,\mathrm{d}x&\approx M_{1_h}=\frac{31}{16}
		\end{align*}
	\item[(c)]
		\begin{alignat*}{2}
			T_{1_f}&=\frac{3}{2} &\qquad S_{2_f}&=\frac{2\cdot\frac{7}{4}+\frac{3}{2}}{3}=\frac{5}{3} \\
			T_{1_g}&=\frac{3}{2} &\qquad S_{2_g}&=\frac{2\cdot\frac{15}{8}+\frac{3}{2}}{3}=\frac{7}{4} \\
			T_{1_h}&=\frac{3}{2} &\qquad S_{2_h}&=\frac{2\cdot\frac{31}{16}+\frac{3}{2}}{3}=\frac{43}{24} 
		\end{alignat*}
	\item[(d)]
		\begin{align*}
			\int_{0}^{1}f(x)\,\mathrm{d}x&=\left[2x-\frac{x^3}{3}\right]_{0}^{1}=\frac{5}{3} \\
			\int_{0}^{1}g(x)\,\mathrm{d}x&=\left[2x-\frac{x^4}{4}\right]_{0}^{1}=\frac{7}{4} \\
			\int_{0}^{1}h(x)\,\mathrm{d}x&=\left[2x-\frac{x^5}{5}\right]_{0}^{1}=\frac{9}{5}
		\end{align*}
	\item[(e)] Simpson's Rule is very accurate even with just a single subdivision
\end{itemize}

\problem{6}
\begin{itemize}
	\item[(a)]
		\begin{equation*}
		    \text{Total Water}=\int_{0}^{60}r(t)\,\mathrm{d}t
		\end{equation*}	
	\item[(b)]
		\begin{equation*}
		    \int_{0}^{60}r(t)\,\mathrm{d}t 
			\approx M_3
			=20\left(2100+3000+5100\right)
			=204000
		\end{equation*}
		The total volume of water to flow through the dam in 60 seconds is approximately 204000 cubic feet. $M_3$ underestimates the value of the integral because $r$ is concave up.
	\item[(c)]
		\begin{align*}
			T_3
			&=20\left(\frac{2000+2400}{2}+\frac{2400+3900}{2}+\frac{3900+6500}{2}\right)
			=211000 \\
			S_6&=\frac{2M_3+T_3}{3}=\frac{2\cdot204000+211000}{3}=\boxed{\frac{619000}{3}}
		\end{align*}
	\item[(d)]
		\begin{align*}
			\frac{1}{60}S_6&=\frac{619000}{3\cdot60}=\frac{30950}{9} \\
			\frac{2000+2100+2400+3000+3900+5100+6500}{7}&=\frac{25000}{7}
		\end{align*}
		Both values esimate the average rate at which water flows through the dam in one minute. The first value (with Simpson's Rule) is more accurate because the second only takes the values of the function at the endpoints of intervals into account.
\end{itemize}

\end{document}
