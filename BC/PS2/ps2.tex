% JUMP TO LINE 60, 73
\documentclass[preview, margin=0.6in]{standalone}
\usepackage[letterpaper,portrait,top=0.4in, left=0.6in, right=0.6in, bottom=1in]{geometry}

\usepackage{amsmath, amsfonts, amsthm, amssymb}
\usepackage{graphicx, float}
\usepackage{mathtools}
\usepackage{titlesec}
\usepackage{interval}
\usepackage{hyperref}
\usepackage{siunitx}
\usepackage{titling}
\usepackage{vwcol}
\usepackage{setspace}
\usepackage{empheq}
\usepackage{cancel}
\usepackage{esdiff}
\usepackage{multicol}
\usepackage{mdframed}
\usepackage{esdiff}
\usepackage{tikzsymbols}
\usepackage{multicol}
\usepackage{tikz}
\usepackage{varwidth}
\usepackage{pgfplots}
\pgfplotsset{compat=1.18}
\intervalconfig {
	soft open fences
}

\newcommand{\alignedintertext}[1]{%
  \noalign{%
    \vskip\belowdisplayshortskip
    \vtop{\hsize=\linewidth#1\par
    \expandafter}%
    \expandafter\prevdepth\the\prevdepth
  }%
}

\newtheorem{lemma}{Lemma}

\renewcommand{\qedsymbol}{\Smiley[1.3]}
\newcommand*{\problem}[1]{\section*{Problem #1}}
\newcommand*{\aps}{\section*{AP Corner}}
\newcommand*{\deriv}[1][x]{\ensuremath{\dfrac{\mathrm{d}}{\mathrm{d}#1}}}
\newcommand*{\floor}[1]{\ensuremath{\lfloor #1\rfloor}}
\newcommand*{\lheqzero}{\ensuremath{\underset{\text{L'H}}{\overset{\left[\frac00\right]}{=}}}}
\newcommand*{\lheqinfty}{\ensuremath{\underset{\text{L'H}}{\overset{\left[\frac{\infty}{\infty}\right]}{=}}}}

\DeclareMathOperator{\DNE}{DNE}
\DeclareMathOperator{\sgn}{sgn}

\DeclareMathOperator{\arccsc}{arccsc}
\DeclareMathOperator{\arcsec}{arcsec}
\DeclareMathOperator{\arccot}{arccot}

\setlength{\parindent}{0pt}

%opening
\title{\vspace*{-30pt}Problem Set \#2}
\author{Jayden Li}
\date{\today}

% \allowdisplaybreaks
\postdisplaypenalty=100000

\begin{document}
\setstretch{1.25}
\fontsize{12pt}{12pt}\selectfont
\setlength{\abovedisplayskip}{0pt}
\maketitle
\problem{1}
\begin{itemize}
	\item[(a)] Done in class.
	\item[(b)]
		\begin{align*}
			L_4&=\frac34\left[v(2+0)+v\left(2+\frac34\right)+v\left(2+\frac32\right)+v\left(2+\frac94\right)\right]
			\approx6.479 \\
			M_4&=\frac34\left[v\left(2+\frac38\right)+v\left(2+\frac98\right)+v\left(2+\frac{15}8\right)+v\left(2+\frac{21}8\right)\right]
			\approx6.635 \\
			R_4&=\frac34\left[v\left(2+\frac34\right)+v\left(2+\frac32\right)+v\left(2+\frac94\right)+v\left(2+3\right)\right]
			\approx6.979
		\end{align*}
	\item[(c)]
		Average does not equal midpoint sum; average should be the Riemann sum if the height of the rectangle is the average of the function evaluated at the left and right ($\Delta x\cdot f(x_{i}+x_{i+1})/2$).
	\item[(d)]
		$L_n$ underestimates on increasing functions and overestimates on decreasing functions. $R_n$ underestimates on decreasing functions and overestimates on increasing functions.
\end{itemize}

\problem{3}
\begin{itemize}
	\item[(a)]
		$\Delta t=4/5, t_0=1, t_1=9/5, t_2=13/5, t_3=17/5, t_4=21/5, t_5=5$.
		\begin{align*}
			M_5&=v\left(\frac{t_0+t_1}{2}\right)\Delta t
			   +v\left(\frac{t_1+t_2}{2}\right)\Delta t
			   +v\left(\frac{t_2+t_3}{2}\right)\Delta t
			   +v\left(\frac{t_3+t_4}{2}\right)\Delta t
			   +v\left(\frac{t_4+t_5}{2}\right)\Delta t \\ 
			   &=\frac45\left[v\left(\frac{7}{5}\right)  
			   +v\left(\frac{11}{5}\right)  
			   +v\left(\frac{15}{5}\right)  
			   +v\left(\frac{19}{5}\right)  
			   +v\left(\frac{23}{5}\right)\right] \\ 
			   &=-1.44
		\end{align*}

	\item[(b)] The object moves $-1.44$ feet to the left.

	\item[(c)]
		\begin{align*}
			\text{Distance}&=\frac45\left[\left|v\left(\frac{7}{5}\right)\right|
			   +\left|v\left(\frac{11}{5}\right)\right|
			   +\left|v\left(\frac{15}{5}\right)\right|
			   +\left|v\left(\frac{19}{5}\right)\right|
			   +\left|v\left(\frac{23}{5}\right)\right|\right] \\ 
			   &=2.336
		\end{align*}
		The object travels $2.336$ feet in total.

	\item[(d)]
		The value of $\lim_{n\to\infty}M_n=\int_{1}^{5}v(t)\,\mathrm{d}t$, or the actual net signed area under/above $v$ on $[1,5]$. This is the change in position of the object.
\end{itemize}

\problem{4}
\begin{itemize}
	\item[(a)]
		\begin{equation*}
		    \sum_{i=1}^{5}2i=2+4+6+8+10
		\end{equation*}
	\item[(b)]
		\begin{equation*}
		    \sum_{i=3}^{7}\left(i^2+1\right)=10+17+26+37+50
		\end{equation*}
	\item[(c)]
		\begin{equation*}
		    \sum_{j=1}^{4}\frac1j=1+\frac12+\frac13+\frac14
		\end{equation*}
	\item[(d)]
		\begin{equation*}
		    \sum_{k=1}^{n}\frac1n(2k+3)=\frac1n\left(5+7+9+\ldots+2n+3\right)
		\end{equation*}
\end{itemize}

\problem{7}
\begin{itemize}
	\item[(a)]
		\begin{align*}
			\sum_{i=1}^{n}\frac{i^2-10}{n^3}
			&=\frac{1}{n^3}\sum_{i=1}^{n}\left[i^2-10\right]
			=\frac{1}{n^3}\left[\sum_{i=1}^{n}i^2-10n\right]
			=\frac{1}{n^3}\left[\frac{n(n+1)(2n+1)}{6}-10n\right] \\ 
			&=\frac{\left(n^2+n\right)(2n+1)-60n}{6n^3}
			=\frac{2n^{2}+3n-59}{6n^2}
		\end{align*}
	\item[(b)]
		\begin{align*}
		    \sum_{i=1}^{n}\left(1+\frac in\right)^2\left(\frac1n\right)
			&=\frac1n \sum_{i=1}^{n}\left(1+\frac{2i}{n}+\frac{i^2}{n^2}\right)
			=\frac1n\left[n+\frac2n \sum_{i=1}^{n}i+\frac{1}{n^2}\sum_{i=1}^{n}i^2\right] \\ 
			&=\frac1n \left[n+\frac{2n(n+1)}{2n}+\frac{n(n+1)(2n+1)}{6n^2}\right] \\
			&=1+\frac{n+1}{n}+\frac{2n^2+3n+1}{6n^2}
		\end{align*}
\end{itemize}

\problem{8}
\begin{itemize}
	\item[(a)]
		\begin{align*}
			\lim_{n\to\infty}\sum_{i=1}^{n}\frac{i^2-10}{n^3}
			&=\lim_{n\to\infty}\frac{2n^2+3n-59}{6n^2}
			=\lim_{n\to\infty}\left[\frac13+\cancelto{0}{\frac{1}{2n}}-\cancelto{0}{\frac{59}{6n^2}}\right]
			=\frac13
		\end{align*}
	\item[(b)]
		\begin{align*}
		    \lim_{n\to\infty}\sum_{i=1}^{n}\left(1+\frac in\right)^2\left(\frac1n\right)
			&=\lim_{n\to\infty}\left[1+\frac{n+1}{n}+\frac{2n^2+3n+1}{6n^2}\right] \\
			&=\lim_{n\to\infty}\left[1+1+\frac1n+\frac13+\frac{1}{2n}+\frac{1}{6n^2}\right]
			=2+\frac13
			=\frac73
		\end{align*}
\end{itemize}

\end{document}
