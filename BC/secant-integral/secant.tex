% JUMP TO LINE 60, 73
\documentclass[preview, margin=0.6in]{standalone}
\usepackage[letterpaper,portrait,top=0.4in, left=0.6in, right=0.6in, bottom=1in]{geometry}

\usepackage{amsmath, amsfonts, amsthm, amssymb}
\usepackage{graphicx, float}
\usepackage{mathtools}
\usepackage{titlesec}
\usepackage{interval}
\usepackage{hyperref}
\usepackage{siunitx}
\usepackage{titling}
\usepackage{vwcol}
\usepackage{setspace}
\usepackage{empheq}
\usepackage{cancel}
\usepackage{esdiff}
\usepackage{multicol}
\usepackage{mdframed}
\usepackage{esdiff}
\usepackage{tikzsymbols}
\usepackage{multicol}
\usepackage{tikz}
\usepackage{varwidth}
\usepackage{pgfplots}
\pgfplotsset{compat=1.18}
\intervalconfig {
	soft open fences
}

\newcommand{\alignedintertext}[1]{%
  \noalign{%
    \vskip\belowdisplayshortskip
    \vtop{\hsize=\linewidth#1\par
    \expandafter}%
    \expandafter\prevdepth\the\prevdepth
  }%
}

\newtheorem{lemma}{Lemma}

\renewcommand{\qedsymbol}{\Smiley[1.3]}
\newcommand*{\problem}[1]{\section*{Problem #1}}
\newcommand*{\aps}{\section*{AP Corner}}
\newcommand*{\deriv}[1][x]{\ensuremath{\dfrac{\mathrm{d}}{\mathrm{d}#1}}}
\newcommand*{\floor}[1]{\ensuremath{\lfloor #1\rfloor}}
\newcommand*{\lheqzero}{\ensuremath{\underset{\text{L'H}}{\overset{\left[\frac00\right]}{=}}}}
\newcommand*{\lheqinfty}{\ensuremath{\underset{\text{L'H}}{\overset{\left[\frac{\infty}{\infty}\right]}{=}}}}

\DeclareMathOperator{\DNE}{DNE}
\DeclareMathOperator{\sgn}{sgn}

\DeclareMathOperator{\arccsc}{arccsc}
\DeclareMathOperator{\arcsec}{arcsec}
\DeclareMathOperator{\arccot}{arccot}

\setlength{\parindent}{0pt}

%opening
\title{\vspace*{-30pt}Integral of $\sec x$ and $\sec^3 x$}
\author{Jayden Li}
\date{\today}

% \allowdisplaybreaks
\postdisplaypenalty=100000

\begin{document}
\setstretch{1.25}
\fontsize{12pt}{12pt}\selectfont
\setlength{\abovedisplayskip}{0pt}
\maketitle
\section{Integral of secant}
\begin{equation}
	\int \sec(x)\,\mathrm{d}x=\boxed{\ln|\sec x+\tan x|+C}
\end{equation}
\subsection{Using black magic}
$\begin{aligned}[t]
	\int \sec(x)\,\mathrm{d}x
	&=\int \frac{\sec(x)(\sec x+\tan x)}{\sec x+\tan x}\,\mathrm{d}x
	=\int \frac{\sec^2x+\sec x\tan x}{\sec x+\tan x}\,\mathrm{d}x \\
	&=\left[\begin{aligned}
			u&=\sec x+\tan x \\
			\mathrm{d}u&=(\sec x\tan x+\sec^2x)\,\mathrm{d}x
	\end{aligned}\right]
	=\int \frac{1}{u}\,\mathrm{d}u
	=\ln|u|
	=\boxed{\ln \left|\sec x+\tan x\right|+C}
\end{aligned}$

\subsection{Using partial fractions}
$\begin{aligned}[t]
    \int \sec(x)\,\mathrm{d}x
	&=\int \frac{\cos x}{\cos^2x}\mathrm{d}x
	=\int \frac{\cos x}{1-\sin^2x}\,\mathrm{d}x
	=\left[\begin{aligned}
			u&=\sin x \\
			\mathrm{d}u&=\cos(x)\,\mathrm{d}x
	\end{aligned}\right]
	\int \frac{1}{1-u^2}\,\mathrm{d}u
	=\int \frac{1}{(1+u)(1-u)}\,\mathrm{d}u \\
	\alignedintertext{\begin{mdframed}
		\[
			\displaystyle\frac{1}{(1+u)(1-u)}=\frac{A}{1+u}+\frac{B}{1-u}=\frac{A-Au+B+Bu}{(1+u)(1-u)}=\frac{(A+B)+(B-A)u}{(1+u)(1-u)}
		\]
		But we know there is no $u$ in the numerator, so $B-A=0\implies B=A$. Also we know $A+B=1$, so $A=1/2,B=1/2$.
	\end{mdframed}}
	&=\int\left(\frac{1/2}{1+u}+\frac{1/2}{1-u}\right)\mathrm{d}u
	=\frac12 \left(\int \frac{1}{1+u}\,\mathrm{d}u+\int \frac{1}{1-u}\,\mathrm{d}u\right)
	=\frac12 \left(\ln|1+u|-\ln|1-u|\right) \\
	&=\boxed{\frac12 \ln \left|\frac{1+\sin x}{1-\sin x}\right|+C}
	=\frac12 \ln \left|\frac{1+2\sin x+\sin^2x}{1-\sin^2x}\right|
	=\frac12 \ln \left|\frac{1+2\sin x+\sin^2x}{\cos^2x}\right| \\
	&=\frac12 \ln \left|\sec^2x+2\tan(x)\sec(x)+\tan^2x\right|
	=\frac12 \ln\left((\sec x+\tan x)^2\right)
	=\boxed{\ln\left|\sec x+\tan x\right|+C}
\end{aligned}$

\section{Integral of secant cubed}
\begin{equation}
	\int \sec^3(x)\,\mathrm{d}x=\frac12\sec(x)\tan(x)+\frac12 \int \sec(x)\,\mathrm{d}x=\boxed{\frac12\sec(x)\tan(x)+\frac12\ln|\sec x+\tan x|+C}
\end{equation}
\subsection{Using integration by parts}
$\begin{aligned}[t]
    \int \sec^3(x)\,\mathrm{d}x
	&=\left[\begin{alignedat}{2}
			u&=\sec x &\quad \mathrm{d}u&=\sec(x)\tan(x)\,\mathrm{d}x \\
			\mathrm{d}v&=\sec^2(x)\,\mathrm{d}x & v&=\tan x
	\end{alignedat}\right]\sec(x)\tan(x)-\int \sec(x)\tan^2(x)\,\mathrm{d}x \\
	&=\sec(x)\tan(x)-\int \sec(x)\left(\sec^2x-1\right)\,\mathrm{d}x
	=\sec(x)\tan(x)-\int \sec^3(x)\,\mathrm{d}x+\int \sec(x)\,\mathrm{d}x \\
	2 \int \sec^3(x)\,\mathrm{d}x&=\sec(x)\tan(x)+\ln|\sec x+\tan x| \\
	\int \sec^3(x)\,\mathrm{d}x&=\boxed{\frac12\sec(x)\tan(x)+\frac12\ln|\sec x+\tan x|+C} \\
\end{aligned}$

\subsection{Using a very bad technique}
$\begin{aligned}[t]
	\int \sec^3x\,\mathrm{d}x
	&=\int \frac{1}{\cos^3x}\,\mathrm{d}x
	=\int \frac{\cos x}{\cos^4x}\,\mathrm{d}x
	=\int \frac{\cos x}{\left(1-\sin^2x\right)^2}\,\mathrm{d}x
	=\left[ \begin{aligned}
		u&=\sin x \\				
		\mathrm{d}u&=\cos(x)\,\mathrm{d}x
	\end{aligned} \right]
	\int \frac{1}{(1-u^2)^2}\,\mathrm{d}u \\
	&=\int \left(\frac{1}{(1-u)(1+u)}\right)^2\,\mathrm{d}u
	=\int\left(\frac12\left(\frac{1}{1+u}+\frac{1}{1-u}\right)\right)^2\mathrm{d}u \\
	&=\frac14 \int \frac{1}{(1+u)^2}\,\mathrm{d}u+\frac12 \underbrace{\int \frac{1}{(1+u)(1-u)}\,\mathrm{d}u}_{\text{integral of secant}}+\frac14 \int \frac{1}{(1-u)^2}\,\mathrm{d}u \\
	&=-\frac{1}{4(1+u)}+\frac{1}{4(1-u)}+\frac12\ln|\sec x+\tan x| \\
	&=\frac14\left(\frac{1}{1-\sin x}-\frac{1}{1+\sin x}\right)+\frac12\ln|\sec x+\tan x| \\
	&=\frac14\cdot \frac{1+\sin x-(1-\sin x)}{(1-\sin x)(1+\sin x)}+\frac12\ln|\sec x+\tan x|
	=\frac{2\sin x}{4\left(1-\sin^2x\right)}+\frac12\ln|\sec x+\tan x| \\
	&=\frac12\cdot\frac{\sin x}{\cos^2x}+\frac12\ln|\sec x+\tan x|
	=\boxed{\frac12\sec(x)\tan(x)+\frac12\ln|\sec x+\tan x|+C}
\end{aligned}$

\end{document}
