% JUMP TO LINE 60, 75
\documentclass{article}
\usepackage[letterpaper,portrait,top=0.4in, left=0.6in, right=0.6in, bottom=1in]{geometry}

\usepackage{amsmath, amsfonts, amsthm, amssymb}
\usepackage{graphicx, float}
\usepackage{mathtools}
\usepackage{titlesec}
\usepackage{interval}
\usepackage{hyperref}
\usepackage{siunitx}
\usepackage{titling}
\usepackage{vwcol}
\usepackage{setspace}
\usepackage{empheq}
\usepackage{cancel}
\usepackage{esdiff}
\usepackage{multicol}
\usepackage{mdframed}
\usepackage{esdiff}
\usepackage{tikzsymbols}
\usepackage{multicol}
\usepackage{tikz}
\usepackage{varwidth}
\usepackage{parskip}
\usepackage{pgfplots}
\pgfplotsset{compat=1.18}
\intervalconfig {
	soft open fences
}

\newcommand{\alignedintertext}[1]{%
  \noalign{%
    \vskip\belowdisplayshortskip
    \vtop{\hsize=\linewidth#1\par
    \expandafter}%
    \expandafter\prevdepth\the\prevdepth
  }%
}

\newtheorem{lemma}{Lemma}

\renewcommand{\qedsymbol}{\Smiley[1.3]}
\newcommand*{\problem}[1]{\section*{Problem #1}}
\newcommand*{\aps}{\section*{AP Corner}}
\newcommand*{\deriv}[1][x]{\ensuremath{\dfrac{\mathrm{d}}{\mathrm{d}#1}}}
\newcommand*{\floor}[1]{\ensuremath{\lfloor #1\rfloor}}
\newcommand*{\lheqzero}{\ensuremath{\underset{\text{L'H}}{\overset{\left[\frac00\right]}{=}}}}
\newcommand*{\lheqinfty}{\ensuremath{\underset{\text{L'H}}{\overset{\left[\frac{\infty}{\infty}\right]}{=}}}}

\DeclareMathOperator{\DNE}{DNE}
\DeclareMathOperator{\sgn}{sgn}

\DeclareMathOperator{\arccsc}{arccsc}
\DeclareMathOperator{\arcsec}{arcsec}
\DeclareMathOperator{\arccot}{arccot}

%opening

\title{\vspace*{-40pt}Problem Set \#44}
\author{Jayden Li}
\date{\today}
% \allowdisplaybreaks
\postdisplaypenalty=100000

\begin{document}
\setstretch{1.25}
\fontsize{12pt}{12pt}\selectfont
\setlength{\abovedisplayskip}{\abovedisplayskip/2}
\setlength{\belowdisplayskip}{\belowdisplayskip/2}
\setlength{\parindent}{0pt}
\setlength{\parskip}{2ex plus 0.5ex minus 0.2ex}
\maketitle

\problem{6}
$\begin{aligned}[t]
    \int x^p\ln x\,\mathrm{d}x
	&=\left[\begin{aligned}
			u&=\ln x &\quad \mathrm{d}u&=\frac{\mathrm{d}x}{x} \\
			\mathrm{d}v&=x^p \,\mathrm{d}x & v&=\frac{x^{p+1}}{p+1}
	\end{aligned}\right]
	\frac{x^{p+1}\ln x}{p+1}-\int \frac 1x \frac{x^{p+1}}{p+1}\,\mathrm{d}x
	=\frac{x^{p+1}\ln x}{p+1}-\frac{1}{p+1}\int x^p\,\mathrm{d}x \\
	&=\frac{x^{p+1}\ln x}{p+1}-\frac{x^{p+1}}{(p+1)^2}+C \\
	\int_{0}^{1}x^p \ln x\,\mathrm{d}x
	&=\lim_{t\to0^+}\int_{t}^{1}x^p\ln x\,\mathrm{d}x
	=\lim_{t\to0^+}\left[\frac{x^{p+1}\ln x}{p+1}-\frac{x^{p+1}}{(p+1)^2}\right]_{t}^{1} \\
	&=0-\frac{1}{(p+1)^2}-\lim_{t\to0^+}\left[\frac{t^{p+1}\ln t}{p+1}+\frac{t^{p+1}}{(p+1)^2}\right]
	=-\frac{1}{(p+1)^2}-\lim_{t\to0^+}\frac{t^{p+1}\ln t}{p+1}+\frac{0}{(p+1)^2} & (1) \\
	\alignedintertext{\begin{mdframed}
			$\begin{aligned}[t]
				\lim_{t\to0^+}t^{p+1}\ln t
				&=\lim_{t\to0^+}\frac{\ln t}{t^{-p-1}}
				\underset{\text{LH}}{\overset{\left[\frac 00\right]}{=}} \lim_{t\to0^+}\left[\frac{\frac{1}{t}}{(-p-1)t^{-p-2}}\cdot\frac tt\right]
				=\lim_{t\to0^+}\frac{1}{-(p+1)t^{-p-1}}
				=\lim_{t\to0^+}\frac{-t^{p+1}}{p+1}
				=0
			\end{aligned}$
	\end{mdframed}}
	&=-\frac{1}{(p+1)^2}-0+0=\boxed{-\frac{1}{(p+1)^2}}
\end{aligned}$

Notice: when $p=-1$, the above is obviously undefined. When $p<-1$, $t^{p+1}$ is not real when $t=1$, so we are not able to evaluate it. So the integral only converges to $-1/(p+1)^2$ when $p>-1$.

\end{document}
