% JUMP TO LINE 60, 75
\documentclass[preview, margin=0.6in]{standalone}
\usepackage[letterpaper,portrait,top=0.4in, left=0.6in, right=0.6in, bottom=1in]{geometry}

\usepackage{amsmath, amsfonts, amsthm, amssymb}
\usepackage{graphicx, float}
\usepackage{mathtools}
\usepackage{titlesec}
\usepackage{interval}
\usepackage{hyperref}
\usepackage{titling}
\usepackage{vwcol}
\usepackage{setspace}
\usepackage{empheq}
\usepackage{cancel}
\usepackage{esdiff}
\usepackage{multicol}
\usepackage{mdframed}
\usepackage{esdiff}
\usepackage{tikzsymbols}
\usepackage{multicol}
\usepackage{tikz}
\usepackage{varwidth}
\usepackage{parskip}
\usepackage{pgfplots}
\usepackage[math]{cellspace}
\pgfplotsset{compat=1.18}
\intervalconfig {
	soft open fences
}

\newcommand{\alignedintertext}[1]{%
  \noalign{%
    \vtop{\hsize=\linewidth#1\par
    \expandafter}%
    \expandafter\prevdepth\the\prevdepth
  }%
}

\newtheorem{lemma}{Lemma}

\renewcommand{\qedsymbol}{\Smiley[1.3]}
\newcommand*{\problem}[1]{\section*{Problem #1}}
\newcommand*{\aps}{\section*{AP Corner}}
\newcommand*{\deriv}[1][x]{\ensuremath{\dfrac{\mathrm{d}}{\mathrm{d}#1}}}
\newcommand*{\floor}[1]{\ensuremath{\lfloor #1\rfloor}}
\newcommand*{\lheqzero}{\ensuremath{\underset{\text{L'H}}{\overset{\left[\frac00\right]}{=}}}}
\newcommand*{\lheqinfty}{\ensuremath{\underset{\text{L'H}}{\overset{\left[\frac{\infty}{\infty}\right]}{=}}}}

\DeclareMathOperator{\DNE}{DNE}
\DeclareMathOperator{\sgn}{sgn}

\DeclareMathOperator{\arccsc}{arccsc}
\DeclareMathOperator{\arcsec}{arcsec}
\DeclareMathOperator{\arccot}{arccot}

%opening

\title{\vspace*{-40pt}Problem Set \#54}
\author{Jayden Li}
\date{\today}
% \allowdisplaybreaks
\postdisplaypenalty=100000

\begin{document}
\setstretch{1.25}
\fontsize{12pt}{12pt}\selectfont
\setlength{\abovedisplayskip}{\abovedisplayskip/2}
\setlength{\belowdisplayskip}{\belowdisplayskip/2}
\setlength{\parindent}{0pt}
\setlength{\parskip}{2ex plus 0.5ex minus 0.2ex}
\setlength{\cellspacetoplimit}{8pt}
\setlength{\cellspacebottomlimit}{8pt}

\maketitle

\problem{1}
\begin{itemize}
	\item[(a)]
		$\begin{aligned}[t]
		    f(x)
			&=\sec x
			=\frac{1}{\cos x}
			=\frac{1}{\sum_{n=0}^{\infty}\frac{(-1)^n x^{2n}}{(2n)!}} \\
			P_2(x)
			&=\frac{1}{\sum_{n=0}^{1}\frac{(-1)^n x^{2n}}{(2n)!}}
			=\frac{1}{1-\frac{x^2}{2!}}
			=\boxed{1+\frac{x^2}{2}}\cancel{+\ldots}
		\end{aligned}$
\end{itemize}

\problem{2}
\begin{itemize}
	\item[(a)]
		\begin{equation*}
			f(x)=\frac 2x=2x^{-1}
		\end{equation*}

		\begin{center}
			$\begin{array}{c|cl|c|c}
				n & f^{(n)}(x) & f^{(n)}(1) & c_n \\
				\hline
				0 & 2x^{-1} & 2 & 2 \\
				1 & -2x^{-2} & -2 & -2 \\
				2 & 4x^{-3} & 4 & 4/2!=2 \\
				3 & -12x^{-4} & -12 & -12/3!=-2
			\end{array}$
		\end{center}

		\begin{equation*}
			\boxed{P_3(x)
			=2-2(x-1)+2(x-a)^2-2(x-1)^3}
		\end{equation*}

	\item[(b)]
		\begin{equation*}
			f(x)=\sqrt[3]{x}=x^{1/3}
		\end{equation*}

		\begin{center}
			$\begin{array}{c|c|c|c}
				n & f^{(n)}(x) & f^{(n)}(8) & c_n \\
				\hline
				0 & x^{1/3} & 2 & 2 \\
				1 & (1/3)x^{-2/3} & 1/12 & 1/12 \\
				2 & (-2/9)x^{-5/3} & -2/(9\cdot32) & -1/288 \\
				3 & (10/27)x^{-8/3} & 10/(27 \cdot 256) & 5/20736
			\end{array}$
		\end{center}

		\begin{equation*}
			\boxed{P_3(x)
			=2+\frac{1}{12}(x-8)-\frac{1}{288}(x-8)^2+\frac{5}{20736}(x-8)^3}
		\end{equation*}
\end{itemize}

\problem{4}
\begin{itemize}
	\item[(b)]
		Let $f(x)=\arcsin x$. We start by calculating the 5th degree Maclaurin polynomial of $f$.

		\begin{center}
			\everymath={\displaystyle}
			% \def\arraystretch{7.0}
			$\begin{array}{c|Sc|c|c}
				n & $\arcsin^{(n)}(x)$ & \arcsin^{(n)}(0) & c_n \\
				\hline
				0 & $\arcsin x$ & 0 & 0 \\
				\hline
				1 & $\frac{1}{\sqrt{1-x^2}}$ & 1 & 1 \\
				\hline
				2 & $\frac{-\frac{-2x}{2 \sqrt{1-x^2}}}{1-x^2}=\frac{x}{\left(1-x^2\right)^{3/2}}$ & 0 & 0 \\
				\hline
				3 & $\begin{aligned}
				    \frac{\left(1-x^2\right)^{3/2}-x\frac32 \sqrt{1-x^2} (-2x)}{\left(1-x^2\right)^3}
					&=\frac{\left(1-x^2\right)^{3/2}+3x^2 \left(1-x^2\right)^{1/2}}{\left(1-x^2\right)^3} \\
					&=\left(1-x^2\right)^{-3/2}+3x^2 \left(1-x^2\right)^{-5/2}
				\end{aligned}$ & 1 & \frac{1}{3!} \\
				\hline
				4 & $\begin{aligned}
					&{}-2x \frac{-3}{2}\left(1-x^2\right)^{-5/2}+6x \left(1-x^2\right)^{-5/2}+3x^2 \frac{-5}{2}(-2x)\left(1-x^2\right)^{-7/2} \\
					={}& 3x \left(1-x^2\right)^{-5/2}+6x \left(1-x^2\right)^{-5/2}+15x^3 \left(1-x^2\right)^{-7/2} \\
					={}& 9x \left(1-x^2\right)^{-5/2}+15x^3 \left(1-x^2\right)^{-7/2}
				\end{aligned}$ & 0 & 0 \\
				\hline
				5 & 
				$\begin{aligned}
					{}&\left[\begin{aligned}
						&9 \left(1-x^2\right)^{-5/2}+9x \frac{-5}{2}(-2x) \left(1-x^2\right)^{-7/2}+ \\ 
						&45x^2 \left(1-x^2\right)^{-7/2}+15x^3 \frac{-7}{2}(-2x) \left(1-x^2\right)^{-9/2}
					\end{aligned}\right] \\
					={}&\left[\begin{aligned}
						&9 \left(1-x^2\right)^{-5/2}+45x^2 \left(1-x^2\right)^{-7/2}+45x^2 \left(1-x^2\right)^{-7/2}+ \\ 
						&105x^4 \left(1-x^2\right)^{-9/2}
					\end{aligned}\right] \\
					={}& 9 \left(1-x^2\right)^{-5/2}+90x^2 \left(1-x^2\right)^{-7/2}+105x^4 \left(1-x^2\right)^{-9/2}
				\end{aligned}$
				  & 9 & \frac{9}{5!}
			\end{array}$
		\end{center}

		Let $R_3$ be the remainder of the 3rd degree Maclaurin polynomial of arcsine:
		\begin{equation*}
		    \arcsin(x)
			\approx P_3(x)
			=x-\frac{x^3}{6}
		\end{equation*}

		By Taylor's Theorem, we have the following upper bound on $R_3(0.4)$:
		\begin{equation*}
		    R_3(0.4)
			\leq \frac{\left|0-0.4\right|^{3+1}}{(3+1)!} \max_{0\leq z\leq 0.4} \left|f^{(3+1)}(z)\right|
			=\frac{(0.4)^4}{24} \max_{0\leq z\leq 0.4} \left|f^{(4)}(z)\right|
		\end{equation*}
		Notice that $\deriv f^{(4)}(x)=f^{(5)}(x)>0$ for all $0\leq x\leq0.4$, so $f^{(4)}$ is increasing. Thus, $f^{(4)}$ reaches its global maximum on the interval $[0,0.4]$ at $z=0.4$. Therefore, we have:
		\begin{equation*}
		    R_3(0.4)
			\leq\frac{(0.4)^4}{24} f^{(4)}(0.4)
			\approx\frac{0.0256}{24}\cdot7.33402
			=\boxed{0.00782}
		\end{equation*}

		The actual error is:
		\begin{equation*}
			\left|\arcsin\left(0.4\right)-\left(0.4+\frac{(0.4)^3}{6}\right)\right|=0.000850 < 0.00782 = \text{Lagrange error bound}
		\end{equation*}
\end{itemize}

\problem{5}
\begin{itemize}
\item[(b)]
	Let $f(x)=e^x$. Then $f^{n}(x)=e^x$ for all $n\in\mathbb N$.
	\begin{equation*}
		f(0.6)
		=e^{0.6}
		\approx P_n(0.6)
		=\sum_{k=0}^{n}\frac{(0.6)^k}{k!}
	\end{equation*}
	The Lagrange error bound of the above approximation, in terms of $n$, is:
	\begin{equation*}
		R_n(0.6)
		\leq \frac{\left|0-0.6\right|^{n+1}}{(n+1)!}\max_{0\leq z\leq 0.6}\left|f^{(n+1)}(z)\right|
		=\frac{(0.6)^{n+1}}{(n+1)!}e^{0.6}\leq 0.001
	\end{equation*}
	Because $f^{n+1}(z)=e^z$ is increasing, the global maximum occurs at the highest value $z=0.6$.

	Using a calculator, we see that the above is true when \boxed{n\geq5}.
\end{itemize}

\problem{6}
\begin{equation*}
	f(x)=\sqrt[3]{x}=x^{1/3}
\end{equation*}
\begin{itemize}
\item[(a)]
	\phantom{}
	\begin{center}
		\everymath={\displaystyle}
		$\begin{array}{c|Sc|c|c}
			n & $f^{(n)}(x)$ & f^{(n)}(8) & c_n \\
			\hline
			0 & $x^{1/3}$ & 2 & 2 \\
			\hline
			1 & $\frac13 x^{-2/3}$ & \frac{1}{12} & \frac{1}{12} \\
			\hline
			2 & $\frac13 \left(-\frac23\right)x^{-5/3}=-\frac{2}{9}x^{-5/3}$ & -\frac{1}{144} & -\frac{1}{288} \\
			\hline
			3 & $-\frac29 \left(-\frac53\right)x^{-8/3}=\frac{10}{27}x^{-8/3}$ & \ldots & \ldots
		\end{array}$
	\end{center}
	\begin{equation*}
	    f(x)
		\approx
		\boxed{P_2(x)=2+\frac{x-8}{12}-\frac{(x-8)^2}{288}}
	\end{equation*}

\item[(b)]
	Let $R_2$ be the remainder of the 2nd degree polynomial: $R_2(x)=\left|f(x)-P_2(x)\right|$. We calculate $R_2(x)$ where $7\leq x\leq 9$.

	Let $I$ be an interval defined as follows:
	\begin{equation*}
		I 
		=\begin{cases}
			[x,8] & 7\leq x\leq 8 \\
			[8,x] & 8<x\leq 9
		\end{cases}
	\end{equation*}
	This is the interval which $z$ must be on.
	\begin{align*}
	    R_2(x)
		&
		\leq \frac{\left|0-x\right|^{2+1}}{(2+1)!} \max_{z\in I} \left|f^{(2+1)}(z)\right|
		=\frac{x^3}{6}\max_{z\in I} \left|f^{(3)}(z)\right|
		=\frac{x^3}{6}\max_{z\in I} \left|\frac{10}{27}z^{-8/3}\right|
		\intertext{
			Notice $z^{-8/3}=1/z^{8/3}$ must be decreasing on $(0,\infty)$ because $z^{8/2}$ is increasing on the same interval. Thus, the maximum value of $z^{-8/3}$ is reached for the lowest value of $z\in I$.
		}
		&\leq\frac{x^3}{6}\frac{10}{27}\begin{cases}
			x & 7\leq x\leq8 \\
			8 & 8\leq x<9
		\end{cases}
		=\boxed{\frac{5x^3}{81}\begin{cases}
			x & 7\leq x\leq8 \\
			8 & 8\leq x<9
		\end{cases}}
	\end{align*}
\end{itemize}

\end{document}
