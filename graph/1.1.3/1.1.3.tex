% JUMP TO LINE 17, 25
% \documentclass[preview, margin=0.6in]{standalone}
\documentclass{article}
\usepackage[letterpaper,portrait,top=0.4in, left=0.6in, right=0.6in, bottom=1in]{geometry}

\usepackage{amsmath, amsfonts, amsthm, amssymb}
\usepackage{setspace}
\usepackage{parskip}

\newtheorem{thm}{Theorem}[section]
\newtheorem{defn}{Definition}[section]
\newtheorem{lemma}{Lemma}[section]
\newtheorem{corollary}{Corollary}[section]
\newtheorem{remark}{Remark}[section]
\newtheorem{example}{Example}[section]

\title{\vspace*{-40pt}Homework 1.1 (3)}
\author{Jayden Li}
\date{\today}

\begin{document}
\setstretch{1.25}
\fontsize{12pt}{12pt}\selectfont
\maketitle

\begin{enumerate}
\item[3.]
	Square matrix made up of four square block matrices, as follows:
	\begin{equation*}
		\begin{array}{c}
			\vdots \\
			\text{$m$ entries} \\
			\vdots \\
			\hline
			\vdots \\
			\text{$n$ entries} \\
			\vdots \\
		\end{array}
		\left[\begin{array}{ccc|ccc}
			0 & \ldots & 0 & 1 & \ldots & 1 \\
			\vdots & \ddots & \vdots & \vdots & \ddots & \vdots \\
			0 & \ldots & 0 & 1 & \ldots & 1 \\
			\hline
			1 & \ldots & 1 & 0 & \ldots & 0 \\
			\vdots & \ddots & \vdots & \vdots & \ddots & \vdots \\
			1 & \ldots & 1 & 0 & \ldots & 0 \\
		\end{array}\right]
	\end{equation*}

\item[12.]
	The Petersen graph is girth $5$ so there is a $5$-cycle, so it is not bipartite.

	The largest independent set is size $4$. There cannot exist an independent set of size $5$. The vertex $12$ is connected to three vertices ($34$, $35$ and $45$) and not connected to six vertices: $S=\{13,14,15,23,24,25\}$. If an independent set of size $5$ exists, then we can add four of these vertices into an independent set with $12$. But for any subset of size $4$ in $S$, at least one of the vertices is adjacent to another vertex. We know this because we can find a cycle in $S$: $13,24,15,23,14,25$. We cannot choose $4$ vertices that are not adjacent to each other because the largest independent set in the cycle is $6/2=3$.

\item[14.]
	Done in class.

\item[15.]
	Done in class.

\item[21.]
	Done in class.	

\item[22.]
	$G_1,G_2,G_5$ are isomorphic: $G_1\cong G_2\cong G_5\cong C_7$.	

	$G_3,G_4$ are isomorphic but are not isomorphic to $C_7$: $G_3\cong G_4$.

\item[24.]
	

\item[25.]
	

\item[26.]
	Done in class.

\item[29.]
	Let each person be a vertex. An ``acquaintance'' of a person is any vertex it is adjacent to. A ``stranger'' is any vertex it is not adjacent to. We are to prove any graph $G$ with six vertices contains either an independent set of size $3$ or a clique of size $3$. This occurs if and only if $\overline G$ has an independent set of size $3$ or a clique of size $3$.

	Let $u\in V(G)$. There might be $3$ or more vertices adjacent to $u$ in $G$. If not, and there are less than $3$ vertices adjacent to $u$ in $G$, then there are less than $3$ vertices not adjacent to $u$ in $\overline G$, so the degree of $u$ in $\overline G$ is greater than or equal to $3$. In either case, $u$ will have degree greater than or equal to $3$ in either $G$ or $\overline G$.

	Let us take the first case ($G$). The vertex $u$ is adjacent to at least three vertices $x,y,z$. If $(x,y)$, ($x,z$) or $(y,z)$ are adjacent, then those two vertices and $x$ form a clique. If not, then $x,y,z$ form an independent set.

	This is also true in $\overline G$.

	There is either an independent set or clique of size $3$ in $G$ or in $\overline G$. If either exists in $\overline G$, it must also exist in $G$.

	\qed
\end{enumerate}


\end{document}
