% JUMP TO LINE 60, 75
% \documentclass[preview, margin=0.6in]{standalone}
\documentclass{article}
\usepackage[letterpaper,portrait,top=0.4in, left=0.6in, right=0.6in, bottom=1in]{geometry}

\usepackage{amsmath, amsfonts, amsthm, amssymb}
\usepackage{graphicx, float}
\usepackage{centernot}
\usepackage{mathtools}
\usepackage{titlesec}
\usepackage{interval}
\usepackage{hyperref}
\usepackage{siunitx}
\usepackage{titling}
\usepackage{vwcol}
\usepackage{setspace}
\usepackage{empheq}
\usepackage{cancel}
\usepackage{esdiff}
\usepackage{multicol}
\usepackage{mdframed}
\usepackage{esdiff}
\usepackage{tikzsymbols}
\usepackage{multicol}
\usepackage{tikz}
\usepackage{varwidth}
\usepackage{parskip}
\usepackage{pgfplots}
\pgfplotsset{compat=1.18}
\intervalconfig {
	soft open fences
}

\newcommand{\alignedintertext}[1]{%
  \noalign{%
		\vskip\belowdisplayshortskip
		\vtop{\hsize=\linewidth#1\par
		\expandafter}
		\vskip\belowdisplayshortskip
		\expandafter\prevdepth\the\prevdepth
  }%
}

\newtheorem{lemma}{Lemma}

\newcommand*{\problem}[1]{\section*{Problem #1}}
\newcommand*{\aps}{\section*{AP Corner}}
\newcommand*{\deriv}[1][x]{\ensuremath{\dfrac{\mathrm{d}}{\mathrm{d}#1}}}
\newcommand*{\floor}[1]{\ensuremath{\lfloor #1\rfloor}}
\newcommand*{\lheqzero}{\ensuremath{\underset{\text{L'H}}{\overset{\left[\frac00\right]}{=}}}}
\newcommand*{\lheqinfty}{\ensuremath{\underset{\text{L'H}}{\overset{\left[\frac{\infty}{\infty}\right]}{=}}}}

\DeclareMathOperator{\DNE}{DNE}
\DeclareMathOperator{\sgn}{sgn}

\DeclareMathOperator{\arccsc}{arccsc}
\DeclareMathOperator{\arcsec}{arcsec}
\DeclareMathOperator{\arccot}{arccot}
\DeclareMathOperator{\rank}{rank}
\DeclareMathOperator{\trace}{tr}
\DeclareMathOperator{\nullity}{nullity}

%opening

\title{\vspace*{-40pt}Problem Set on Diagonalization and Similar Matrices}
\author{Jayden Li}
\date{\today}
% \allowdisplaybreaks
\postdisplaypenalty=100000

\begin{document}
\setstretch{1.25}
\fontsize{12pt}{12pt}\selectfont
\setlength{\abovedisplayskip}{\abovedisplayskip/2}
\setlength{\belowdisplayskip}{\belowdisplayskip/2}
\setlength{\parindent}{0pt}
\setlength{\parskip}{2ex plus 0.5ex minus 0.2ex}
\maketitle

\problem{1}
\begin{align*}
	\det \left(A-\lambda I\right)
	&=\det \left(PBP^{-1}-\lambda I\right)
	=\det \left(P \left(BP^{-1}-P^{-1}\lambda I\right)\right)
	=\det(P)\det \left(BP^{-1}-\lambda IP^{-1}\right) \\
	&=\det(P) \det \left( \left(B-\lambda I\right)P^{-1}\right)
	=\det(P)\det(B-\lambda I)\det \left(P^{-1}\right) \\
	&=\det(P)\det(B-\lambda I)\cdot \frac{1}{\det(P)}
	=\det(B-\lambda I)
\end{align*}

\problem{2}
If $A$ is similar to $B$, then $A=PBP^{-1}$ for some nonsingular matrix $P$. Then we have:
\begin{equation*}
	\det A=\det \left(PBP^{-1}\right)=\det(P)\det(B)\det(P^{-1})=\det(P)\det(B)\cdot\frac{1}{\det(P)}=\det B
\end{equation*}

\problem{3}
\begin{itemize}
	\item[(a)]
		Let $A,B$ be $n\times n$ matrices, and $B$ is full rank.

		\textit{Proof that $C(A)\subseteq C(AB)$.} Suppose $b\in C(A)$, then $Ax=b$ for some $x\in\mathbb R^n$. Let $y=B^{-1}x$, then $x=By$. So $A(By)=(AB)y=b$, so $b\in C(AB)$.

		\textit{Proof that $C(AB)\subseteq C(A)$.} Suppose $b\in C(AB)$, then $(AB)x=A(Bx)=b$ for some $x\in\mathbb R^n$. Since $Bx$ can be any vector in $\mathbb R^n$, let $y=Bx$, then $Ay=b \iff b\in C(A)$. 

		Because $C(AB)\subseteq C(A)$ and $C(A)\subseteq C(AB)$, we have $C(A)=C(AB)$.

	\item[(b)]
		Since $A$ and $B$ are similar, let $P$ be an $n\times n$ nonsingular matrix such that $A=PBP^{-1}$. Then $P$ and $P{-1}$ are full rank.
		\begin{align*}
			\rank(A)
			=\rank PBP^{-1}
			&=\rank \left((PB)P^{-1}\right) \\
			&=\rank(PB) & \quad \text{($P^{-1}$ is full rank)} \\
			&=\rank(B) & \text{($P$ is full rank)} \\
		\end{align*}
\end{itemize}

\problem{4}
\begin{itemize}
	\item[(a)]
	\begin{align*}
		\trace AB 
		&=\trace\begin{bmatrix}
			a_{11} & a_{12} & \cdots & a_{1n} \\
			a_{21} & a_{22} & \cdots & a_{2n} \\ 
			\vdots & \vdots & \ddots & \vdots \\
			a_{n1} & a_{n2} & \cdots & a_{nn}
		\end{bmatrix}
		\begin{bmatrix}
			b_{11} & b_{12} & \cdots & b_{1n} \\
			b_{21} & b_{22} & \cdots & b_{2n} \\ 
			\vdots & \vdots & \ddots & \vdots \\
			b_{n1} & b_{n2} & \cdots & b_{nn}
		\end{bmatrix} \\
		&=\trace\begin{bmatrix}
			a_{11}b_{11}+\ldots+a_{1n}b_{n1} & \cdots & \cdots & \cdots \\
			\cdots & a_{21}b_{12}+\ldots+a_{2n}b_{n2} & \cdots & \cdots \\
			\vdots & \vdots & \ddots &\vdots \\
			\cdots & \cdots & \cdots & a_{n1}b_{1n}+\ldots+a_{nn}b_{nn}  \\
		\end{bmatrix} \\
		&=\sum_{i=1}^{n} \left(a_{i1}b_{1i}+\ldots+a_{in}b_{ni}\right)
		=\sum_{i=1}^{n} \sum_{j=1}^{n} a_{ij}b_{ji} \\
		\\
		\trace BA
		&=\trace\begin{bmatrix}
			b_{11} & b_{12} & \cdots & b_{1n} \\
			b_{21} & b_{22} & \cdots & b_{2n} \\ 
			\vdots & \vdots & \ddots & \vdots \\
			b_{n1} & b_{n2} & \cdots & b_{nn}
		\end{bmatrix}
		\begin{bmatrix}
			a_{11} & a_{12} & \cdots & a_{1n} \\
			a_{21} & a_{22} & \cdots & a_{2n} \\ 
			\vdots & \vdots & \ddots & \vdots \\
			a_{n1} & a_{n2} & \cdots & a_{nn}
		\end{bmatrix} \\
		&=\trace\begin{bmatrix}
			b_{11}a_{11}+\ldots+b_{1n}a_{n1} & \cdots & \cdots & \cdots \\
			\cdots & b_{21}a_{12}+\ldots+b_{2n}a_{n2} & \cdots & \cdots \\
			\vdots & \vdots & \ddots &\vdots \\
			\cdots & \cdots & \cdots & b_{n1}a_{1n}+\ldots+b_{nn}a_{nn}  \\
		\end{bmatrix} \\
		&=\sum_{i=1}^{n} \left(b_{i1}a_{1i}+\ldots+b_{in}a_{ni}\right)
		=\sum_{i=1}^{n} \sum_{j=1}^{n} b_{ij}a_{ji}
		=\sum_{i=1}^{n} \sum_{j=1}^{n} a_{ij}b_{ji}
	\end{align*}
	Hence $\trace AB=\trace BA$.

	\item[(b)]
	If $A$ is similar to $B$, then $A=PBP^{-1}$ for some nonsingular matrix $P$.
	\begin{equation*}
		\trace A
		=\trace \left((PB)P^{-1}\right)
		=\trace \left(P^{-1}(PB)\right)
		=\trace \left( \left(P^{-1}P\right)B\right)
		=\trace (IB)
		=\trace B
	\end{equation*}
\end{itemize}

\problem{5}
\begin{itemize}
	\item[(a)]
		\begin{equation*}
		    \det(A)=1-4=-3 \neq 2=1-(-1)=\det(B)
		\end{equation*}
		Since the determinants of $A$ and $B$ are not equal, they must not be similar.
	
	\item[(b)]
		Notice that column 1 of $A$ equals the sum of columns 2 and 3, but columns 2 and 3 are independent. Therefore $\rank A=2$.

		In $B$, row 2 equals $-2$ times row 1, and row 3 equals $-3$ times row 1, so there is only one independent row and $\rank B=1$.

		Since $\rank A\neq \rank B$, they must not be similar.
\end{itemize}

\problem{6}
\begin{gather*}
    \rank A=2,\rank B=2 \\
	\det A=1-0=1, \det B=1-0=1 \\
	\trace A=1+1=2, \trace B=1+1=2
\end{gather*}
The characteristic polynomials of $A$ and $B$ are:
\begin{equation*}
	\left\{ \begin{aligned}
			\det(A-\lambda I)&=0 \\
			\det(I-\lambda I)&=0
	\end{aligned}\right.
	\implies
	\left\{ \begin{aligned}
			\det \begin{pmatrix}
				1-\lambda & 1 \\
				0 & 1-\lambda
			\end{pmatrix}&=0 \\
			\det \begin{pmatrix}
				1-\lambda & 0 \\
				0 & 1-\lambda
			\end{pmatrix}&=0 \\
	\end{aligned}\right.
	\implies \boxed{(1-\lambda)^2=0}
\end{equation*}
So the eigenvalues of $A$ and $I$ is $\lambda=1$ with algebraic multiplicity of $2$.

Suppose they are similar, then $A$ can be written in the form $A=PIP^{-1}$ where $P$ is a nonsingular matrix. But $PIP^{-1}=PP^{-1}=I\neq A$. So such a $P$ cannot exist and the matrices are not similar.

This example shows us that the theorem is a one-way implication, not an equivalence. It says that:
\begin{equation*}
	\text{($A$ and $B$ are similar)} \implies \text{($A$ and $B$ have the same \ldots)}
\end{equation*}
But the converse is not necessarily true:
\begin{equation*}
	\text{($A$ and $B$ are similar)} \centernot\impliedby \text{($A$ and $B$ have the same \ldots)}
\end{equation*}

\problem{7}
\begin{align*}
	&\det(A-\lambda I)
	=\det \begin{pmatrix}
		1-\lambda & -1 \\
		1 & 1-\lambda
	\end{pmatrix}
	=(1-\lambda)^2-(-1)
	=0 \\
	\implies{}& (1-\lambda)^2=-1
	\implies \lambda = 1\pm i
	\implies \begin{cases}
	    \lambda_1=1+i \\
		\lambda_2=1-i
	\end{cases}
\end{align*}

Eigenvectors corresponding to $\lambda_1=1+i$:
\begin{align*}
	&A-(1+i)I 
	= \begin{bmatrix}
		1 & -1 \\
		1 & 1
	\end{bmatrix}
	- \begin{bmatrix}
		1+i & 0 \\
		0 & 1+i
	\end{bmatrix}
	=\begin{bmatrix}
		-i & -1 \\
		1 & -i
	\end{bmatrix} \\
	\rightarrow{}& \begin{bmatrix}
		i & 1 \\ 
		0 & 0
	\end{bmatrix}
	\rightarrow \begin{bmatrix}
		1 & -i
	\end{bmatrix}
	\implies x_1= \begin{bmatrix}
	    i \\ 1
	\end{bmatrix}
\end{align*}

Eigenvectors corresponding to $\lambda_2=1-i$:
\begin{align*}
	&A-(1-i)I 
	= \begin{bmatrix}
		1 & -1 \\
		1 & 1
	\end{bmatrix}
	- \begin{bmatrix}
		1-i & 0 \\
		0 & 1-i
	\end{bmatrix}
	=\begin{bmatrix}
		i & -1 \\
		1 & i
	\end{bmatrix} \\
	\rightarrow{}& \begin{bmatrix}
		i & -1 \\ 
		0 & 0
	\end{bmatrix}
	\rightarrow \begin{bmatrix}
		1 & i
	\end{bmatrix}
	\implies x_2= \begin{bmatrix}
	    -i \\ 1
	\end{bmatrix}
\end{align*}

$A$ is diagonalizable, as the algebraic and geometric multiplicities of $\lambda_1$ and $\lambda_2$ are both $1$.

The diagonalization is $A=PDP^{-1}$, where:
\begin{equation*}
	P= \begin{bmatrix}
		i & -i \\
		1 & 1
	\end{bmatrix}
	D=\begin{bmatrix}
		1+i & 0 \\
		0 & 1-i
	\end{bmatrix}
\end{equation*}

\problem{8}
When $B=A_1^{-1}$, then $B^{-1}=A_1$.
\begin{equation*}
	B(A_1A_2)B^{-1}
	=A_1^{-1}(A_1A_2)A_1
	= \left(A_1^{-1}A_1\right)(A_2A_1)
	=A_2A_1
\end{equation*}

\problem{9}
Finding eigenvalues of $\displaystyle A=\begin{bmatrix}
	4 & 6 & 6 \\
	1 & 3 & 2 \\ 
	-1 & -5 & -2
\end{bmatrix}$:

$\begin{aligned}[t]
	\det(A-\lambda I)
	={}&\det \begin{pmatrix}
		4-\lambda & 6 & 6 \\
		1 & 3-\lambda & 2 \\
		-1 & -5 & -2-\lambda
	\end{pmatrix} \\
	={}&(4-\lambda)((3-\lambda)(-2-\lambda)-2(-5))-6(-2-\lambda-2(-1))+6(-5-(-1)(3-\lambda)) \\
	={}&(4-\lambda) \left(-6-3\lambda+2\lambda+\lambda^2+10\right)-6(-2-\lambda+2)+6(-5+3-\lambda) \\
	={}&(4-\lambda)\left(\lambda^2-\lambda+4\right)+6\lambda-12-6\lambda
	=4\lambda^2-4\lambda+16-\lambda^3+\lambda^2-4\lambda-12 \\
	={}&-\lambda^3+5\lambda^2-8\lambda+4=0 \implies \lambda^3-5\lambda^2+8\lambda-4=0 \\
	\alignedintertext{$\lambda=1$ is a solution, polynomial division with $(\lambda-1)$ gives $\lambda^2-4\lambda+4$.}
	\implies{}& (\lambda-1)\left(\lambda^2-4\lambda+4\right)
	=(\lambda-1)(\lambda-2)^2=0
\end{aligned}$

$\lambda_1=1$ has algebraic multiplicity 1. We have $1 \leq\mathrm{GM}\leq \mathrm{AM}=1$ so GM must be $1$. (Cannot have $\mathrm{GM}=0$ because there must be at least one independent eigenvector as $\det(A-I)=0\implies \nullity(A-I)>0$.)

$\lambda_2=2$ has algebraic multiplicity 1. Its eigenvectors are:
\begin{align*}
	&A-2I
	= \begin{bmatrix}
		2 & 6 & 6 \\
		1 & 1 & 2 \\ 
		-1 & -5 & -4
	\end{bmatrix}
	\rightarrow \begin{bmatrix}
		1 & 3 & 3 \\
		1 & 1 & 2 \\
		-1 & -5 & -4
	\end{bmatrix}
	\rightarrow \begin{bmatrix}
		1 & 3 & 3 \\
		0 & -2 & -1 \\
		0 & -2 & -1
	\end{bmatrix}
	\rightarrow \begin{bmatrix}
		1 & 3 & 3 \\
		0 & 1 & 1/2
	\end{bmatrix}
	\rightarrow \begin{bmatrix}
		1 & 0 & 1/2 \\
		0 & 1 & 1/2
	\end{bmatrix}
\end{align*}
From here we see that $\nullity (A-2I)=1$ so there is only $1$ unique eigenvector, and $\mathrm{GM}=1\neq2=\mathrm{AM}$.

Since the eigenvalue $\lambda_2=2$ is deficient the matrix $A$ cannot be diagonalizable.

\problem{10}
We have $Ax=\lambda x$. Then we have
\begin{equation*}
	B \left(P^{-1}x\right)
	=P^{-1}APP^{-1}x
	=P^{-1}(Ax)
	=P^{-1}(\lambda x)
	=\lambda \left(P^{-1}x\right)
\end{equation*}
So $P^{-1}x$ is an eigenvector of $B$ corresponding to the eigenvalue $\lambda$.

\problem{11}
\begin{multicols}{5}
	\begin{itemize}
		\item[(a)] True
		\item[(b)] False
		\item[(c)] False
		\item[(d)] True
		\item[(e)] False
		\item[(f)] True
		\item[(g)] True
		\item[(h)] True
		\item[(i)] True
	\end{itemize}
\end{multicols}

\problem{12}
\begin{itemize}
\item[(a)]
	$\begin{aligned}[t]
		\det(A-\lambda I)
		&=\det \begin{pmatrix}
			1-\lambda & -6 \\
			2 & -6-\lambda
		\end{pmatrix}
		=(1-\lambda)(-6-\lambda)-2(-6)
		=-6-\lambda+6\lambda+\lambda^2+12 \\
		&=\lambda^2+5\lambda+6
		=(\lambda+2)(\lambda+3)=0
		\implies \lambda_1=-2,\lambda_2=-3
	\end{aligned}$

	Eigenvectors corresponding to $\lambda_1=-2$:
	\begin{align*}
		A-(-2)I=\begin{bmatrix}
			3 & -6 \\
			2 & -4
		\end{bmatrix}
		\rightarrow \begin{bmatrix}
			1 & -2
		\end{bmatrix}
		\implies x_1=\begin{bmatrix}
			2 \\ 1
		\end{bmatrix}
	\end{align*}

	Eigenvectors corresponding to $\lambda_2=-3$:
	\begin{align*}
		A-(-3)I=\begin{bmatrix}
			4 & -6 \\
			2 & -3
		\end{bmatrix}
		\rightarrow \begin{bmatrix}
			1 & -3/2
		\end{bmatrix}
		\implies \frac12x_2=\begin{bmatrix}
			3/2 \\ 1
		\end{bmatrix}
		\implies x_2=\begin{bmatrix}
			3 \\ 2
		\end{bmatrix}
	\end{align*}

	The diagonalization $A=PDP^{-1}$ is:
	\begin{equation*}
		\boxed{P=\begin{bmatrix}
			2 & 3 \\
			1 & 2
		\end{bmatrix},
		D=\begin{bmatrix}
			-2 & 0 \\
			0 & -3
		\end{bmatrix},
		P^{-1}=\begin{bmatrix}
			2 & -3 \\
			-1 & 2
		\end{bmatrix}}
	\end{equation*}

	We can calculate $A^5$ as follows:

	$\begin{aligned}[t]
		A^5
		&=\left(PDP^{-1}\right)^5
		=PD^5P^{-1}
		=\begin{bmatrix}
			2 & 3 \\
			1 & 2
		\end{bmatrix}
		\begin{bmatrix}
			(-2)^5 & 0 \\
			0 & (-3)^5
		\end{bmatrix}
		\begin{bmatrix}
			2 & -3 \\
			-1 & 2
		\end{bmatrix} \\
		&=\begin{bmatrix}
			2 & 3 \\
			1 & 2
		\end{bmatrix}
		\begin{bmatrix}
			-32 & 0 \\
			0 & -243
		\end{bmatrix}
		\begin{bmatrix}
			2 & -3 \\
			-1 & 2
		\end{bmatrix}
		=\begin{bmatrix}
			-64 & -729 \\
			-32 & -486
		\end{bmatrix}
		\begin{bmatrix}
			2 & -3 \\
			-1 & 2
		\end{bmatrix}
		=\begin{bmatrix}
			601 & -1266 \\
			422 & -876
		\end{bmatrix}
	\end{aligned}$

\item[(b)]
	$\begin{aligned}[t]
	    \sqrt{A}
		&=A^{1/2}
		=PD^{1/2}P^{-1}
		=\begin{bmatrix}
			2 & 3 \\
			1 & 2
		\end{bmatrix}
		\begin{bmatrix}
			\sqrt{-2} & 0 \\
			0 & \sqrt{-3}
		\end{bmatrix}
		\begin{bmatrix}
			2 & -3 \\
			-1 & 2
		\end{bmatrix}
		=\begin{bmatrix}
			2 & 3 \\
			1 & 2
		\end{bmatrix}
		\begin{bmatrix}
			i \sqrt{2} & 0 \\
			0 & i \sqrt{3}
		\end{bmatrix}
		\begin{bmatrix}
			2 & -3 \\
			-1 & 2
		\end{bmatrix} \\
		&=\begin{bmatrix}
			2i \sqrt{2} & 3i \sqrt{3} \\
			i \sqrt{2} & 2i \sqrt{3}
		\end{bmatrix}
		\begin{bmatrix}
			2 & -3 \\
			-1 & 2
		\end{bmatrix}
		=\begin{bmatrix}
			4i \sqrt{2}-3i \sqrt{3} & -6i \sqrt{2}+6i \sqrt{3} \\
			2i \sqrt{2}-2i \sqrt{3} & -3i \sqrt{2}+4i \sqrt{3}
		\end{bmatrix}
	\end{aligned}$

\end{itemize}

\problem{13}
Let $P=TS^{-1}$. Then:
\begin{equation*}
	PAP^{-1}
	=\left(TS^{-1}\right)S\Lambda S^{-1} \left(ST^{-1}\right)
	=T \left(S^{-1}S\right)\Lambda \left(S^{-1}S\right)T^{-1}
	=TI\Lambda IT^{-1}
	=T\Lambda T^{-1}
	=B
\end{equation*}
So $A$ is similar to $B$.

\problem{14}
\begin{itemize}
	\item[(a)] Since $A$ is diagonalizable, let $A=PDP^{-1}$, then $A^T=\left(P^{-1}\right)^T D^T P^T=\left(P^T\right)^{-1} D^TP^T$. 

	Let $R=PP^T$ so $R^{-1}=\left(P^{T}\right)^{-1}P^{-1}$. $R$ is invertible because $P,P^T$ are full rank. Since, $D$ is diagonal, it is symmetric: $D^T=D$.
	\begin{equation*}
		RA^TR^{-1}
		=PP^T \left(P^T\right)^{-1}D^T P^T \left(P^T\right)^{-1}P^{-1}
		=PID^TIP^{-1}
		=PDP^{-1}
		=A
	\end{equation*}
	So, there exists an invertible matrix $R$ such that $A=RA^TR^{-1}$, so $A$ and $A^T$ are similar.

	\item[(b)]  \phantom{}

	\begin{mdframed}
		If two matrices are similar, then they have the same eigenvalues. Its contrapositive states that if two matrices do not have the same eigenvalues, they are not similar.

		Consider any diagonalizable matric $A$ with eigenvalues $\lambda_1,\ldots,\lambda_n$. Its inverse $A^{-1}$ has eigenvalues $\lambda_1^{-1},\ldots,\lambda_n^{-1}$. If the $\lambda$ do not equal $1$, then $A$ and $A^{-1}$ do not have the same eigenvalues, then $A$ and $A^{-1}$ are not similar.
	\end{mdframed}

	Let $A=2I$, then the eigenvalues of $A$ are $\lambda=2$ with algebraic (and geometric) multiplicity $2$.
	\begin{equation*}
		A^{-1}=\begin{bmatrix}
			2 & 0 \\
			0 & 2
		\end{bmatrix}^{-1}
		=\frac{1}{2\cdot2-0}\begin{bmatrix}
			2 & 0 \\
			0 & 2
		\end{bmatrix}
		=\frac{1}{2}\begin{bmatrix}
			1 & 0 \\
			0 & 1
		\end{bmatrix}
		=\frac12I
	\end{equation*}
	The eigenvalue of $A^{-1}=I/2$ is $1/2$ ($\mathrm{AM}=\mathrm{GM}=2$). Since the eigenvalues of $A$ and the eigenvalues of $A^{-1}$ are not the same they are not similar.
\end{itemize}

\pagebreak

\problem{15}
$\begin{aligned}[t]
    \det(A-\lambda I)
	&=\det \begin{pmatrix}
		-1-\lambda & -19 & -4 \\
		0 & -2-\lambda & 0 \\
		0 & 15 & 3-\lambda
	\end{pmatrix}
	=(-1-\lambda)((-2-\lambda)(3-\lambda)-0\cdot15) \\
	&=(-1-\lambda)(-2-\lambda)(3-\lambda)
	=0 \implies \lambda_1=-1,\lambda_2=-2,\lambda_3=3
\end{aligned}$

Eigenvectors corresponding to $\lambda_1=-1$:
\begin{align*}
    A-(-1)I
	=\begin{bmatrix}
		0 & -19 & -4 \\
		0 & -1 & 0 \\
		0 & 15 & 4
	\end{bmatrix}
	\implies x_1=\begin{bmatrix}
		1 \\ 0 \\ 0
	\end{bmatrix}
\end{align*}

Eigenvectors corresponding to $\lambda_2=-2$:
\begin{align*}
    A-(-2)I
	=\begin{bmatrix}
		1 & -19 & -4 \\
		0 & 0 & 0 \\
		0 & 15 & 5
	\end{bmatrix}
	\rightarrow \begin{bmatrix}
		1 & -19 & -4 \\
		0 & 3 & 1 \\
	\end{bmatrix}
	\rightarrow \begin{bmatrix}
		1 & 0 & 7/3 \\
		0 & 3 & 1 \\
	\end{bmatrix}
	\rightarrow \begin{bmatrix}
		1 & 0 & 7/3 \\
		0 & 1 & 1/2 \\
	\end{bmatrix}
	\implies x_2 = \begin{bmatrix}
	    -7 \\ -1 \\ 3
	\end{bmatrix}
\end{align*}

Eigenvectors corresponding to $\lambda_3=3$:
\begin{align*}
    A-3I
	=\begin{bmatrix}
		-4 & -19 & -4 \\
		0 & -5 & 0 \\
		0 & 15 & 0
	\end{bmatrix}
	\rightarrow\begin{bmatrix}
		-4 & 0 & -4 \\
		0 & -5 & 0 \\
	\end{bmatrix}
	\rightarrow\begin{bmatrix}
		1 & 0 & 1 \\
		0 & 1 & 0 \\
	\end{bmatrix}
	\implies x_3=\begin{bmatrix}
		-1 \\ 0 \\ 1
	\end{bmatrix}
\end{align*}

We have:
\begin{equation*}
    P=\begin{bmatrix}
		1 & -7 & -1 \\
		0 & -1 & 0 \\
		0 & 3 & 1
    \end{bmatrix}, 
	D=\begin{bmatrix}
		-1 & 0 & 0 \\
		0 & -2 & 0 \\
		0 & 0 & 3
	\end{bmatrix}
\end{equation*}

We calculuate $P^{-1}$ by Gaussian elimination:
\begin{align*}
	&{}
	\left[ \begin{array}{ccc|ccc}
		1 & -7 & -1 & 1 & 0 & 0 \\
		0 & -1 & 0 & 0 & 1 & 0 \\
		0 & 3 & 1 & 0 & 0 & 1
	\end{array} \right]
	\rightarrow \left[ \begin{array}{ccc|ccc}
		1 & -7 & -1 & 1 & 0 & 0 \\
		0 & 1 & 0 & 0 & -1 & 0 \\
		0 & 0 & 1 & 0 & 3 & 1
	\end{array} \right]
	\rightarrow \left[ \begin{array}{ccc|ccc}
		1 & 0 & 0 & 1 & -4 & 1 \\
		0 & 1 & 0 & 0 & -1 & 0 \\
		0 & 0 & 1 & 0 & 3 & 1
	\end{array} \right]
\end{align*}

The diagonalization is:
\begin{equation*}
	A=
	\underset{P}{\begin{bmatrix}
		1 & -7 & -1 \\
		0 & -1 & 0 \\
		0 & 3 & 1
	\end{bmatrix}}
	\underset{D}{\begin{bmatrix}
		-1 & 0 & 0 \\
		0 & -2 & 0 \\
		0 & 0 & 3
	\end{bmatrix}}
	\underset{P^{-1}}{\begin{bmatrix}
		1 & -4 & 1 \\
		0 & -1 & 0 \\
		0 & 3 & 1
	\end{bmatrix}}
\end{equation*}

\pagebreak

\problem{16}
$\begin{aligned}[t]
    \det(A-\lambda I)
	={}&\det \begin{pmatrix}
		5-\lambda & -2 \\
		1 & 3-\lambda
	\end{pmatrix}
	=(5-\lambda)(3-\lambda)-1(-2)
	=15-5\lambda-3\lambda+\lambda^2+2 \\
	={}&\lambda^2-8\lambda+17
	=(\lambda-4)^2+1=0
	\implies \lambda-4=\pm i
	\implies \lambda=4\pm i
\end{aligned}$

Eigenvectors corresponding to $\lambda_1=4+i$:
\begin{align*}
    A-(4+i)I
	={}&\begin{bmatrix}
		1-i & -2 \\
		1 & -1-i
	\end{bmatrix}
	\rightarrow\begin{bmatrix}
		1-i & -2 \\
		1-i & -(1+i)(1-i)
	\end{bmatrix}
	=\begin{bmatrix}
		1-i & -2 \\
		1-i & -2
	\end{bmatrix}
	\rightarrow\begin{bmatrix}
		1-i & -2 \\
	\end{bmatrix} \\
	\rightarrow{}& \begin{bmatrix}
		1 & -\dfrac{2}{1-i}
	\end{bmatrix}
	= \begin{bmatrix}
		1 & -\dfrac{2(1+i)}{2}
	\end{bmatrix}
	\implies x_1=\begin{bmatrix}
	    1+i \\
		1
	\end{bmatrix}
\end{align*}

Eigenvectors corresponding to $\lambda_2=4-i$:
\begin{align*}
    A-(4-i)I 
	={}& \begin{bmatrix}
		1+i & -2 \\
		1 & -1+i
	\end{bmatrix}
	\rightarrow \begin{bmatrix}
		1+i & -2 \\
		1+i & -2 \\
	\end{bmatrix}
	\rightarrow \begin{bmatrix}
		1+i & -2 \\
	\end{bmatrix}
	\rightarrow \begin{bmatrix}
		1 & -\dfrac{2}{1+i}
	\end{bmatrix} \\
	\implies{}& x_2=\begin{bmatrix}
	    2 \frac{1-i}{(1-i)(1+i)} \\
		1
	\end{bmatrix}
	=\begin{bmatrix}
	    1-i \\ 1
	\end{bmatrix}
\end{align*}

If $A=PDP^{-1}$, then:
\begin{equation*}
	P=\begin{bmatrix}
		1+i & 1-i \\
		1 & 1
	\end{bmatrix},
	D=\begin{bmatrix}
		4+i & 0 \\
		0 & 4-i
	\end{bmatrix}
\end{equation*}

($\det P=1(1+i)-1(1-i)=1+i-1+i=2i\neq0$ so $P$ is invertible.)

\end{document}

