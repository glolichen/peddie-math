% JUMP TO LINE 60, 73
\documentclass{article}
\usepackage[letterpaper,portrait,top=0.4in, left=0.6in, right=0.6in, bottom=1in]{geometry}

\usepackage{amsmath, amsfonts, amsthm, amssymb}
\usepackage{graphicx, float}
\usepackage{mathtools}
\usepackage{titlesec}
\usepackage{interval}
\usepackage{hyperref}
\usepackage{siunitx}
\usepackage{titling}
\usepackage{vwcol}
\usepackage{setspace}
\usepackage{empheq}
\usepackage{cancel}
\usepackage{esdiff}
\usepackage{multicol}
\usepackage{mdframed}
\usepackage{esdiff}
\usepackage{tikzsymbols}
\usepackage{multicol}
\usepackage{tikz}
\usepackage{varwidth}
\usepackage{pgfplots}
\pgfplotsset{compat=1.18}
\intervalconfig {
	soft open fences
}

\newcommand{\alignedintertext}[1]{%
  \noalign{%
    \vskip\belowdisplayshortskip
    \vtop{\hsize=\linewidth#1\par
    \expandafter}%
    \expandafter\prevdepth\the\prevdepth
  }%
}

\newtheorem{lemma}{Lemma}

\renewcommand{\qedsymbol}{\Smiley[1.3]}
\newcommand*{\problem}[1]{\section*{Problem #1}}
\newcommand*{\aps}{\section*{AP Corner}}
\newcommand*{\deriv}[1][x]{\ensuremath{\dfrac{\mathrm{d}}{\mathrm{d}#1}}}
\newcommand*{\floor}[1]{\ensuremath{\lfloor #1\rfloor}}
\newcommand*{\lheqzero}{\ensuremath{\underset{\text{L'H}}{\overset{\left[\frac00\right]}{=}}}}
\newcommand*{\lheqinfty}{\ensuremath{\underset{\text{L'H}}{\overset{\left[\frac{\infty}{\infty}\right]}{=}}}}

\DeclareMathOperator{\DNE}{DNE}
\DeclareMathOperator{\sgn}{sgn}

\DeclareMathOperator{\arccsc}{arccsc}
\DeclareMathOperator{\arcsec}{arcsec}
\DeclareMathOperator{\arccot}{arccot}

\setlength{\parindent}{0pt}

%opening
\title{\vspace*{-30pt}Linear Algebra 3.2}
\author{Jayden Li}
\date{\today}

% \allowdisplaybreaks
\postdisplaypenalty=100000

\begin{document}
\setstretch{1.25}
\fontsize{12pt}{12pt}\selectfont
\setlength{\abovedisplayskip}{0pt}
\maketitle
\begin{enumerate}
	\item
		If $A\mathbf x=\mathbf0$, then $EA\mathbf x=E(A\mathbf x)=E\mathbf0=\mathbf0$.

		If $EA\mathbf x=\mathbf 0$, then $E^{-1}EA \mathbf{x}=E^{-1}\mathbf{0}=IA\mathbf x=\mathbf 0$ ($E^{-1}$ must exist because $E$ is invertible)

		So $EA\mathbf x=\mathbf 0\iff A\mathbf x=\mathbf 0$, which means that $N(A)=N(EA)$.

	\item
		$\begin{aligned}[t]
			A=&\begin{bmatrix}
				1 & 2 & 1 \\
				3 & 6 & 3 \\
				4 & 8 & c
		    \end{bmatrix}\to
			\begin{bmatrix}
				1 & 2 & 1 \\
				0 & 0 & 0 \\
				0 & 0 & c-4
		    \end{bmatrix}\to
			\begin{bmatrix}
				1 & 2 & 1 \\
				0 & 0 & c-4
		    \end{bmatrix}
		\end{aligned}$
		
		If $c=4$, then $R=\begin{bmatrix}
			1 & 2 & 1 
		\end{bmatrix}$ is rank 1; column $1$ is the pivot and columns $2$ and $3$ are free.

		The special solutions are $\begin{bmatrix} -2 & 1 & 0 \end{bmatrix}^T$ and $\begin{bmatrix} -1 & 0 & 1 \end{bmatrix}^T$.

		If $c\neq4$, then $R=\begin{bmatrix}
			1 & 2 & 1 \\
			0 & 0 & c-4
		\end{bmatrix}\to
		\begin{bmatrix}
			1 & 2 & 0
		\end{bmatrix}$ is rank 2; columns $1$ and $3$ are the pivots.

		The special solution is $\begin{bmatrix} -2 & 1 & 0\end{bmatrix}^T$.

		\hrulefill

		If $c\neq0$, $B=\begin{bmatrix}
			c & c \\ c & c
		\end{bmatrix}\to \begin{bmatrix}
			c & c \\ 0 & 0
		\end{bmatrix}\to \begin{bmatrix}
			1 & 1
		\end{bmatrix}$ is rank 1; column $1$ is the pivot and column $2$ is free.

		The special solution is $\begin{bmatrix}
			-1 & 1
		\end{bmatrix}^T$.

		If $c=0$, then $B=\begin{bmatrix}
			0 & 0 \\ 0 & 0
		\end{bmatrix}$ is rank 0; both columns are free.

		The special solutions are $\begin{bmatrix}
			 0 & 1
		\end{bmatrix}^T$ and $\begin{bmatrix}
			 1 & 0
		\end{bmatrix}^T$.

	\item 
		$\begin{aligned}[t]
			S&=\begin{bmatrix}
				s_1 & s_2
			\end{bmatrix}=\begin{bmatrix}
				-3 & -2 \\
				1 & 0 \\
				0 & -6 \\
				0 & 1
			\end{bmatrix}=\begin{bmatrix}
				1 & 0 & 0 & 0 \\
				0 & 0 & 1 & 0 \\
				0 & 1 & 0 & 0 \\
				0 & 0 & 0 & 1
			\end{bmatrix}\begin{bmatrix}
				-3 & -2 \\
				0 & -6 \\
				1 & 0 \\
				0 & 1
			\end{bmatrix}=P^T\begin{bmatrix}
				-F \\ I
			\end{bmatrix} \\
			R&=\begin{bmatrix}
				I & F
			\end{bmatrix}P=\begin{bmatrix}
				1 & 0 & 3 & 2 \\
				0 & 1 & 0 & 6
			\end{bmatrix}\begin{bmatrix}
				1 & 0 & 0 & 0 \\
				0 & 0 & 1 & 0 \\
				0 & 1 & 0 & 0 \\
				0 & 0 & 0 & 1
			\end{bmatrix}=\begin{bmatrix}
				1 & 3 & 0 & 2 \\
				0 & 0 & 1 & 6
			\end{bmatrix} \\
			N(A)&=\left\{\left.cs_1+ds_2\right|c,d\in\mathbb R\right\}
			=\left\{\left.c \begin{bmatrix}
			    -3 \\ 1 \\ 0 \\ 0
			\end{bmatrix}+d \begin{bmatrix}
				-2 \\ 0 \\ -6 \\ 1
			\end{bmatrix}\right|c,d\in\mathbb R\right\}
		\end{aligned}$
	
	\item
		$\begin{aligned}[t]
			A&=\begin{bmatrix}
				1 & 2 & 2 & 4 & 6 \\
				1 & 2 & 3 & 6 & 9 \\
				0 & 0 & 1 & 2 & 3
			\end{bmatrix}\to
			\begin{bmatrix}
				1 & 2 & 2 & 4 & 6 \\
				0 & 0 & -1 & -2 & -3 \\
				0 & 0 & 1 & 2 & 3
			\end{bmatrix}\to
			\begin{bmatrix}
				1 & 2 & 2 & 4 & 6 \\
				0 & 0 & 1 & 2 & 3 \\
				0 & 0 & 0 & 0 & 0
			\end{bmatrix}\to
			\begin{bmatrix}
				1 & 2 & 0 & 0 & 0 \\
				0 & 0 & 1 & 2 & 3
			\end{bmatrix} \\
			A&=CR=\begin{bmatrix}
				1 & 2 \\
				1 & 3 \\
				0 & 1
			\end{bmatrix}
			\begin{bmatrix}
				1 & 2 & 0 & 0 & 0 \\
				0 & 0 & 1 & 2 & 3
			\end{bmatrix} \\
			B&=\begin{bmatrix}
				2 & 4 & 2 \\
				0 & 4 & 4 \\
				0 & 8 & 8
			\end{bmatrix}\to
			\begin{bmatrix}
				1 & 2 & 1 \\
				0 & 1 & 1 \\
				0 & 0 & 0
			\end{bmatrix}\to
			\begin{bmatrix}
				1 & 0 & -1 \\
				0 & 1 & 1
			\end{bmatrix} \\
			B&=CR=\begin{bmatrix}
				2 & 4 \\
				0 & 4 \\
				0 & 8
			\end{bmatrix}
			\begin{bmatrix}
				1 & 0 & -1 \\
				0 & 1 & 1
			\end{bmatrix}
		\end{aligned}$

	\item 
		Columns 1 and 3 ($x_1,x_3$) are pivots, columns 2, 4 and 5 ($x_2,x_4,x_5$) are free.

		When $x_2=1,x_4=0,x_5=0$, the special solution is $\begin{bmatrix}
			-2 & 1 & 0 & 0 & 0
		\end{bmatrix}^T$.

		When $x_2=0,x_4=1,x_5=0$, the special solution is $\begin{bmatrix}
			0 & 0 & -2 & 1 & 0
		\end{bmatrix}^T$.

		When $x_2=0,x_4=0,x_5=1$, the special solution is $\begin{bmatrix}
			0 & 0 & -3 & 0 & 1
		\end{bmatrix}^T$.
	
	\item
	\begin{itemize}
		\item[(a)] False. In $A=\begin{bmatrix}
				0 & 0 & 0 \\
				0 & 0 & 0 \\
				0 & 0 & 0
		\end{bmatrix}$, every variable is free.
		\item[(b)] True. An invertible matrix must be square and nonsingular. For a nonsingular square matrix $A$, then the only solution to $A \mathbf{x}=0$ is $\mathbf{x}=\mathbf{0}$, so there are no free variables.
		\item[(c)] True. Rank of matrix is at most $\min(m,n)$. In any case, $\text{rank}\leq\min(m,n)\leq n$, so $\text{rank}\leq n$.
		\item[(d)] True. Rank of matrix is at most $\min(m,n)$. In any case, $\text{rank}\leq\min(m,n)\leq m$, so $\text{rank}\leq m$.
	\end{itemize}
\end{enumerate}


\end{document}
