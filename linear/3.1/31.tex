% JUMP TO LINE 60, 73
% \documentclass[preview, margin=0.6in]{standaloen}
\documentclass{article}
\usepackage[letterpaper,portrait,top=0.4in, left=0.6in, right=0.6in, bottom=1in]{geometry}

\usepackage{amsmath, amsfonts, amsthm, amssymb}
\usepackage{graphicx, float}
\usepackage{mathtools}
\usepackage{titlesec}
\usepackage{interval}
\usepackage{hyperref}
\usepackage{siunitx}
\usepackage{titling}
\usepackage{vwcol}
\usepackage{setspace}
\usepackage{empheq}
\usepackage{cancel}
\usepackage{esdiff}
\usepackage{multicol}
\usepackage{mdframed}
\usepackage{esdiff}
\usepackage{tikzsymbols}
\usepackage{multicol}
\usepackage{tikz}
\usepackage{varwidth}
\usepackage{pgfplots}
\pgfplotsset{compat=1.18}
\intervalconfig {
	soft open fences
}

\newcommand{\alignedintertext}[1]{%
  \noalign{%
    \vskip\belowdisplayshortskip
    \vtop{\hsize=\linewidth#1\par
    \expandafter}%
    \expandafter\prevdepth\the\prevdepth
  }%
}

\newtheorem{lemma}{Lemma}

\renewcommand{\qedsymbol}{\Smiley[1.3]}
\newcommand*{\problem}[1]{\section*{Problem #1}}
\newcommand*{\aps}{\section*{AP Corner}}
\newcommand*{\deriv}[1][x]{\ensuremath{\dfrac{\mathrm{d}}{\mathrm{d}#1}}}
\newcommand*{\floor}[1]{\ensuremath{\lfloor #1\rfloor}}
\newcommand*{\lheqzero}{\ensuremath{\underset{\text{L'H}}{\overset{\left[\frac00\right]}{=}}}}
\newcommand*{\lheqinfty}{\ensuremath{\underset{\text{L'H}}{\overset{\left[\frac{\infty}{\infty}\right]}{=}}}}

\DeclareMathOperator{\DNE}{DNE}
\DeclareMathOperator{\sgn}{sgn}

\DeclareMathOperator{\arccsc}{arccsc}
\DeclareMathOperator{\arcsec}{arcsec}
\DeclareMathOperator{\arccot}{arccot}

\setlength{\parindent}{0pt}

%opening
\title{\vspace*{-30pt}Linear Algebra 3.1}
\author{Jayden Li}
\date{\today}

% \allowdisplaybreaks
\postdisplaypenalty=100000

\begin{document}
\setstretch{1.25}
\fontsize{12pt}{12pt}\selectfont
\setlength{\abovedisplayskip}{0pt}
\maketitle
\begin{enumerate}
	\item[2.] No; multiplication by scalar $1$ is not the identity operation. $1\cdot(a,b)=(a,0)\neq(a,b)$.
	\item[3.] \begin{itemize}
		\item[(a)] Not closed under scalar multiplication. $-1\cdot(2)=(-2)\not\in\mathrm V$.
		\item[(b)]
			$c(\mathbf x+\mathbf y)=3(\mathbf 2+\mathbf 1)=3\cdot\mathbf2=2^3=8\overset?=c\mathbf x+c\mathbf y=3\cdot\mathbf 2+3\cdot\mathbf 1=\mathbf8+\mathbf1=8\cdot1=8$. It works.

			``Zero vector'' is the scalar value $1$.
	\end{itemize}
	\item[4.]
		Zero vector is $\begin{bmatrix}
			0 & 0 \\
			0 & 0
		\end{bmatrix}$.
		$\dfrac12A=\begin{bmatrix}
			1 & -1 \\
			1 & -1
		\end{bmatrix}$.
		$-A=\begin{bmatrix}
			-2 & 2 \\
			-2 & 2
		\end{bmatrix}$.

		Smallest subspace containing $A$ is
		$\left\{\left.a\cdot\begin{bmatrix}
			-1 & 1 \\
			-1 & 1
		\end{bmatrix}\right|a\in\mathbb R\right\}$.
	\item[5.]
		\begin{itemize}
			\item[(a)]
				$\left\{\left.a\cdot\begin{bmatrix}
					1 & 0 \\
					0 & 0
				\end{bmatrix}\right|a\in\mathbb R\right\}$.

			\item[(b)]
				Yes. Any vector space containing both $A$ and $B$ must be closed under linear combination.

				\[\underbrace{1\cdot A+(-1)\cdot B}_{\text{linear combination}}=1\cdot\begin{bmatrix}
					1 & 0 \\
					0 & 0
				\end{bmatrix}+(-1)\cdot\begin{bmatrix}
					0 & 0 \\
					0 & -1
				\end{bmatrix}=\begin{bmatrix}
					1 & 0 \\ 
					0 & 1
				\end{bmatrix}=I\]

				Because the set is closed, $I$ must be in it.

			\item[(c)]
				$\left\{\begin{bmatrix}
					0 & 0 \\ 
					0 & 0
				\end{bmatrix}\right\}$
		\end{itemize}

	\item[8.]
		\begin{itemize}
			\item[(a)]
				$\left\{\left.\begin{bmatrix}
							a \\ b
				\end{bmatrix}\right|a,b\in\mathbb Z\right\}$

			\item[(b)]
				$\left\{\left.\begin{bmatrix}
							a \\ b
				\end{bmatrix}\right|a\in\mathbb R,b\in\mathbb R\setminus\{0\}\right\}$

				(product of two nonzero real numbers cannot be zero, but their sum could be)
		\end{itemize}

	\item[9.]
		\begin{itemize}
			\item[(a)] Yes.	$r(a_1,a_2,a_3)+s(b_1,b_2,b_3)=(ra_1+sb_1,ra_1+sb_1,ra_3+sb_3)$.

			\item[(b)] No. Not closed under addition. $(1,0,0)+(1,0,0)=(2,0,0)$, $2\neq1$.

			\item[(c)] No. Not closed under addition. $(1,1,0)+(0,1,1)=(1,2,1)$, $(1)(2)(1)=2\neq0$.

			\item[(d)] Yes. Span is a vector space.

			\item[(e)] Yes. The plane passes through the origin.

			\item[(f)] No. Not closed under scalar multiplication. $-1\cdot(1,2,3)=(-1,-2,-3)$, $-1\not\leq-2$.

		\end{itemize}

	
	\item[10.]
		\begin{itemize}
			\item[(a)]
				$\left\{\left.\begin{bmatrix}
							a & b \\ 0 & 0
				\end{bmatrix}\right|a,b\in\mathbb R\right\}$

			\item[(b)]
				$\left\{\left.\begin{bmatrix}
							a & a \\ 0 & 0
				\end{bmatrix}\right|a\in\mathbb R\right\}$
				
			\item[(c)]
				$\left\{\left.\begin{bmatrix}
							a & 0 \\ 0 & b
				\end{bmatrix}\right|a,b\in\mathbb R\right\}$
		\end{itemize}
	
	\item[11.]
		$(1,1,-1)\in P$ ($1+1+(-2)(-1)=1+1+2=4$)

		$(4,0,0)\in P$ ($4+0+0=4$)

		$(1,1,-1)+(4,0,0)=(5,1,-1)\not\in P$ ($5+1+(-2)(-1)=5+1+2=8\neq4$)

	\item[12.]
		\boxed{\left\{\left.(x,y,z)\right|x+y-2z=0\right\}}

		$(1,1,1)\in P$ ($1+1+(-2)(1)=1+1-2=0$)

		$(4,-4,0)\in P$ ($4+(-4)+0=0$)

		$(1,1,1)+(4,-4,0)=(5,-3,1)\not\in P$ ($5+(-3)+(-2)(1)=5-3-2=0$)
	
	\item[14.]
		\begin{itemize}
		    \item[(a)]
				$\begin{bmatrix}
					1 & 0 \\ 
					0 & 1
				\end{bmatrix}+\begin{bmatrix}
					0 & 1 \\
					1 & 0
				\end{bmatrix}=\begin{bmatrix}
					1 & 1 \\ 
					1 & 1
				\end{bmatrix}$ 
				which is singular and not invertible.

		    \item[(b)]
				$\begin{bmatrix}
					1 & 0 \\ 
					0 & 0
				\end{bmatrix}+\begin{bmatrix}
					0 & 0 \\
					0 & 1
				\end{bmatrix}=\begin{bmatrix}
					1 & 0 \\ 
					0 & 1
				\end{bmatrix}$ 
				which is nonsingular.
		\end{itemize}
	
	\item[15.]
		\begin{itemize}
			\item[(a)]
				True.
				$\begin{bmatrix}
					1 & 2 & 3 \\
					2 & 4 & 5 \\
					3 & 5 & 6
				\end{bmatrix}+\begin{bmatrix}
					0 & 2 & 4 \\
					2 & 4 & 9 \\
					4 & 9 & 1
				\end{bmatrix}=\begin{bmatrix}
					1 & 4 & 7 \\
					4 & 8 & 14 \\
					7 & 14 & 7
				\end{bmatrix}=\begin{bmatrix}
					1 & 4 & 7 \\
					4 & 8 & 14 \\
					7 & 14 & 7
				\end{bmatrix}^T$ 

			\item[(b)]
				True.
				$\begin{bmatrix}
					0 & 2 & 3 \\
					-2 & 0 & 5 \\
					-3 & -5 & 0
				\end{bmatrix}+\begin{bmatrix}
					0 & 2 & 4 \\
					-2 & 0 & 9 \\
					-4 & -9 & 0
				\end{bmatrix}=\begin{bmatrix}
					0 & 4 & 7 \\
					-4 & 0 & 14 \\
					-7 & -14 & 0
				\end{bmatrix}=-\left(\begin{bmatrix}
					0 & 4 & 7 \\
					-4 & 0 & 14 \\
					-7 & -14 & 0
				\end{bmatrix}\right)^T$ 

			\item[(c)]
				True.
				$\begin{bmatrix}
					1 & 2 & 3 \\
					4 & 5 & 6 \\
					7 & 8 & 9
				\end{bmatrix}+\begin{bmatrix}
					2 & 4 & 6 \\
					1 & 3 & 5 \\
					0 & 2 & 4
				\end{bmatrix}=\begin{bmatrix}
					3 & 6 & 9 \\
					5 & 8 & 11 \\
					7 & 10 & 13
				\end{bmatrix}$ 
				which is not symmetric.
		\end{itemize}
	
\end{enumerate}


\end{document}
